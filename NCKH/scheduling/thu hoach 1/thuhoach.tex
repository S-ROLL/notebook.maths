\documentclass[12pt,a4paper]{report}
\usepackage[utf8]{vietnam}\usepackage{amsmath, amsthm, amssymb,latexsym,amscd,amsfonts,enumerate}
\usepackage[top=3.5cm, bottom=3.0cm, left=3cm, right=3.0cm]{geometry} 
\usepackage{color, fancyhdr, graphicx, wrapfig}
\usepackage[unicode]{hyperref}
\usepackage[vietnamese]{babel}
\usepackage{titling}
\usepackage{amsmath, amsthm, amssymb,latexsym,amscd,amsfonts,enumerate}
\usepackage{subfigure}
\usepackage{secdot}
\usepackage{graphicx}
\usepackage{booktabs}
\usepackage{pgfgantt}
\usepackage{amsthm}
\usepackage{amsmath}
\usepackage{amsfonts}
\usepackage{amssymb}
\usepackage{graphicx} 
\usepackage{titling}
\usepackage{secdot}
\usepackage{enumitem}
\usepackage{tikz}
\usepackage{array}
\usetikzlibrary{calc}
\usepackage{longtable}
\usepackage{indentfirst}
\usepackage{fancyhdr}
\usepackage{exscale,relsize,makeidx}
\usepackage{color, fancyhdr, graphicx, wrapfig}
\graphicspath{ {figures/} }

\renewcommand{\listfigurename}{List of plots}
\renewcommand{\listtablename}{List of Tables}
\usepackage{amsmath}
\usepackage{textpos}
\usepackage{pgfplots}
\usepackage{tikz}
\usepackage{hyperref}
\usepackage{caption}
\usetikzlibrary{shapes.geometric, arrows}
\usetikzlibrary{datavisualization} 
\pgfplotsset{compat=1.18, width = 7cm}
\usetikzlibrary{patterns}
\usepackage{enumitem}
\usepackage{array}
\usepackage[tikz]{ocgx2}
\usepackage{xcolor}
\usepackage{blindtext}
\usepackage{multicol}
\usepackage{tikz}
\usepackage{subcaption}
\usepackage{changepage}
\usepackage{float}
\usepackage{pgfplotstable}
\usepackage{pgfplots}
\usepackage{blindtext}
\usepackage{titlesec}
\usepackage{mathtools}
\usepackage{tabularx}
\usepackage{nccmath}
\usetikzlibrary{calc}
\usepackage{longtable}
\usepackage{indentfirst}
\usepackage{fancyhdr}
\usepackage{exscale,relsize,makeidx}
\setcounter{tocdepth}{4}
\setcounter{secnumdepth}{4}
\newtheorem{dn}{Định nghĩa}
\newtheorem{tc}{Tính chất}
\newtheorem{dl}{Định lý}
\newtheorem{md}{Mệnh đề}
\newtheorem{bd}{Bổ đề}
\newtheorem{hq}{Hệ quả}
\newtheorem{nx}{Nhận xét}
\newtheorem{vd}{Ví dụ}
\newtheorem{cm}{Chứng minh}
\newtheorem{cy}{Chú ý}
\newtheorem{ttoan}{Thuật toán}
\pagenumbering{roman}\pagestyle{plain}
\renewcommand{\headrulewidth}{1,2pt} 			
\renewcommand{\footrulewidth}{1,2pt}
\newcommand{\dstc}[2]
{
	\newdimen\stringwidth\setbox0=\hbox{#1}
	\stringwidth=\wd0
	\hspace*{-\parindent}\hspace*{.5\textwidth}\hspace*{-.5\wd0}#1\hfill #2\bigskip
}  
\usepackage{scrextend}
\fancyhf{}
\lhead{}
\chead{\thepage}
\rhead{}
\cfoot{}
\rfoot{}
\lfoot{}
\pagestyle{fancy}
\renewcommand{\headrulewidth}{1pt}
\begin{document} 
    \begin{titlepage}
	\centering
    \phantom{}\par
	\vspace{3cm}
	{\LARGE\textbf{BÀI BÁO CÁO THU HOẠCH}\par}
	\vspace{1cm}
	\rule{5cm}{0.5pt}\par
	\vspace{1cm}
		{\LARGE\textbf{SINGLE MACHINE SCHEDULING}\par}
	\vspace{1cm}
	\Large\textbf{Nguyễn Chí Bằng}\par		
	\vspace{1cm}
    \today
    \end{titlepage}

	\addcontentsline{toc}{chapter}{Mục lục}
	\tableofcontents

	\chapter*{Danh mục các kí hiệu}
	\thispagestyle{fancy}
	\addcontentsline{toc}{chapter}{Danh mục các kí hiệu}
	\begin{longtable}{l l}
		$\mathbb{R}$ & Tập các số thực.\\
        $\mathbb{Z}$ & Tập số nguyên. \\
		$\emptyset $ & Tập rỗng.\\
		$\mathbb{R}^n$ & Không gian Euclide $n$-chiều.\\
		$F$ & Tập chấp nhận được của bài toán tối ưu.\\
		$\langle u, v \rangle$ & Tích vô hướng của hai véc tơ $u$ và $v$ trong $\mathbb{R}^n$.\\
        $A$ & Ma trận thực cỡ $m \times n$, ta có thể ký hiệu $A = (a_{ij})_{m \times n}$, \\
        & trong đó $a_{ij}$ là phần tử nằm trên hàng $i$ cột $j$. \\
        $A^T$ & Ma trận chuyển vị của $A$, cỡ $n \times m$, ký hiệu $A^T = (a'_{ij})$, \\
        & trong đó $a'_{ij} = a_{ji}$, nhận được từ $A$ bằng cách viết các hàng \\
        & (tương ứng, các cột) của $A$ thành các cột (tương ứng, các hàng) \\
        & của $A^T$ tương ứng. \\
        $I_n$ & Ma trận đơn vị cấp $n$, trong đó ma trận đơn vị là ma trận $A$ \\
        & với $a_{ij}=0, i \neq j$ và $a_{ij}=1, i = j$. \\
        $A^{-1}$ & Nếu tồn tại ma trận vuông $B$ cấp $n$ sao cho $AB=BA=I_n$ thì \\
        & $B$ được gọi là ma trận nghịch đảo của $A$, ta ký hiệu $B = A^{-1}$.\\
        $Ax \leq b$ & Hệ bất phương trình tuyến tính. \\
        $\lfloor a \rfloor$ & Phần nguyên nhỏ nhất của $a$. \\
        $\lceil a \rceil$ & Phần nguyên lớn nhất của $a$. \\
	\end{longtable}
\newpage

\chapter{Giới thiệu}
\section{Vấn đề}
\section{Ví dụ minh hoạ}
\chapter{Phương pháp}
\section{EDD}
\subsection{Lmax}
\subsection{Tmax}
\section{SPT}
\subsection{1}
\subsection{tong quat}
\section{release date}
\subsection{preemptive}
\subsection{non-preemptive}
\chapter{Kết luận}
\end{document}