\documentclass[12pt,a4paper]{report}
\usepackage[utf8]{vietnam}\usepackage{amsmath, amsthm, amssymb,latexsym,amscd,amsfonts,enumerate}
\usepackage[top=3.5cm, bottom=3.0cm, left=3cm, right=3.0cm]{geometry} 
\usepackage{color, fancyhdr, graphicx, wrapfig}
\usepackage[unicode]{hyperref}
\usepackage[vietnamese]{babel}
\usepackage{amsthm}
\usepackage{amsmath}
\usepackage{amsfonts}
\usepackage{amssymb}
\usepackage{graphicx} 
\usepackage{titling}
\usepackage{secdot}
\usepackage{enumitem}
\usepackage{tikz}
\usepackage{array}
\usetikzlibrary{calc}
\usepackage{longtable}
\usepackage{indentfirst}
\usepackage{fancyhdr}
\usepackage{exscale,relsize,makeidx}
%\usepackage{refcheck}
\setcounter{tocdepth}{4}
\setcounter{secnumdepth}{4}
\newtheorem{dn}{Định nghĩa}[chapter]
\newtheorem{tc}{Tính chất}[chapter]
\newtheorem{dl}{Định lý}[chapter]
\newtheorem{md}{Mệnh đề}[chapter]
\newtheorem{bd}{Bổ đề}[chapter]
\newtheorem{hq}{Hệ quả}[chapter]
\newtheorem{nx}{Nhận xét}[chapter]
\newtheorem{vd}{Ví dụ}[chapter]
\newtheorem{cy}{Chú ý}[chapter]
\pagenumbering{roman}\pagestyle{plain}
%\pagestyle{fancy}
%\lhead{\it \changefontsizes{11pt}Luận văn thạc sĩ:}
%\rhead{\it Một số phương pháp vô hướng hóa cơ bản trong tối ưu đa mục tiêu}
%\lfoot{\it Nguyễn Văn Vân } 			         
%\rfoot{\it K19.2 trường ĐHSG}
\renewcommand{\headrulewidth}{1,2pt} 			
\renewcommand{\footrulewidth}{1,2pt}
\newcommand{\dstc}[2]
{
	\newdimen\stringwidth\setbox0=\hbox{#1}
	\stringwidth=\wd0
	\hspace*{-\parindent}\hspace*{.5\textwidth}\hspace*{-.5\wd0}#1\hfill #2\bigskip
	
}  
\usepackage{scrextend}
\fancyhf{}
\lhead{}
\chead{\thepage}
\rhead{}
\cfoot{}
\rfoot{}
\lfoot{}
\pagestyle{fancy}
\renewcommand{\headrulewidth}{1pt}
\begin{document} 

\subsection{Bài toán dạng chính tắc và chuẩn tắc}
\begin{itemize}
    \item \textbf{Bài toán dạng chính tắc}
    
    Ta có bài toán dạng:
    \begin{equation} \small \label{F}
        \begin{split}
        f(x) & = c^Tx \quad \longrightarrow \text{Min} \\
            & \left\{
            \begin{split}
            & \sum _{j=1}^n a_{ij} x_j = b_i, \: i=1,2,\ldots,m \\
            & x_j \geq 0, \: j=1,2,\ldots,n.
            \end{split}
            \right.    
        \end{split}
    \end{equation}

    Thì ta gọi bài toán trên có dạng chính tắc và có thể viết lại dưới dạng:

    \begin{equation} \small \label{chinhtac}
        \begin{split}
        (P) \quad \text{Min } & f(x) = c^Tx \\
            & \left\{
            \begin{split}
            & Ax=b, \\
            & x_j \geq 0.
            \end{split}
            \right.    
        \end{split}
    \end{equation}

    Trong đó $A$ là ma trận $m\times n$, $b=\begin{pmatrix}
        b_1 \\
        b_2 \\
        \vdots \\
        b_m
        \end{pmatrix}$ và $c^T=(c_1 \: c_2 \: \ldots \: c_n)$.

    \item \textbf{Bài toán dạng chuẩn tắc}
    
    Nếu bài toán có dạng:

    \begin{equation} \small \label{F}
        \begin{split}
        f(x) & = c^Tx \quad \longrightarrow \text{Min} \\
            & \left\{
            \begin{split}
            & \sum _{j=1}^n a_{ij} x_j \geq b_i, \: i=1,2,\ldots,m \\
            & x_j \geq 0, \: j=1,2,\ldots,n.
            \end{split}
            \right.    
        \end{split}
    \end{equation}

    Thì ta gọi bài toán trên có dạng chuẩn tắc và có thể viết lại dưới dạng:

    \begin{equation} \small \label{F}
        \begin{split}
        (P) \quad \text{Min } & f(x) = c^Tx \\
            & \left\{
            \begin{split}
            & Ax \geq b, \\
            & x_j \geq 0.
            \end{split}
            \right.    
        \end{split}
    \end{equation}

    Tương tự $A$ là ma trận $m\times n$, $b=\begin{pmatrix}
        b_1 \\
        b_2 \\
        \vdots \\
        b_m
        \end{pmatrix}$ và $c^T=(c_1 \: c_2 \: \ldots \: c_n)$.

\end{itemize}

\subsection{Chuyển bài toán về dạng chính tắc}

Để thuận tiện, ta chỉ xét dạng bài toán quy hoạch tuyến tính tổng quát là \textbf{dạng chính tắc} và bất kỳ bài toán nào cũng có thể đưa về dạng chính tắc.

\begin{itemize}
\item \textbf{Phương pháp đưa về dạng chính tắc:}
    \begin{itemize}
    \item Bài toán $\max f(x) \longrightarrow -\min [-f(x)]$.
    \item Bằng cách thêm ẩn phụ $x_{n+i}$ tương ứng có hệ số trong hàm mục tiêu là $c_{n+i}=0$, ta có thể đưa bất đẳng thức 
    \begin{equation*}
    \sum _{j=1}^n a_{ij} x_j \geq b_i
    \end{equation*}
    hoặc
    \begin{equation*}
    \sum _{j=1}^n a_{ij} x_j \leq b_i
    \end{equation*}
    lần lượt thành đẳng thức
    \begin{equation*}
    \sum _{j=1}^n a_{ij} x_j - x_{n+i} = b_i
    \end{equation*}
    hoặc
    \begin{equation*}
    \sum _{j=1}^n a_{ij} x_j + x_{n+i} = b_i
    \end{equation*}
    \item Nếu tồn tại bất kỳ $x_k$ không có ràng buộc thì ta viết
    \begin{equation*}
    x_k = x_k^{'} - x_k^{''}
    \end{equation*}
    với $x_k^{'} \geq 0$ và $x_k^{''} \geq 0$.
    \end{itemize}
\end{itemize}
Kể từ đây, ta chỉ quan tâm bài toán \eqref{chinhtac}.

\end{document}