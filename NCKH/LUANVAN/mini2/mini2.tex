\documentclass[12pt,a4paper]{report}
\usepackage[utf8]{vietnam}\usepackage{amsmath, amsthm, amssymb,latexsym,amscd,amsfonts,enumerate}
\usepackage[top=3.5cm, bottom=3.0cm, left=3cm, right=3.0cm]{geometry} 
\usepackage{color, fancyhdr, graphicx, wrapfig}
\usepackage[unicode]{hyperref}
\usepackage[vietnamese]{babel}
\usepackage{titling}
\usepackage{subfigure}
\usepackage{secdot}
\usepackage{enumitem}
\usepackage{array}
\usepackage[tikz]{ocgx2}
\usepackage{xcolor}
\usepackage{blindtext}
\usepackage{multicol}
\usepackage{tikz}
\usepackage{subcaption}
\usepackage{changepage}
\usepackage{float}
\usepackage{pgfplotstable}
\usepackage{pgfplots}
\usepackage{blindtext}
\usepackage{titlesec}
\usepackage{mathtools}
\usepackage{tabularx}
\usepackage{nccmath}
\usetikzlibrary{calc}
\usepackage{longtable}
\usepackage{indentfirst}
\usepackage{fancyhdr}
\usepackage{exscale,relsize,makeidx}
%\usepackage{refcheck}
\setcounter{tocdepth}{4}
\setcounter{secnumdepth}{4}
\newtheorem{dn}{Định nghĩa}
\newtheorem{tc}{Tính chất}
\newtheorem{dl}{Định lý}
\newtheorem{md}{Mệnh đề}
\newtheorem{bd}{Bổ đề}
\newtheorem{hq}{Hệ quả}
\newtheorem{nx}{Nhận xét}
\newtheorem{vd}{Ví dụ}
\newtheorem{cy}{Chú ý}
\newtheorem{ttoan}{Thuật toán}
\pagenumbering{roman}\pagestyle{plain}
%\pagestyle{fancy}
%\lhead{\it \changefontsizes{11pt}Luận văn thạc sĩ:}
%\rhead{\it Một số phương pháp vô hướng hóa cơ bản trong tối ưu đa mục tiêu}
%\lfoot{\it Nguyễn Văn Vân } 			         
%\rfoot{\it K19.2 trường ĐHSG}
\renewcommand{\headrulewidth}{1,2pt} 			
\renewcommand{\footrulewidth}{1,2pt}
\newcommand{\dstc}[2]
{
	\newdimen\stringwidth\setbox0=\hbox{#1}
	\stringwidth=\wd0
	\hspace*{-\parindent}\hspace*{.5\textwidth}\hspace*{-.5\wd0}#1\hfill #2\bigskip
	
}  
\usepackage{scrextend}
\fancyhf{}
\lhead{}
\chead{\thepage}
\rhead{}
\cfoot{}
\rfoot{}
\lfoot{}
\pagestyle{fancy}
\renewcommand{\headrulewidth}{1pt}
\begin{document} 
\subsection{Phương pháp đơn hình}
\begin{ttoan}[Đơn hình]
\setlength{\parindent}{4em}
Ta có dạng chính tắc của bài toán được viết lại dưới dạng
\begin{equation}
\left\{\begin{split}
(-z) + 0x_B + c^Tx_N &= -z_0 \\
Ix_B + Ax_N &= b.
\end{split}\right.
\end{equation}
với $x_B \geq 0, \: x_N=0, \: z=z_0$. Thuật toán đơn hình theo các bước sau:

\noindent \textbf{Bước 1. Thiết lập.}
Xác định biến $c_j$ nhỏ nhất, ký hiệu
\begin{equation}
c_s = \underset{j}{\min} c_j
\end{equation}

\noindent \textbf{Bước 2. Kiểm tra tối ưu.}
Nếu $\forall c_s \geq 0$, bài toán được giải và thuật toán dừng lại. Nếu $\exists c_s \leq 0$, chuyển sang bước 3. 

\noindent \textbf{Bước 3. Chọn biến vào.}
Nếu $\exists c_s < 0$, đánh dấu $c_s$ là biến vào và chuyển sang bước 4.

\noindent \textbf{Bước 4. Kiểm tra giới hạn.}
Nếu $A_s \leq 0$, ta thực hiện loạt biến đổi sau:
\begin{equation}
z=z_0+c_sx_s, \: x_B = b - A_sx_s, \: x_j=0 \quad (j\neq s)
\end{equation}
trong đó $x_j$ là biến phi cơ sở. Nếu $z \rightarrow -\infty $ tương ứng $x_s \rightarrow \infty$, bài toán được giải và thuật toán kết thúc, nếu không chuyển sang bước 5.

\noindent \textbf{Bước 5. Chọn biến ra.}
Ta đánh dấu biến $x_j$ đã thực hiện trước đó, ký hiệu $x_r$ thành biến ra $x_s$ với điều kiện:
\begin{equation}
x_s = \frac{b_r}{a_{rs}} = \underset{a_{is}>0}{\min} \quad \frac{b_i}{a_{is}} \geq 0, \quad (a_{rs}>0).
\end{equation}
Sau đó chuyển sang bước 6.

\noindent \textbf{Bước 6. Xoay trục.}
Chọn $a_{rs}$ làm phần tử trụ, xác định $a_{ij}$ mới ký hiệu $a_{ij}^{'}$ bằng cách thực hiện thao tác:
\begin{equation}
a_{ij}^{'} = a_{ij} - \frac{a_{is}a_{rj}}{a_{rs}}
\end{equation}
sau đó quay trở lại bước 1.

\end{ttoan}
\end{document}