\documentclass{beamer}
\mode<presentation>
\setbeamertemplate{bibliography item}{}
\usepackage[utf8]{vietnam}
\usepackage{beamerthemesplit}
\usepackage{graphicx}
\usepackage{booktabs}
\usepackage{amsmath}
\usepackage{textpos}
\usepackage{pgfplots}
\usepackage{tikz}
\usepackage{hyperref}
\usepackage{caption}
\usetikzlibrary{shapes.geometric, arrows}
\usetikzlibrary {datavisualization} 
\pgfplotsset{compat=1.18, width = 7cm}
\usetikzlibrary{patterns}
\setbeamertemplate{bibliography item}[text]
\usetheme{Ilmenau} % AnnArbor, Ilmenau, Darmstadt, Dresden, CambridgeUS, Frankfurt, Singapore
\newtheorem{dn}{Định nghĩa}[section]
\newtheorem{dl}{Định lý}[section]
\newtheorem{tc}{Tính chất}[section]
\newtheorem{hq}{Hệ quả}[section]
\newtheorem{bd}{Bổ đề}[section]
\newtheorem{md}{Mệnh đề}[section]
\newtheorem{vd}{Ví dụ}[section]
\newtheorem{nx}{Nhận xét}[section]
\newcommand{\dom}{\text{{\rm dom}}}
\newcommand{\epi}{\text{{\rm epi}}}
\newcommand{\Min}{\text{{\rm Min}}}
\setbeamertemplate{theorems}[numbered]
\setbeamertemplate{definitions}[numbered]
\setbeamertemplate{footline}[frame number]
\usepackage{algorithm}
\usepackage{color}
\usepackage{algorithmic}
\usepackage{footmisc}
\usepackage{indentfirst} 
\usepackage{comment}
\AtBeginEnvironment{proof}{%
  \setbeamercolor{block title}{use=example text,fg=white,bg=example text.fg!75!black}
  \setbeamercolor{block body}{parent=normal text,use=block title example,bg=block title example.bg!10!bg}
}
\renewcommand{\thefootnote}{\arabic{footnote}}
\usefonttheme{professionalfonts}
\setbeamercolor{normal text}{bg=white,fg=black}
\renewcommand{\thefootnote}{\arabic{footnote}}
\beamertemplatetransparentcoveredhigh
\title[]{\fontsize{13pt}{10pt}\selectfont {\bf \LARGE Tối ưu phân tuyến tính \\ cho nghiệm nguyên}\\}
\author[]{\bf Nguyễn Chí Bằng \\}
\small{\date{\today}}

\begin{document}

\begin{frame}
    \titlepage
\end{frame}

\begin{frame}{TÓM TẮT}
\Large
\begin{itemize}
\item Giới thiệu về bài toán tối ưu phân tuyến tính:
\begin{itemize} \Large
\item Cơ sở lý thuyết.
\item Thuật toán Dinkelbach.
\end{itemize}
\item Phương pháp giải bài toán tối ưu phân tuyến tính cho nghiệm nguyên bằng thuật toán nhánh cận (LandDoig).
\end{itemize}
\end{frame}

\begin{frame}
    \frametitle{NỘI DUNG}
    \tableofcontents
\end{frame}

\section{Giới thiệu}
%FLP-begin
\begin{frame}
   \center 
   \Huge Giới thiệu bài toán 
\end{frame}
\begin{frame}{Tối ưu phân tuyến tính (Linear-Fractional Programming)}
    \begin{equation} \small \label{F}
        \begin{split}
        (F) \quad Q(x) & = \frac{P(x)}{D(x)} \quad \longrightarrow Max \\
            & \left\{
            \begin{split}
            &Ax \leq  b, \\
            &x \geq 0. \\
            \end{split}
            \right.    
        \end{split}
    \end{equation}            
    \begin{itemize} \small
    \item Bài toán $(F)$ gọi là bài toán \textbf{Tối ưu phân tuyến tính.}
    \item Trong đó $A$ là ma trận $m\times n$, $b=\begin{pmatrix}
        b_1 \\
        b_2 \\
        \vdots \\
        b_m
        \end{pmatrix}$, với $x\in \mathbb{R}^n_+$. Tập $S_F:=\{x\in \mathbb{R}^n_+: Ax\leq b\}$ là tập nghiệm của bài toán Tối ưu phân tuyến tính. 
    \item $P(x)=p^Tx+p_0$, với $p^T = (p_1 \: p_2 \: \ldots \: p_n)$ và $D(x)=d^Tx+d_0$, với $d^T = (d_1 \: d_2 \: \ldots \: d_n)$ ($D(x)>0, \forall x \in S_F$).
    \end{itemize}
\end{frame}
\begin{frame}{Bài toán minh hoạ} \Large
    \begin{equation}
    \begin{split}
    \quad Q(x) & = \frac{4x_1+2x_2-6}{3x_1+2x_2-5} \quad \longrightarrow Max \\
        & \left\{
        \begin{split}
        & x_1 + x_2 \geq 6 \\
        & x_1 + 2x_2 \leq 12 \\
        &x_1, x_2 \geq 0. \\
        \end{split}\right.    
    \end{split}
    \end{equation}            
\end{frame}
\begin{frame}{Mối quan hệ với bài toán tối ưu tuyến tính}
\begin{itemize}
\item Nếu $d^T=0$ và $d_0=1$, bài toán (F) trở thành bài toán tối ưu tuyến tính (P) và ta gọi (F) là bài toán mở rộng của (P):
\begin{equation} \small \label{P}
    \begin{split}
    (P) \quad P(x) & = p^Tx+p_0 \quad \longrightarrow Max \\
        & \left\{
        \begin{split}
        &Ax \leq  b, \\
        &x \geq 0. \\
        \end{split}
        \right.    
    \end{split}
\end{equation}
\item Nếu $d^T=0$ và $d_0 \neq 0$, ta thu được bài toán tuyến tính (Q):
\begin{equation} \small \label{Q}
    \begin{split}
    (Q) \quad Q(x) & = \frac{p^T}{d_0}x+\frac{p_0}{d_0}=\frac{P(x)}{d_0} \quad \longrightarrow Max \\
        & \left\{
        \begin{split}
        &Ax \leq  b, \\
        &x \geq 0. \\
        \end{split}
        \right.    
    \end{split}
\end{equation}
\end{itemize}
\end{frame}
\begin{frame}
\begin{itemize}
\item Ngược lại nếu $p^T=0$ và $p_0 \neq 0$:
\begin{equation} \small \label{Q}
    \begin{split}
    (Q) \quad Q(x) & = \frac{p_0}{d^Tx+d_0} = \frac{p_0}{D(x)} \quad \longrightarrow Max \\
        & \left\{
        \begin{split}
        &Ax \leq  b, \\
        &x \geq 0. \\
        \end{split}
        \right.    
    \end{split}
\end{equation}
Tương tự bài toán:
\begin{equation} \small \label{Q}
    \begin{split}
    (Q) \quad Q(x) & = \frac{d^Tx+d_0}{p_0} = \frac{D(x)}{p_0} \quad \longrightarrow Min \\
        & \left\{
        \begin{split}
        &Ax \leq  b, \\
        &x \geq 0. \\
        \end{split}
        \right.    
    \end{split}
\end{equation}
\item Nếu $p^T$ và $d^T$ phụ thuộc tuyến tính, khi đó tồn tại $\mu \neq 0$ và $p^T=\mu d^T$, ta thu được hàm:
\end{itemize}
\end{frame}
\begin{frame} \large
\begin{equation} \label{Q}
    \begin{split}
    (Q) \quad Q(x) & = \frac{\mu d^Tx + p_0}{d^Tx+d_0} = \mu + \frac{p_0-\mu d_0}{d^Tx+d_0} \quad \\
        & \left\{
        \begin{split}
        &Ax \leq  b, \\
        &x \geq 0. \\
        \end{split}
        \right.    
    \end{split}
\end{equation}
Ta thay bằng hàm $D(x)$ với điều kiện:
\begin{itemize}
\item Nếu $p_0 - \mu d_0 > 0$, $D(x) \longrightarrow Min$.
\item Nếu $p_0 - \mu d_0 < 0$, $D(x) \longrightarrow Max$.
\item Nếu $p_0 - \mu d_0 = 0$ thì $Q(x)=\mu= \text{hằng số} \: (\forall x \in S_F)$, ta bỏ qua bài toán.
\end{itemize}
\end{frame}
%FLP-end

%GRAPHICAL-BEGIN
\section{Phương pháp hình học}
\begin{frame}
   \center 
   \Huge Phương pháp hình học 
\end{frame}

\begin{frame}{Bài toán trên không gian $\mathbb{R}^2$}
\begin{equation} \large
    \begin{split}
    (F) \quad Q(x) & = \frac{P(x)}{D(x)} = \frac{p_1x_1 + p_2 x_2 + p_0}{d_1x_1+d_2x_2 + d_0}\quad \longrightarrow Max \\
        & \left\{
        \begin{split}
        &Ax \leq  b, \text{ trong đó } A=m \times 2 \\
        &x_1, x_2 \geq 0. \\
        \end{split}
        \right.    
    \end{split}
\end{equation}
\end{frame}

\begin{frame}
\begin{figure}
    \begin{tikzpicture}[scale=1.2]
    \begin{axis}
        [
        xmin=0,xmax=80,
        ymin=0,ymax=80,
        xlabel={$x_1$},
        ylabel={$x_2$},
        grid style={line width=.1pt, draw=darkgray!50},
        major grid style={line width=.2pt,draw=darkgray!50},
        axis lines=middle,
        yticklabel=\empty,
        xticklabel=\empty,
        enlargelimits={abs=0},
        samples=100,
        domain = -20:20,
        ]
        \filldraw[blue!30, pattern=north west lines, pattern color=blue!30, line width=1pt] (30, 25) -- (50, 20) -- (70, 30) -- (70, 50) -- (50, 65) -- (20, 50) -- cycle;
        \node[blue!50] at (60,40) {\tiny $S_F$};
    \end{axis}
    \end{tikzpicture}  
    \caption{Tập nghiệm minh hoạ của bài toán (F)}
    \end{figure}
\end{frame}
\begin{frame}{Tính chất}
Đặt $Q(x)=K$ ($K \in \mathbb{R}$), ta được:
\begin{equation} \label{pt0}
(p_1-Kd_1)x_1+(p_2-Kd_2)x_2+(p_0-Kd_0) = 0
\end{equation}
\begin{equation} \label{pt1}
\implies
\left\{ \large
\begin{array}{rcr}
p_1x_1+p_2x_2+p_0 &=& 0 \\
d_1x_1+d_2x_2+d_0 &=& 0 \\
\end{array}
\right.
\end{equation}
\begin{itemize}
\item $Q(x)=K$ là đường mức quét qua tập $S_F$, đến khi gặp \textsl{cực điểm} thì ở đó ta nhận được giá trị $K$ là giá trị tối ưu của bài toán (F).
\item Ta xác định được điểm cố định $F$ là nghiệm của phương trình \eqref{pt0}, nói cách khác, điểm cố định $F$ là điểm giao của 2 đường thẳng $P(x)=0$ và $D(x)=0$.
\item Trường hợp $P(x)=0$ song song với $D(x)=0$, hay nói cách khác hệ \eqref{pt0} vô nghiệm thì điểm cố định $F$ không tồn tại.
\end{itemize}
\end{frame}
\begin{frame}
\begin{figure}
    \begin{tikzpicture}[scale=1.2]
    \begin{axis}
        [
        xmin=-10,xmax=80,
        ymin=-10,ymax=80,
        xlabel={$x_1$},
        ylabel={$x_2$},
        grid style={line width=.1pt, draw=darkgray!50},
        major grid style={line width=.2pt,draw=darkgray!50},
        axis lines=middle,
        yticklabel=\empty,
        xticklabel=\empty,
        enlargelimits={abs=0},
        samples=100,
        domain = -20:20,
        ]
        \filldraw[blue!30, pattern=north west lines, pattern color=blue!30, line width=1pt] (30, 25) -- (50, 20) -- (70, 30) -- (70, 50) -- (50, 65) -- (20, 50) -- cycle;
        \node[blue!50] at (60,40) {\tiny $S_F$};
        \node at (20, 25) {\small $F$};
        \node at (20,19.5) {\large \textbullet};
        \node at (55, -7) {\tiny $D(x)=0$};
        \node[rotate=60] at (42, 70) {\tiny $P(x)=0$};
        \draw (15, 10) -- (50, 80);
        \draw (10, 25) -- (70, -7);
    \end{axis}
    \end{tikzpicture}  
    \caption{Minh hoạ điểm cố định $F$}
    \end{figure}
\end{frame}
\begin{frame}
\begin{figure}
    \begin{tikzpicture}[scale=1.2]
    \begin{axis}
        [
        xmin=-10,xmax=80,
        ymin=-10,ymax=80,
        xlabel={$x_1$},
        ylabel={$x_2$},
        grid style={line width=.1pt, draw=darkgray!50},
        major grid style={line width=.2pt,draw=darkgray!50},
        axis lines=middle,
        yticklabel=\empty,
        xticklabel=\empty,
        enlargelimits={abs=0},
        samples=100,
        domain = -20:20,
        ]
        \filldraw[blue!30, pattern=north west lines, pattern color=blue!30, line width=1pt] (30, 25) -- (50, 20) -- (70, 30) -- (70, 50) -- (50, 65) -- (20, 50) -- cycle;
        \node[blue!50] at (60,40) {\tiny $S_F$};
        \node at (20, 25) {\small $F$};
        \node at (20,19.5) {\large \textbullet};
        \node at (55, -7) {\tiny $D(x)=0$};
        \node[rotate=60] at (42, 70) {\tiny $P(x)=0$};
        \draw (15, 10) -- (50, 80);
        \draw (10, 25) -- (70, -7);
        \draw[blue] (5, 19) -- (80, 20);
        \draw[blue] (20, 11) -- (20, 72);
        \draw [red] (6, 7) -- (77, 73);
        \node at (65, 15) {\tiny \color{blue} $Q(x)= Min$};
        \node[rotate=90] at (15, 60) {\tiny \color{blue} $Q(x)= Max$};
        \node[rotate=35] at (70, 60) {\tiny \color{red} $Q(x)=K$};
    \end{axis}
    \end{tikzpicture}  
    \caption{Minh hoạ đường mức $Q(x)=K$}
    \end{figure}
\end{frame}
\begin{frame}{Xét tính biến thiên} \large
Từ phương trình \eqref{pt0}, ta có:
\begin{equation}
x_2=-\frac{p_1-Kd_1}{p_2-Kd_2}x_1-\frac{p_0-Kd_0}{p_2-Kd_2}.
\end{equation}
Đạo hàm 2 vế ta được:
\begin{equation*}
\begin{split}
    \frac{dx_2}{dx_1} &= \frac{d}{dx_1} \Bigg[-\frac{p_1-Kd_1}{p_2-Kd_2}x_1\Bigg] - \frac{d}{dx_1} \Bigg[\frac{p_0-Kd_0}{p_2-Kd_2} \Bigg] \\
    \frac{dx_2}{dx_1} &= -\frac{p_1-Kd_1}{p_2-Kd_2} \\
    k &= -\frac{p_1-Kd_1}{p_2-Kd_2} \text{    (Đặt $k=\frac{dx_2}{dx_1}$)}
\end{split} 
\end{equation*}
\end{frame}

\begin{frame} \large
\begin{itemize}
\item Ta thấy hệ số góc $k$ phụ thuộc vào tham số $K$, Khảo sát sự biến thiên của $k$ theo $K$, ta có:
\begin{equation}
\frac{dk}{dK}=\frac{d_1p_2-d_2p_1}{(p_2-Kd_2)^2}
\end{equation}
\item Giá trị của $Q(x)$ tăng hay giảm phụ thuộc vào $(d_1p_2-d_2p_1)$, do đó $k$ biến thiên theo 1 chiều nhất định. Từ đây, ta có thể tìm được nghiệm tối ưu của bài toán (F).
\item Quay đường mức $Q(x)=K$ quanh điểm $F$ đến khi trùng với cực điểm $x^*$ ta nhận được giá trị cực đại của hàm $Q(x)$, $x^{**}$ ta nhận được giá trị cực tiểu của hàm $Q(x)$. 
\end{itemize}
\end{frame}

\begin{frame}
\begin{figure}
    \begin{tikzpicture}[scale=1.2]
    \begin{axis}
        [
        xmin=-10,xmax=80,
        ymin=-10,ymax=80,
        xlabel={$x_1$},
        ylabel={$x_2$},
        grid style={line width=.1pt, draw=darkgray!50},
        major grid style={line width=.2pt,draw=darkgray!50},
        axis lines=middle,
        yticklabel=\empty,
        xticklabel=\empty,
        enlargelimits={abs=0},
        samples=100,
        domain = -20:20,
        ]
        \filldraw[blue!30, pattern=north west lines, pattern color=blue!30, line width=1pt] (30, 25) -- (50, 20) -- (70, 30) -- (70, 50) -- (50, 65) -- (20, 50) -- cycle;
        \node[blue!50] at (60,40) {\tiny $S_F$};
        \node at (20, 25) {\small $F$};
        \node at (20,19.5) {\large \textbullet};
        \node at (55, -7) {\tiny $D(x)=0$};
        \node[rotate=60] at (42, 70) {\tiny $P(x)=0$};
        \draw (15, 10) -- (50, 80);
        \draw (10, 25) -- (70, -7);
        \draw[red] (5, 19) -- (80, 20);
        \draw[blue] (20, 11) -- (20, 72);
        \node at (60, 15) {\tiny \color{red} $Q(x)= Min = K$};
        \node[rotate=90] at (15, 60) {\tiny \color{blue} $Q(x)= Max$};
        \node[blue] at (53, 27) {\small $x^{**}$};
        \node[blue] at (50,20) {\large \textbullet};
    \end{axis}
    \end{tikzpicture}  
    \caption{$Q(x)$ đạt giá trị tối ưu lại điểm $x^{**}$}
    \end{figure}
\end{frame}
\begin{frame}
\begin{figure}
    \begin{tikzpicture}[scale=1.2]
    \begin{axis}
        [
        xmin=-10,xmax=80,
        ymin=-10,ymax=80,
        xlabel={$x_1$},
        ylabel={$x_2$},
        grid style={line width=.1pt, draw=darkgray!50},
        major grid style={line width=.2pt,draw=darkgray!50},
        axis lines=middle,
        yticklabel=\empty,
        xticklabel=\empty,
        enlargelimits={abs=0},
        samples=100,
        domain = -20:20,
        ]
        \filldraw[blue!30, pattern=north west lines, pattern color=blue!30, line width=1pt] (30, 25) -- (50, 20) -- (70, 30) -- (70, 50) -- (50, 65) -- (20, 50) -- cycle;
        \node[blue!50] at (60,40) {\tiny $S_F$};
        \node at (20, 25) {\small $F$};
        \node at (20,19.5) {\large \textbullet};
        \node at (55, -7) {\tiny $D(x)=0$};
        \node[rotate=60] at (42, 70) {\tiny $P(x)=0$};
        \draw (15, 10) -- (50, 80);
        \draw (10, 25) -- (70, -7);
        \draw[blue] (5, 19) -- (80, 20);
        \draw[red] (20, 11) -- (20, 72);
        \node at (65, 15) {\tiny \color{blue} $Q(x)= Min$};
        \node[rotate=90] at (15, 60) {\tiny \color{red} $Q(x)= Max=K$};
        \node[blue] at (25, 50) {\small $x^{*}$};
        \node[blue] at (20,50) {\large \textbullet};
    \end{axis}
    \end{tikzpicture}  
    \caption{$Q(x)$ đạt giá trị tối ưu lại điểm $x^*$}
    \end{figure}
\end{frame}

\begin{frame}{Ví dụ minh hoạ}
\end{frame}
%GRAPHICAL-END
%DINKELBACH-start
\section{Thuật toán Dinkelbach}
\begin{frame}
   \center 
   \Huge Thuật toán Dinkelbach
\end{frame}

\begin{frame}{Tính chất} \large
\begin{itemize}
\item Quay lại bài toán:
\begin{equation} \label{F}
    \begin{split}
    (F) \quad Q(x) & = \frac{P(x)}{D(x)} \quad \longrightarrow Max \\
        & \left\{
        \begin{split}
        &Ax \leq  b, \\
        &x \geq 0. \\
        \end{split}
        \right.    
    \end{split}
\end{equation}            
\item Ta đặt hàm:
\begin{equation}
F(\lambda) = \underset{x \in S_F}{\max}\{P(x)-\lambda D(x)\}, \: \lambda \in \mathbb{R}
\end{equation}
\end{itemize}
\end{frame}

\begin{frame}
\begin{dl}
Vector $x^*$ là nghiệm tối ưu của bài toán (F) nếu và chỉ nếu 
\begin{equation}
F(\lambda^*) = \underset{x \in S_F}{\max}\{P(x)-\lambda^* D(x)\} = 0 
\end{equation}
Với
\begin{equation}
\lambda^* = \frac{P(x^*)}{D(x^*)}
\end{equation}
\end{dl}
\end{frame}

\begin{frame}
\begin{proof}
Nếu vector $x^*$ là nghiệm tối ưu của bài toán (F), ta có:
\begin{equation*}
\lambda^* = \frac{P(x^*)}{D(x^*)} \geq \frac{P(x)}{D(x)}, \: \forall x \in S_F
\end{equation*}
Tương tự,
\begin{equation} \label{pt2}
P(x)-\lambda^* D(x) \leq 0,  \forall x \in S_F
\end{equation}
Từ bất phương trình \eqref{pt2}, ta được:
\begin{equation*}
\Rightarrow \underset{x \in S_F}{\max}\{P(x)-\lambda^* D(x)\} = 0
\end{equation*}
Nếu vector $x^*$ là nghiệm tối ưu thì:
\begin{equation*}
P(x)-\lambda^*D(x) \leq P(x^*)-\lambda^* D(x^*) = 0, \: \forall x \in S_F
\end{equation*}
\end{proof}
\end{frame}

\begin{frame} \Large
Với $D(x)>0, \: \forall x \in S_F$, ta có:
\begin{equation}
\frac{\partial F(\lambda)}{\partial \lambda}= -D(x) < 0
\end{equation}
Đồng nghĩa $F(\lambda)$ giảm theo $\lambda$, từ đó ta thiết lập được thuật toán Dinkelbach.
\end{frame}

\begin{frame}{Thuật toán Dinkelbach}
\setlength{\parindent}{4em}
\noindent \textbf{Bước 1. Thiết lập} \\
Đặt $x^{(0)} \in S_F$, tính $\lambda^{(1)}:=\frac{P(x^{(0)})}{D(x^{(0)})}$, $k:=1$. \\
\noindent \textbf{Bước 2. Tìm nghiệm} \\
Tìm $x^{(k)}$ là nghiệm của bài toán $\underset{x \in S_F}{\max}\{P(x)-\lambda^{(k)} D(x)\}$ \\
\noindent \textbf{Bước 3. Kiểm tra} \\
Nếu $F(\lambda^{(k)})=0$ thì $x^*=x^{(k)}$ là nghiệm tối ưu và bài \\ toán được giải, nếu không chuyển sang bước 4. \\
\noindent \textbf{Bước 4. Cải thiện} \\
Đặt $\lambda^{(k+1)}:=\frac{P(x^{(k)})}{D(x^{(k)})}$, $k:=k+1$ và quay lại bước 2.
\end{frame}

\begin{frame}{Ví dụ minh hoạ}
\end{frame}
%DINKELBACH-end
\section{Thuật toán LandDoig - Dinkelbach}
\begin{frame}
   \center 
   \Huge Thuật toán LandDoig - Dinkelbach
\end{frame}

\begin{frame}
\end{frame}

\end{document}