\documentclass{beamer}
\mode<presentation>
\setbeamertemplate{bibliography item}{}
\usepackage[utf8]{vietnam}
\usepackage{beamerthemesplit}
\usepackage{graphicx}
\usepackage{booktabs}
\usepackage{amsmath}
\usepackage{textpos}
\usepackage{pgfplots}
\usepackage{tikz}
\usepackage{hyperref}
\usepackage{caption}
\usetikzlibrary{shapes.geometric, arrows}
\usetikzlibrary {datavisualization} 
\pgfplotsset{compat=1.18, width = 7cm}
\usetikzlibrary{patterns}
\setbeamertemplate{bibliography item}[text]
\usetheme{Ilmenau} % AnnArbor, Ilmenau, Darmstadt, Dresden, CambridgeUS, Frankfurt, Singapore
\newtheorem{dn}{Định nghĩa}[section]
\newtheorem{dl}{Định lý}[section]
\newtheorem{tc}{Tính chất}[section]
\newtheorem{hq}{Hệ quả}[section]
\newtheorem{bd}{Bổ đề}[section]
\newtheorem{md}{Mệnh đề}[section]
\newtheorem{vd}{Ví dụ}[section]
\newtheorem{nx}{Nhận xét}[section]
\newcommand{\dom}{\text{{\rm dom}}}
\newcommand{\epi}{\text{{\rm epi}}}
\newcommand{\Min}{\text{{\rm Min}}}
\setbeamertemplate{theorems}[numbered]
\setbeamertemplate{definitions}[numbered]
\setbeamertemplate{footline}[frame number]
\usepackage{algorithm}
\usepackage{color}
\usepackage{algorithmic}
\usepackage{footmisc}
\usepackage{indentfirst} 
\usepackage{comment}
\AtBeginEnvironment{proof}{%
  \setbeamercolor{block title}{use=example text,fg=white,bg=example text.fg!75!black}
  \setbeamercolor{block body}{parent=normal text,use=block title example,bg=block title example.bg!10!bg}
}
\renewcommand{\thefootnote}{\arabic{footnote}}
\usefonttheme{professionalfonts}
\setbeamercolor{normal text}{bg=white,fg=black}
\renewcommand{\thefootnote}{\arabic{footnote}}
\beamertemplatetransparentcoveredhigh
\title[]{\fontsize{13pt}{10pt}\selectfont {\bf \LARGE Tối ưu phân tuyến tính \\ cho nghiệm nguyên}\\}
\author[]{\bf Nguyễn Chí Bằng \\}
\small{\date{\today}}

\begin{document}

\begin{frame}
    \titlepage
\end{frame}

\begin{frame}{TÓM TẮT}
\begin{itemize}
\item Giới thiệu về bài toán tối ưu phân tuyến tính:
\begin{itemize}
\item Cơ sở lý thuyết.
\item Thuật toán Dinkelbach.
\end{itemize}
\item Phương pháp giải bài toán tối ưu phân tuyến tính cho nghiệm nguyên bằng thuật toán nhánh cận (LandDoig).
\end{itemize}
\end{frame}

\begin{frame}
    \frametitle{NỘI DUNG}
    \tableofcontents
\end{frame}

\section{Giới thiệu}
%FLP-begin

\begin{frame}
   \center 
   \huge Giới thiệu bài toán 
\end{frame}

\begin{frame}{Tối ưu phân tuyến tính (Linear-Fractional Programming)}
    \begin{equation} \label{H}
        \begin{split}
        (F) \quad & Q(x) = \frac{P(x)}{D(x)} \quad \longrightarrow Max \: (Min) \\
            & \left\{
            \begin{split}
            &Ax \leq  b, \\
            &x \geq 0. \\
            \end{split}
            \right.    
        \end{split}
    \end{equation}            
    \begin{itemize} \small
    \item Trong đó $P(x)=p^Tx+p_0$, với $p^T = (p_1 \: p_2 \: \ldots \: p_n)$ và $D(x)=d^Tx+d_0$, với $d^T = (d_1 \: d_2 \: \ldots \: d_n)$. $A$ là ma trận $m\times n$, $b=\begin{pmatrix}
        b_1 \\
        b_2 \\
        \vdots \\
        b_m
        \end{pmatrix}$, với $x\in \mathbb{R}^n_+$.
    \item Bài toán $(F)$ gọi là bài toán \textbf{Tối ưu phân tuyến tính.}
    \item Tập $S_F:=\{x\in \mathbb{R}^n_+: Ax\leq b\}$ là tập nghiệm của bài toán Tối ưu phân tuyến tính.
    \end{itemize}
\end{frame}


\begin{frame}{Bài toán minh hoạ}
    
\end{frame}

%FLP-end

\section{Thuật toán Dinkelbach}
\section{Thuật toán LandDoig - Dinkelbach}

\begin{frame}
\end{frame}

\end{document}