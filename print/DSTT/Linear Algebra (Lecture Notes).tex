\documentclass[11pt]{article}

    \usepackage[breakable]{tcolorbox}
    \usepackage{parskip} % Stop auto-indenting (to mimic markdown behaviour)
    

    % Basic figure setup, for now with no caption control since it's done
    % automatically by Pandoc (which extracts ![](path) syntax from Markdown).
    \usepackage{graphicx}
    % Maintain compatibility with old templates. Remove in nbconvert 6.0
    \let\Oldincludegraphics\includegraphics
    % Ensure that by default, figures have no caption (until we provide a
    % proper Figure object with a Caption API and a way to capture that
    % in the conversion process - todo).
    \usepackage{caption}
    \DeclareCaptionFormat{nocaption}{}
    \captionsetup{format=nocaption,aboveskip=0pt,belowskip=0pt}

    \usepackage{float}
    \floatplacement{figure}{H} % forces figures to be placed at the correct location
    \usepackage{xcolor} % Allow colors to be defined
    \usepackage{enumerate} % Needed for markdown enumerations to work
    \usepackage{geometry} % Used to adjust the document margins
    \usepackage{amsmath} % Equations
    \usepackage{amssymb} % Equations
    \usepackage{textcomp} % defines textquotesingle
    % Hack from http://tex.stackexchange.com/a/47451/13684:
    \AtBeginDocument{%
        \def\PYZsq{\textquotesingle}% Upright quotes in Pygmentized code
    }
    \usepackage{upquote} % Upright quotes for verbatim code
    \usepackage{eurosym} % defines \euro

    \usepackage{iftex}
    \ifPDFTeX
        \usepackage[T1]{fontenc}
        \IfFileExists{alphabeta.sty}{
              \usepackage{alphabeta}
          }{
              \usepackage[mathletters]{ucs}
              \usepackage[utf8x]{inputenc}
          }
    \else
        \usepackage{fontspec}
        \usepackage{unicode-math}
    \fi

    \usepackage{fancyvrb} % verbatim replacement that allows latex
    \usepackage{grffile} % extends the file name processing of package graphics
                         % to support a larger range
    \makeatletter % fix for old versions of grffile with XeLaTeX
    \@ifpackagelater{grffile}{2019/11/01}
    {
      % Do nothing on new versions
    }
    {
      \def\Gread@@xetex#1{%
        \IfFileExists{"\Gin@base".bb}%
        {\Gread@eps{\Gin@base.bb}}%
        {\Gread@@xetex@aux#1}%
      }
    }
    \makeatother
    \usepackage[Export]{adjustbox} % Used to constrain images to a maximum size
    \adjustboxset{max size={0.9\linewidth}{0.9\paperheight}}

    % The hyperref package gives us a pdf with properly built
    % internal navigation ('pdf bookmarks' for the table of contents,
    % internal cross-reference links, web links for URLs, etc.)
    \usepackage{hyperref}
    % The default LaTeX title has an obnoxious amount of whitespace. By default,
    % titling removes some of it. It also provides customization options.
    \usepackage{titling}
    \usepackage{longtable} % longtable support required by pandoc >1.10
    \usepackage{booktabs}  % table support for pandoc > 1.12.2
    \usepackage{array}     % table support for pandoc >= 2.11.3
    \usepackage{calc}      % table minipage width calculation for pandoc >= 2.11.1
    \usepackage[inline]{enumitem} % IRkernel/repr support (it uses the enumerate* environment)
    \usepackage[normalem]{ulem} % ulem is needed to support strikethroughs (\sout)
                                % normalem makes italics be italics, not underlines
    \usepackage{soul}      % strikethrough (\st) support for pandoc >= 3.0.0
    \usepackage{mathrsfs}
    
    % Define pandocbounded command for image inclusion
    \providecommand{\pandocbounded}[1]{#1}

    
    % Colors for the hyperref package
    \definecolor{urlcolor}{rgb}{0,.145,.698}
    \definecolor{linkcolor}{rgb}{.71,0.21,0.01}
    \definecolor{citecolor}{rgb}{.12,.54,.11}

    % ANSI colors
    \definecolor{ansi-black}{HTML}{3E424D}
    \definecolor{ansi-black-intense}{HTML}{282C36}
    \definecolor{ansi-red}{HTML}{E75C58}
    \definecolor{ansi-red-intense}{HTML}{B22B31}
    \definecolor{ansi-green}{HTML}{00A250}
    \definecolor{ansi-green-intense}{HTML}{007427}
    \definecolor{ansi-yellow}{HTML}{DDB62B}
    \definecolor{ansi-yellow-intense}{HTML}{B27D12}
    \definecolor{ansi-blue}{HTML}{208FFB}
    \definecolor{ansi-blue-intense}{HTML}{0065CA}
    \definecolor{ansi-magenta}{HTML}{D160C4}
    \definecolor{ansi-magenta-intense}{HTML}{A03196}
    \definecolor{ansi-cyan}{HTML}{60C6C8}
    \definecolor{ansi-cyan-intense}{HTML}{258F8F}
    \definecolor{ansi-white}{HTML}{C5C1B4}
    \definecolor{ansi-white-intense}{HTML}{A1A6B2}
    \definecolor{ansi-default-inverse-fg}{HTML}{FFFFFF}
    \definecolor{ansi-default-inverse-bg}{HTML}{000000}

    % common color for the border for error outputs.
    \definecolor{outerrorbackground}{HTML}{FFDFDF}

    % commands and environments needed by pandoc snippets
    % extracted from the output of `pandoc -s`
    \providecommand{\tightlist}{%
      \setlength{\itemsep}{0pt}\setlength{\parskip}{0pt}}
    \DefineVerbatimEnvironment{Highlighting}{Verbatim}{commandchars=\\\{\}}
    % Add ',fontsize=\small' for more characters per line
    \newenvironment{Shaded}{}{}
    \newcommand{\KeywordTok}[1]{\textcolor[rgb]{0.00,0.44,0.13}{\textbf{{#1}}}}
    \newcommand{\DataTypeTok}[1]{\textcolor[rgb]{0.56,0.13,0.00}{{#1}}}
    \newcommand{\DecValTok}[1]{\textcolor[rgb]{0.25,0.63,0.44}{{#1}}}
    \newcommand{\BaseNTok}[1]{\textcolor[rgb]{0.25,0.63,0.44}{{#1}}}
    \newcommand{\FloatTok}[1]{\textcolor[rgb]{0.25,0.63,0.44}{{#1}}}
    \newcommand{\CharTok}[1]{\textcolor[rgb]{0.25,0.44,0.63}{{#1}}}
    \newcommand{\StringTok}[1]{\textcolor[rgb]{0.25,0.44,0.63}{{#1}}}
    \newcommand{\CommentTok}[1]{\textcolor[rgb]{0.38,0.63,0.69}{\textit{{#1}}}}
    \newcommand{\OtherTok}[1]{\textcolor[rgb]{0.00,0.44,0.13}{{#1}}}
    \newcommand{\AlertTok}[1]{\textcolor[rgb]{1.00,0.00,0.00}{\textbf{{#1}}}}
    \newcommand{\FunctionTok}[1]{\textcolor[rgb]{0.02,0.16,0.49}{{#1}}}
    \newcommand{\RegionMarkerTok}[1]{{#1}}
    \newcommand{\ErrorTok}[1]{\textcolor[rgb]{1.00,0.00,0.00}{\textbf{{#1}}}}
    \newcommand{\NormalTok}[1]{{#1}}

    % Additional commands for more recent versions of Pandoc
    \newcommand{\ConstantTok}[1]{\textcolor[rgb]{0.53,0.00,0.00}{{#1}}}
    \newcommand{\SpecialCharTok}[1]{\textcolor[rgb]{0.25,0.44,0.63}{{#1}}}
    \newcommand{\VerbatimStringTok}[1]{\textcolor[rgb]{0.25,0.44,0.63}{{#1}}}
    \newcommand{\SpecialStringTok}[1]{\textcolor[rgb]{0.73,0.40,0.53}{{#1}}}
    \newcommand{\ImportTok}[1]{{#1}}
    \newcommand{\DocumentationTok}[1]{\textcolor[rgb]{0.73,0.13,0.13}{\textit{{#1}}}}
    \newcommand{\AnnotationTok}[1]{\textcolor[rgb]{0.38,0.63,0.69}{\textbf{\textit{{#1}}}}}
    \newcommand{\CommentVarTok}[1]{\textcolor[rgb]{0.38,0.63,0.69}{\textbf{\textit{{#1}}}}}
    \newcommand{\VariableTok}[1]{\textcolor[rgb]{0.10,0.09,0.49}{{#1}}}
    \newcommand{\ControlFlowTok}[1]{\textcolor[rgb]{0.00,0.44,0.13}{\textbf{{#1}}}}
    \newcommand{\OperatorTok}[1]{\textcolor[rgb]{0.40,0.40,0.40}{{#1}}}
    \newcommand{\BuiltInTok}[1]{{#1}}
    \newcommand{\ExtensionTok}[1]{{#1}}
    \newcommand{\PreprocessorTok}[1]{\textcolor[rgb]{0.74,0.48,0.00}{{#1}}}
    \newcommand{\AttributeTok}[1]{\textcolor[rgb]{0.49,0.56,0.16}{{#1}}}
    \newcommand{\InformationTok}[1]{\textcolor[rgb]{0.38,0.63,0.69}{\textbf{\textit{{#1}}}}}
    \newcommand{\WarningTok}[1]{\textcolor[rgb]{0.38,0.63,0.69}{\textbf{\textit{{#1}}}}}


    % Define a nice break command that doesn't care if a line doesn't already
    % exist.
    \def\br{\hspace*{\fill} \\* }
    % Math Jax compatibility definitions
    \def\gt{>}
    \def\lt{<}
    \let\Oldtex\TeX
    \let\Oldlatex\LaTeX
    \renewcommand{\TeX}{\textrm{\Oldtex}}
    \renewcommand{\LaTeX}{\textrm{\Oldlatex}}
    % Document parameters
    % Document title
    \title{Linear Algebra (Lecture Notes)}
    
    
    
    
    
    
    
% Pygments definitions
\makeatletter
\def\PY@reset{\let\PY@it=\relax \let\PY@bf=\relax%
    \let\PY@ul=\relax \let\PY@tc=\relax%
    \let\PY@bc=\relax \let\PY@ff=\relax}
\def\PY@tok#1{\csname PY@tok@#1\endcsname}
\def\PY@toks#1+{\ifx\relax#1\empty\else%
    \PY@tok{#1}\expandafter\PY@toks\fi}
\def\PY@do#1{\PY@bc{\PY@tc{\PY@ul{%
    \PY@it{\PY@bf{\PY@ff{#1}}}}}}}
\def\PY#1#2{\PY@reset\PY@toks#1+\relax+\PY@do{#2}}

\@namedef{PY@tok@w}{\def\PY@tc##1{\textcolor[rgb]{0.73,0.73,0.73}{##1}}}
\@namedef{PY@tok@c}{\let\PY@it=\textit\def\PY@tc##1{\textcolor[rgb]{0.24,0.48,0.48}{##1}}}
\@namedef{PY@tok@cp}{\def\PY@tc##1{\textcolor[rgb]{0.61,0.40,0.00}{##1}}}
\@namedef{PY@tok@k}{\let\PY@bf=\textbf\def\PY@tc##1{\textcolor[rgb]{0.00,0.50,0.00}{##1}}}
\@namedef{PY@tok@kp}{\def\PY@tc##1{\textcolor[rgb]{0.00,0.50,0.00}{##1}}}
\@namedef{PY@tok@kt}{\def\PY@tc##1{\textcolor[rgb]{0.69,0.00,0.25}{##1}}}
\@namedef{PY@tok@o}{\def\PY@tc##1{\textcolor[rgb]{0.40,0.40,0.40}{##1}}}
\@namedef{PY@tok@ow}{\let\PY@bf=\textbf\def\PY@tc##1{\textcolor[rgb]{0.67,0.13,1.00}{##1}}}
\@namedef{PY@tok@nb}{\def\PY@tc##1{\textcolor[rgb]{0.00,0.50,0.00}{##1}}}
\@namedef{PY@tok@nf}{\def\PY@tc##1{\textcolor[rgb]{0.00,0.00,1.00}{##1}}}
\@namedef{PY@tok@nc}{\let\PY@bf=\textbf\def\PY@tc##1{\textcolor[rgb]{0.00,0.00,1.00}{##1}}}
\@namedef{PY@tok@nn}{\let\PY@bf=\textbf\def\PY@tc##1{\textcolor[rgb]{0.00,0.00,1.00}{##1}}}
\@namedef{PY@tok@ne}{\let\PY@bf=\textbf\def\PY@tc##1{\textcolor[rgb]{0.80,0.25,0.22}{##1}}}
\@namedef{PY@tok@nv}{\def\PY@tc##1{\textcolor[rgb]{0.10,0.09,0.49}{##1}}}
\@namedef{PY@tok@no}{\def\PY@tc##1{\textcolor[rgb]{0.53,0.00,0.00}{##1}}}
\@namedef{PY@tok@nl}{\def\PY@tc##1{\textcolor[rgb]{0.46,0.46,0.00}{##1}}}
\@namedef{PY@tok@ni}{\let\PY@bf=\textbf\def\PY@tc##1{\textcolor[rgb]{0.44,0.44,0.44}{##1}}}
\@namedef{PY@tok@na}{\def\PY@tc##1{\textcolor[rgb]{0.41,0.47,0.13}{##1}}}
\@namedef{PY@tok@nt}{\let\PY@bf=\textbf\def\PY@tc##1{\textcolor[rgb]{0.00,0.50,0.00}{##1}}}
\@namedef{PY@tok@nd}{\def\PY@tc##1{\textcolor[rgb]{0.67,0.13,1.00}{##1}}}
\@namedef{PY@tok@s}{\def\PY@tc##1{\textcolor[rgb]{0.73,0.13,0.13}{##1}}}
\@namedef{PY@tok@sd}{\let\PY@it=\textit\def\PY@tc##1{\textcolor[rgb]{0.73,0.13,0.13}{##1}}}
\@namedef{PY@tok@si}{\let\PY@bf=\textbf\def\PY@tc##1{\textcolor[rgb]{0.64,0.35,0.47}{##1}}}
\@namedef{PY@tok@se}{\let\PY@bf=\textbf\def\PY@tc##1{\textcolor[rgb]{0.67,0.36,0.12}{##1}}}
\@namedef{PY@tok@sr}{\def\PY@tc##1{\textcolor[rgb]{0.64,0.35,0.47}{##1}}}
\@namedef{PY@tok@ss}{\def\PY@tc##1{\textcolor[rgb]{0.10,0.09,0.49}{##1}}}
\@namedef{PY@tok@sx}{\def\PY@tc##1{\textcolor[rgb]{0.00,0.50,0.00}{##1}}}
\@namedef{PY@tok@m}{\def\PY@tc##1{\textcolor[rgb]{0.40,0.40,0.40}{##1}}}
\@namedef{PY@tok@gh}{\let\PY@bf=\textbf\def\PY@tc##1{\textcolor[rgb]{0.00,0.00,0.50}{##1}}}
\@namedef{PY@tok@gu}{\let\PY@bf=\textbf\def\PY@tc##1{\textcolor[rgb]{0.50,0.00,0.50}{##1}}}
\@namedef{PY@tok@gd}{\def\PY@tc##1{\textcolor[rgb]{0.63,0.00,0.00}{##1}}}
\@namedef{PY@tok@gi}{\def\PY@tc##1{\textcolor[rgb]{0.00,0.52,0.00}{##1}}}
\@namedef{PY@tok@gr}{\def\PY@tc##1{\textcolor[rgb]{0.89,0.00,0.00}{##1}}}
\@namedef{PY@tok@ge}{\let\PY@it=\textit}
\@namedef{PY@tok@gs}{\let\PY@bf=\textbf}
\@namedef{PY@tok@ges}{\let\PY@bf=\textbf\let\PY@it=\textit}
\@namedef{PY@tok@gp}{\let\PY@bf=\textbf\def\PY@tc##1{\textcolor[rgb]{0.00,0.00,0.50}{##1}}}
\@namedef{PY@tok@go}{\def\PY@tc##1{\textcolor[rgb]{0.44,0.44,0.44}{##1}}}
\@namedef{PY@tok@gt}{\def\PY@tc##1{\textcolor[rgb]{0.00,0.27,0.87}{##1}}}
\@namedef{PY@tok@err}{\def\PY@bc##1{{\setlength{\fboxsep}{\string -\fboxrule}\fcolorbox[rgb]{1.00,0.00,0.00}{1,1,1}{\strut ##1}}}}
\@namedef{PY@tok@kc}{\let\PY@bf=\textbf\def\PY@tc##1{\textcolor[rgb]{0.00,0.50,0.00}{##1}}}
\@namedef{PY@tok@kd}{\let\PY@bf=\textbf\def\PY@tc##1{\textcolor[rgb]{0.00,0.50,0.00}{##1}}}
\@namedef{PY@tok@kn}{\let\PY@bf=\textbf\def\PY@tc##1{\textcolor[rgb]{0.00,0.50,0.00}{##1}}}
\@namedef{PY@tok@kr}{\let\PY@bf=\textbf\def\PY@tc##1{\textcolor[rgb]{0.00,0.50,0.00}{##1}}}
\@namedef{PY@tok@bp}{\def\PY@tc##1{\textcolor[rgb]{0.00,0.50,0.00}{##1}}}
\@namedef{PY@tok@fm}{\def\PY@tc##1{\textcolor[rgb]{0.00,0.00,1.00}{##1}}}
\@namedef{PY@tok@vc}{\def\PY@tc##1{\textcolor[rgb]{0.10,0.09,0.49}{##1}}}
\@namedef{PY@tok@vg}{\def\PY@tc##1{\textcolor[rgb]{0.10,0.09,0.49}{##1}}}
\@namedef{PY@tok@vi}{\def\PY@tc##1{\textcolor[rgb]{0.10,0.09,0.49}{##1}}}
\@namedef{PY@tok@vm}{\def\PY@tc##1{\textcolor[rgb]{0.10,0.09,0.49}{##1}}}
\@namedef{PY@tok@sa}{\def\PY@tc##1{\textcolor[rgb]{0.73,0.13,0.13}{##1}}}
\@namedef{PY@tok@sb}{\def\PY@tc##1{\textcolor[rgb]{0.73,0.13,0.13}{##1}}}
\@namedef{PY@tok@sc}{\def\PY@tc##1{\textcolor[rgb]{0.73,0.13,0.13}{##1}}}
\@namedef{PY@tok@dl}{\def\PY@tc##1{\textcolor[rgb]{0.73,0.13,0.13}{##1}}}
\@namedef{PY@tok@s2}{\def\PY@tc##1{\textcolor[rgb]{0.73,0.13,0.13}{##1}}}
\@namedef{PY@tok@sh}{\def\PY@tc##1{\textcolor[rgb]{0.73,0.13,0.13}{##1}}}
\@namedef{PY@tok@s1}{\def\PY@tc##1{\textcolor[rgb]{0.73,0.13,0.13}{##1}}}
\@namedef{PY@tok@mb}{\def\PY@tc##1{\textcolor[rgb]{0.40,0.40,0.40}{##1}}}
\@namedef{PY@tok@mf}{\def\PY@tc##1{\textcolor[rgb]{0.40,0.40,0.40}{##1}}}
\@namedef{PY@tok@mh}{\def\PY@tc##1{\textcolor[rgb]{0.40,0.40,0.40}{##1}}}
\@namedef{PY@tok@mi}{\def\PY@tc##1{\textcolor[rgb]{0.40,0.40,0.40}{##1}}}
\@namedef{PY@tok@il}{\def\PY@tc##1{\textcolor[rgb]{0.40,0.40,0.40}{##1}}}
\@namedef{PY@tok@mo}{\def\PY@tc##1{\textcolor[rgb]{0.40,0.40,0.40}{##1}}}
\@namedef{PY@tok@ch}{\let\PY@it=\textit\def\PY@tc##1{\textcolor[rgb]{0.24,0.48,0.48}{##1}}}
\@namedef{PY@tok@cm}{\let\PY@it=\textit\def\PY@tc##1{\textcolor[rgb]{0.24,0.48,0.48}{##1}}}
\@namedef{PY@tok@cpf}{\let\PY@it=\textit\def\PY@tc##1{\textcolor[rgb]{0.24,0.48,0.48}{##1}}}
\@namedef{PY@tok@c1}{\let\PY@it=\textit\def\PY@tc##1{\textcolor[rgb]{0.24,0.48,0.48}{##1}}}
\@namedef{PY@tok@cs}{\let\PY@it=\textit\def\PY@tc##1{\textcolor[rgb]{0.24,0.48,0.48}{##1}}}

\def\PYZbs{\char`\\}
\def\PYZus{\char`\_}
\def\PYZob{\char`\{}
\def\PYZcb{\char`\}}
\def\PYZca{\char`\^}
\def\PYZam{\char`\&}
\def\PYZlt{\char`\<}
\def\PYZgt{\char`\>}
\def\PYZsh{\char`\#}
\def\PYZpc{\char`\%}
\def\PYZdl{\char`\$}
\def\PYZhy{\char`\-}
\def\PYZsq{\char`\'}
\def\PYZdq{\char`\"}
\def\PYZti{\char`\~}
% for compatibility with earlier versions
\def\PYZat{@}
\def\PYZlb{[}
\def\PYZrb{]}
\makeatother


    % For linebreaks inside Verbatim environment from package fancyvrb.
    \makeatletter
        \newbox\Wrappedcontinuationbox
        \newbox\Wrappedvisiblespacebox
        \newcommand*\Wrappedvisiblespace {\textcolor{red}{\textvisiblespace}}
        \newcommand*\Wrappedcontinuationsymbol {\textcolor{red}{\llap{\tiny$\m@th\hookrightarrow$}}}
        \newcommand*\Wrappedcontinuationindent {3ex }
        \newcommand*\Wrappedafterbreak {\kern\Wrappedcontinuationindent\copy\Wrappedcontinuationbox}
        % Take advantage of the already applied Pygments mark-up to insert
        % potential linebreaks for TeX processing.
        %        {, <, #, %, $, ' and ": go to next line.
        %        _, }, ^, &, >, - and ~: stay at end of broken line.
        % Use of \textquotesingle for straight quote.
        \newcommand*\Wrappedbreaksatspecials {%
            \def\PYGZus{\discretionary{\char`\_}{\Wrappedafterbreak}{\char`\_}}%
            \def\PYGZob{\discretionary{}{\Wrappedafterbreak\char`\{}{\char`\{}}%
            \def\PYGZcb{\discretionary{\char`\}}{\Wrappedafterbreak}{\char`\}}}%
            \def\PYGZca{\discretionary{\char`\^}{\Wrappedafterbreak}{\char`\^}}%
            \def\PYGZam{\discretionary{\char`\&}{\Wrappedafterbreak}{\char`\&}}%
            \def\PYGZlt{\discretionary{}{\Wrappedafterbreak\char`\<}{\char`\<}}%
            \def\PYGZgt{\discretionary{\char`\>}{\Wrappedafterbreak}{\char`\>}}%
            \def\PYGZsh{\discretionary{}{\Wrappedafterbreak\char`\#}{\char`\#}}%
            \def\PYGZpc{\discretionary{}{\Wrappedafterbreak\char`\%}{\char`\%}}%
            \def\PYGZdl{\discretionary{}{\Wrappedafterbreak\char`\$}{\char`\$}}%
            \def\PYGZhy{\discretionary{\char`\-}{\Wrappedafterbreak}{\char`\-}}%
            \def\PYGZsq{\discretionary{}{\Wrappedafterbreak\textquotesingle}{\textquotesingle}}%
            \def\PYGZdq{\discretionary{}{\Wrappedafterbreak\char`\"}{\char`\"}}%
            \def\PYGZti{\discretionary{\char`\~}{\Wrappedafterbreak}{\char`\~}}%
        }
        % Some characters . , ; ? ! / are not pygmentized.
        % This macro makes them "active" and they will insert potential linebreaks
        \newcommand*\Wrappedbreaksatpunct {%
            \lccode`\~`\.\lowercase{\def~}{\discretionary{\hbox{\char`\.}}{\Wrappedafterbreak}{\hbox{\char`\.}}}%
            \lccode`\~`\,\lowercase{\def~}{\discretionary{\hbox{\char`\,}}{\Wrappedafterbreak}{\hbox{\char`\,}}}%
            \lccode`\~`\;\lowercase{\def~}{\discretionary{\hbox{\char`\;}}{\Wrappedafterbreak}{\hbox{\char`\;}}}%
            \lccode`\~`\:\lowercase{\def~}{\discretionary{\hbox{\char`\:}}{\Wrappedafterbreak}{\hbox{\char`\:}}}%
            \lccode`\~`\?\lowercase{\def~}{\discretionary{\hbox{\char`\?}}{\Wrappedafterbreak}{\hbox{\char`\?}}}%
            \lccode`\~`\!\lowercase{\def~}{\discretionary{\hbox{\char`\!}}{\Wrappedafterbreak}{\hbox{\char`\!}}}%
            \lccode`\~`\/\lowercase{\def~}{\discretionary{\hbox{\char`\/}}{\Wrappedafterbreak}{\hbox{\char`\/}}}%
            \catcode`\.\active
            \catcode`\,\active
            \catcode`\;\active
            \catcode`\:\active
            \catcode`\?\active
            \catcode`\!\active
            \catcode`\/\active
            \lccode`\~`\~
        }
    \makeatother

    \let\OriginalVerbatim=\Verbatim
    \makeatletter
    \renewcommand{\Verbatim}[1][1]{%
        %\parskip\z@skip
        \sbox\Wrappedcontinuationbox {\Wrappedcontinuationsymbol}%
        \sbox\Wrappedvisiblespacebox {\FV@SetupFont\Wrappedvisiblespace}%
        \def\FancyVerbFormatLine ##1{\hsize\linewidth
            \vtop{\raggedright\hyphenpenalty\z@\exhyphenpenalty\z@
                \doublehyphendemerits\z@\finalhyphendemerits\z@
                \strut ##1\strut}%
        }%
        % If the linebreak is at a space, the latter will be displayed as visible
        % space at end of first line, and a continuation symbol starts next line.
        % Stretch/shrink are however usually zero for typewriter font.
        \def\FV@Space {%
            \nobreak\hskip\z@ plus\fontdimen3\font minus\fontdimen4\font
            \discretionary{\copy\Wrappedvisiblespacebox}{\Wrappedafterbreak}
            {\kern\fontdimen2\font}%
        }%

        % Allow breaks at special characters using \PYG... macros.
        \Wrappedbreaksatspecials
        % Breaks at punctuation characters . , ; ? ! and / need catcode=\active
        \OriginalVerbatim[#1,codes*=\Wrappedbreaksatpunct]%
    }
    \makeatother

    % Exact colors from NB
    \definecolor{incolor}{HTML}{303F9F}
    \definecolor{outcolor}{HTML}{D84315}
    \definecolor{cellborder}{HTML}{CFCFCF}
    \definecolor{cellbackground}{HTML}{F7F7F7}

    % prompt
    \makeatletter
    \newcommand{\boxspacing}{\kern\kvtcb@left@rule\kern\kvtcb@boxsep}
    \makeatother
    \newcommand{\prompt}[4]{
        {\ttfamily\llap{{\color{#2}[#3]:\hspace{3pt}#4}}\vspace{-\baselineskip}}
    }
    

    
    % Prevent overflowing lines due to hard-to-break entities
    \sloppy
    % Setup hyperref package
    \hypersetup{
      breaklinks=true,  % so long urls are correctly broken across lines
      colorlinks=true,
      urlcolor=urlcolor,
      linkcolor=linkcolor,
      citecolor=citecolor,
      }
    % Slightly bigger margins than the latex defaults
    
    \geometry{verbose,tmargin=1in,bmargin=1in,lmargin=1in,rmargin=1in}
    
    

\begin{document}
    
    \maketitle
    
    

    
    \section{Ma trận}\label{ma-trux1eadn}

    Đặt \(\mathbb{K} = \mathbb{Q}, \mathbb{R}, \mathbb{C}\)

và \(\mathbb{N} = \{1, \ldots\}\)

với \(m, n \in \mathbb{N}\)

Ma trận \(m \times n\)

\[
A = \begin{pmatrix}
a_{11} & a_{12} & \ldots & a_{1n}
\\
a_{21} & a_{22} & \ldots & a_{2n}
\\
\vdots & \vdots & \ddots & \vdots
\\
a_{m1} & a_{m2} & \ldots & a_{mn}
\end{pmatrix}
\]

Trong đó \(a_{ij} \in \mathbb{K}\)

    \subsection{Loại ma trận}\label{loux1ea1i-ma-trux1eadn}

    \begin{itemize}
\item
  Ma trận hàng.
\item
  Ma trận cột.
\item
  Ma trận \(0\).
\item
  Ma trận vuông \(\Leftarrow\) đường chéo.
\item
  Ma trận bằng nhau.
\item
  \(M(m \times n, \mathbb{K}) = M_{m \times n} (\mathbb{K}) = M_{\mathbb{K}}(m \times n) = \{ A = (a_{ij})_{m \times n} | a_{ij} \in \mathbb{K}\}\).
\item
  Ma trận tam giác trên và Ma trận tam giác dưới, gọi chung là Ma trận
  tam giác.
\item
  Ma trận đường chéo (vừa là mt tam giác trên và dưới).
\item
  Ma trận đơn vị.
\end{itemize}

    \subsection{Phép toán}\label{phuxe9p-touxe1n}

    \subsubsection{Cộng}\label{cux1ed9ng}

    \[A + B = (a_j + b_j)_{m \times n}\]

    \subsubsection{Nhân}\label{nhuxe2n}

    \paragraph{1 số với ma trận}\label{sux1ed1-vux1edbi-ma-trux1eadn}

    \paragraph{2 ma trận}\label{ma-trux1eadn}

    \[A = (a_{ik})_{m \times n}, \: B = (b_{kj})_{n \times p}\]

\[AB = C = (c_{ij})_{m \times p}\]

\[c_{ij} = a_{i1}b_{1j} + a_{i2}b_{2j} + \ldots + a_{in}b_{nj}\]

    \textbf{Thuật toán}

\begin{itemize}
\item
  Check cột \(A\) = hàng \(B\):

  \begin{itemize}
  \item
    Nếu không \(\to\) không giải được.
  \item
    Nếu có \(\to\) Ma trận đầu ra là ma trận (hàng \(A\)) \(\times\)
    (cột \(B\)) \(\to\) Tính \(c_{ij} =\) giao hàng \(A\) và cột \(B\).
  \end{itemize}
\end{itemize}

    \textbf{Note}

\(AB = AC\) và \(A \neq 0\) \(\not\to B = C\).

    \paragraph{Ma trận chuyển vị}\label{ma-trux1eadn-chuyux1ec3n-vux1ecb}

    \[A = (a_{ij})_{m \times n}\]

\[A^T = (a_{ji})_{n \times m}\]

    \subsection{Ma trận bậc thang \& ma trận rút
gọn}\label{ma-trux1eadn-bux1eadc-thang-ma-trux1eadn-ruxfat-gux1ecdn}

    \subsubsection{Ma trận bậc thang}\label{ma-trux1eadn-bux1eadc-thang}

    \begin{itemize}
\item
  Hàng không (nếu có).
\item
  Hàng khác không

  \begin{itemize}
  \tightlist
  \item
    Phần tử chính (PTC) \(\to\) PTC bên dưới luôn nằm bên phải PTC bên
    trên.
  \end{itemize}
\end{itemize}

    \subsubsection{Ma trận rút gọn}\label{ma-trux1eadn-ruxfat-gux1ecdn}

    \begin{itemize}
\item
  PTC \(= 1\)
\item
  Cột chứa PTC \(\to\) PTC là phần tử \(\neq 0\) duy nhất.
\end{itemize}

    \subsection{Phép biến đổi}\label{phuxe9p-biux1ebfn-ux111ux1ed5i}

    \begin{itemize}
\item
  Phép biến đổi sơ cấp, phép biến đổi hàng

  \begin{itemize}
  \item
    Đổi hàng \(h_i \leftrightarrow h_j\)
  \item
    Thay thế tỷ lệ \(h_i \leftarrow \alpha h_i\)
  \item
    Thay thế hàng \(h_i \leftarrow h_i + k h_j \quad (j \neq i)\)
  \end{itemize}
\end{itemize}

    \begin{itemize}
\item
  Tương đương hàng

  \begin{itemize}
  \item
    \(A \to \ldots \to B\), \(B \sim A\)
  \item
    \(A \sim A\)
  \item
    \(A \sim B \to B \sim A\)
  \item
    \(A \sim B, B \sim C \to A \sim C\)
  \end{itemize}
\end{itemize}

    \subsubsection{Thuật toán cho số phép toán thực hiện nhỏ
nhất}\label{thuux1eadt-touxe1n-cho-sux1ed1-phuxe9p-touxe1n-thux1ef1c-hiux1ec7n-nhux1ecf-nhux1ea5t}

    \pandocbounded{\includegraphics[keepaspectratio]{pbdsc_algorithm.png}}

    \subsection{Hạng}\label{hux1ea1ng}

    \[r(A) = \text{rank}(A) \text{ với } A=(a_{ij})_{m \times n}\]

    \begin{itemize}
\tightlist
\item
  Số hàng \(\neq 0\) trong \textbf{dạng rút gọn} (hoặc \textbf{dạng bậc
  thang}) của \(A\) (\(A \sim\) Dạng rút gọn (bậc thang))
\end{itemize}

    \[
0 \leq \text{rank}(A) \leq \min \{ m, n \}
\]

    \subsection{Ma trận khả nghịch}\label{ma-trux1eadn-khux1ea3-nghux1ecbch}

    \[
AB = BA = I_n
\]

    \begin{itemize}
\item
  Trong đó, \(A\) là ma trận vuông cấp \(n\)
\item
  \(A\) là \textbf{ma trận khả nghịch}.
\item
  \(B\) là \textbf{ma trận nghịch đảo} của \(A\).
\item
  \(B = A^{-1}\).
\item
  Nếu tồn tại \(B\), \(B\) là duy nhất.
\end{itemize}

    \textbf{Định lý}

\begin{itemize}
\item
  \((A^{-1})^{-1} = A\).
\item
  \(\alpha A\) khả nghịch và
  \((\alpha A)^{-1} = \frac{1}{\alpha}A^{-1}\).
\item
  \(AB\) khả nghịch và \((AB)^{-1} = B^{-1}A^{-1}\).
\item
  \(A^T\) khả nghịch và \((A^T)^{-1} = (A^{-1})^T\).
\end{itemize}

    \textbf{Note}

    \begin{itemize}
\tightlist
\item
  \(A^k = A.A...A\)

  \begin{itemize}
  \item
    \(A^k\) khả nghịch.
  \item
    \(A^{-k} = (A^k)^{-1}\).
  \end{itemize}
\end{itemize}

    \subsubsection{Ma trận sơ cấp}\label{ma-trux1eadn-sux1a1-cux1ea5p}

    \begin{itemize}
\tightlist
\item
  Ta thực hiện 1 phép biến đổi sơ cấp trên \(I_n\)
\end{itemize}

    \[
I_n \overset{e}{\rightarrow} E
\]

    \begin{itemize}
\item
  \(E\) đgl \textbf{ma trận sơ cấp}
\item
  Tồn tại 3 loại ma trận sơ cấp \(E\) tương ứng với 3 loại phép biến đổi
  sơ cấp.
\end{itemize}

    \[EA = \overset{e}{\leftarrow}A\]

\begin{itemize}
\tightlist
\item
  Trong đó \(\overset{e}{\leftarrow}A\) là ma trận A sau khi đã thực
  hiện phép biến đổi sơ cấp \(e\).
\end{itemize}

    \[A \overset{e_1}{\rightarrow} A_1 \overset{e_2}{\rightarrow} A_2 \ldots \overset{e_k}{\rightarrow} D\]

Ta được

\[
\begin{split}
A_1 &= E_1 A \\
\\
A_2 &= E_2 A_1 = E_2 E_1 A \\
& \vdots
\\
D &= (E_k \ldots E_1) A
\end{split}
\]

Mà \(A\) KN \(\Leftrightarrow \: A \sim I_n\)

Do đó,

\[
A \overset{e_1}{\rightarrow} \overset{e_2}{\rightarrow} \ldots \overset{e_k}{\rightarrow} I_n
\]

Vậy

\[
I_n \overset{e_k}{\rightarrow} \overset{e_{k-1}}{\rightarrow} \ldots \overset{e_1}{\rightarrow} A^{-1}
\]

Hay

\[
\begin{split}
I_n &= (E_1 \ldots E_k) A \\
\\
A^{-1} &= (E_1 \ldots E_k) I_n
\end{split}
\]

    \subsubsection{Thuật toán tìm ma trận khả
nghịch}\label{thuux1eadt-touxe1n-tuxecm-ma-trux1eadn-khux1ea3-nghux1ecbch}

    Cho \(A_{n \times n}\)

\begin{itemize}
\item
  B1. Thiết lập \((A | I_n)\)
\item
  B2.
  \((A | I_n) \overset{\text{Biến đổi thành ma trận rút gọn}}{\to} (D|B)\)

  \begin{itemize}
  \item
    Nếu \(D = I_n\) \(\to A\) khả nghịch và \(A^{-1}=B\).
  \item
    Nếu \(D \neq I_n\) \(\to A\) không khả nghịch.
  \end{itemize}
\end{itemize}

    \subsubsection{Tính chất}\label{tuxednh-chux1ea5t}

    \begin{itemize}
\item
  \(A \in M(n \times n, \mathbb{K})\)
\item
  \(A\) khả nghịch.
\item
  \(r(A) = n\).
\item
  \(A\) là tích hữu hạn các ma trận sơ cấp

  \begin{itemize}
  \item
    \(I_n = (E_1 \ldots E_k) A\)
  \item
    \(A = E_{k}^{-1} \ldots E_1^{-1}\)
  \end{itemize}
\item
  \(AX=B\) có nghiệm duy nhất
  \(\forall B \in M(n \times p, \mathbb{K})\).
\item
  \(\exists B\) ma trận vuông cấp \(n\) sao cho \(AB=I_n\).
\item
  \(\exists C\) ma trận vuông cấp \(n\) sao cho \(CA = I_n\)
\item
  \(A^T\) khả nghịch.
\end{itemize}

    \section{Định thức}\label{ux111ux1ecbnh-thux1ee9c}

    \subsection{Nền tảng}\label{nux1ec1n-tux1ea3ng}

    \subsubsection{Phép thế}\label{phuxe9p-thux1ebf}

    \begin{itemize}
\item
  Cho \(X = \{ 1, 2, \ldots , n \}\)
\item
  Song ánh \(\sigma : X \to X\) đgl phép thế bậc \(n\).
\end{itemize}

\[
\sigma = 
\begin{pmatrix}
1 & 2 & \ldots & n \\
\sigma(1) & \sigma(2) & \ldots & \sigma(n)
\end{pmatrix}
\]

\begin{itemize}
\tightlist
\item
  Tập hợp các phép thế bậc \(n\) k/h \(|S_n|=n!\).
\end{itemize}

\[
S_n = \{ \sigma : X \to X | \sigma \text{ là song ánh} \}
\]

\begin{itemize}
\item
  Phép thế đơn vị
\item
  Phép thế sơ cấp
\item
  Cấu trúc

  \begin{itemize}
  \item
    Mỗi phép thế đều phân tích được thành tích của các tích độc lập
  \item
    Tích phép thế sơ cấp.
  \end{itemize}
\end{itemize}

    \subsubsection{Dấu}\label{dux1ea5u}

    \[
\text{sgn}(\sigma) = \underset{{1 \leq i < j \leq n}}{\pi} \frac{\sigma (i) - \sigma (j)}{i - j} \quad \in \{ \pm 1 \}
\]

    \subsubsection{Nghịch thế}\label{nghux1ecbch-thux1ebf}

    \begin{itemize}
\tightlist
\item
  Là số lượng \(\sigma(i) - \sigma (j)\) ngược với \(i - j\) hay số
  lượng
\end{itemize}

\[
\frac{\sigma(i) - \sigma(j)}{i - j} < 0
\]

\begin{itemize}
\item
  Nếu số lượng nghịch thế

  \begin{itemize}
  \item
    Chẳn \(\to \text{sgn}(\sigma) = 1\).
  \item
    Lẻ \(\to \text{sgn}(\sigma) = -1\).
  \end{itemize}
\end{itemize}

    \begin{itemize}
\tightlist
\item
  \textbf{Note:} Phép thế sơ cấp là phép thế lẻ.
\end{itemize}

    \subsection{Công thức}\label{cuxf4ng-thux1ee9c}

    \begin{itemize}
\tightlist
\item
  Cho \(A = (a_{ij})_{n \times n}\)
\end{itemize}

\[
\det(A) = |A| = \sum_{\sigma \in S_n} \text{sgn} (\sigma) a_{\sigma (1)1} a_{\sigma (2)2} \ldots a_{\sigma (n) n}
\]

    \begin{itemize}
\item
  Cấp 2
\item
  Cấp 3
\end{itemize}

    \subsection{Tính chất - hệ quả - định
lý}\label{tuxednh-chux1ea5t---hux1ec7-quux1ea3---ux111ux1ecbnh-luxfd}

    \subsubsection{Tính chất}\label{tuxednh-chux1ea5t}

    \begin{itemize}
\tightlist
\item
  Đa tuyến tính
\end{itemize}

\[
\det(A_1 \ldots \alpha A_j + \beta A'_j \ldots A_n) = \alpha \det(A_1 \ldots A_j \ldots A_n) + \beta \det(A_1 \ldots A'_j \ldots A_n)
\]

\begin{itemize}
\tightlist
\item
  Thay phiên
\end{itemize}

\[
\det(A_1 \ldots A_i \ldots A_i \ldots A_n) = 0
\]

\begin{itemize}
\tightlist
\item
  Chuẩn hoá
\end{itemize}

\[
\det(I_n) = 1
\]

    \subsubsection{Hệ quả}\label{hux1ec7-quux1ea3}

    \begin{itemize}
\item
  \(\det(A_1 \ldots A_i \ldots A_j \ldots A_n) = \det(A_1 \ldots A_i + \alpha A_j \ldots A_j \ldots A_n\).
\item
  \(\det(A_1 \ldots A_i \ldots \alpha A_i \ldots A_n) = 0\).
\item
  \(\det(A_1 \ldots A_i \ldots A_j \ldots A_n) = - \det(A_1 \ldots A_j \ldots A_i \ldots A_n)\)

  \begin{itemize}
  \item
    Đổi chỗ chẳn lần \(\to 1\).
  \item
    Đổi chỗ lẻ lần \(\to -1\).
  \end{itemize}
\end{itemize}

    \subsubsection{Định lý}\label{ux111ux1ecbnh-luxfd}

    \begin{itemize}
\item
  \(\det(A^T) = \det(A)\).
\item
  \(\det(AB) = \det(A) \det(B) = \det(BA)\)
\item
  \(A\) khả nghịch \(\Leftrightarrow \det(A) \neq 0\)
\end{itemize}

\[
\det(A^{-1}) = \frac{1}{\det(A)}
\]

    \subsection{Định thức con và phần bù đại
số}\label{ux111ux1ecbnh-thux1ee9c-con-vuxe0-phux1ea7n-buxf9-ux111ux1ea1i-sux1ed1}

    \(i \leq k \leq n\)

Chọn \(1 \leq i_1 \leq i_2 \leq \ldots \leq i_k \leq n\) và
\(1 \leq j_1 \leq j_2 \leq \ldots \leq j_k \leq n\)

    \begin{itemize}
\item
  \(D_{i_1 \ldots i_k}^{j_1 \ldots j_k}\) là định thức con.
\item
  \(\overline{D_{i_1 \ldots i_k}^{j_1 \ldots j_k}}\)

  \begin{itemize}
  \item
    Đgl phần bù của \(D_{i_1 \ldots i_k}^{j_1 \ldots j_k}\).
  \item
    Ma trận sau khi bỏ hàng \(i_k\) và cột \(j_k\) từ
    \(D_{i_1 \ldots i_k}^{j_1 \ldots j_k}\).
  \end{itemize}
\end{itemize}

    \begin{itemize}
\tightlist
\item
  Lấy theo cột, \(1 \leq j_1 \leq j_2 \leq \ldots \leq j_k \leq n\)
\end{itemize}

\[
\det (A) = \sum_{1 \leq i_1 \leq i_2 \leq \ldots \leq i_k \leq n} (-1)^{i_1 + \ldots + i_k + j_1 + \ldots + j_k} D_{i_1 \ldots i_k}^{j_1 \ldots j_k} \overline{D_{i_1 \ldots i_k}^{j_1 \ldots j_k}}
\]

    Với \(k=1\),

\[
\det(A) = (-1)^{1+j} a_{1j} \overline{D_1^j} + (-1)^{2+j} a_{2j} \overline{D_2^j} + \ldots + (-1)^{n+j} a_{nj} \overline{D_n^j} = \sum_{i=1}^n (-1)^{i+j} a_{ij} \overline{D_i^j}
\]

    \begin{itemize}
\tightlist
\item
  Lấy theo hàng, \(1 \leq i_1 \leq i_2 \leq \ldots \leq i_k \leq n\)
\end{itemize}

\[
\det (A) = \sum_{1 \leq j_1 \leq j_2 \leq \ldots \leq j_k \leq n} (-1)^{i_1 + \ldots + i_k + j_1 + \ldots + j_k} D_{i_1 \ldots i_k}^{j_1 \ldots j_k} \overline{D_{i_1 \ldots i_k}^{j_1 \ldots j_k}}
\]

    Với \(k=1\),

\[
\det(A) = (-1)^{i+1} a_{i1} \overline{D_i^1} + (-1)^{i+2} a_{i2} \overline{D_i^2} + \ldots + (-1)^{i+n} a_{in} \overline{D_i^n} = \sum_{j=1}^n (-1)^{i+j} a_{ij} \overline{D_i^j}
\]

    \begin{itemize}
\item
  \textbf{Note}

  \begin{itemize}
  \item
    Ưu tiên chọn cột (hoặc hàng) nhiều \(0\). Nếu không có \(0 \to\) kết
    hợp các tính chất, định lý và hệ quả trước đó \(\to\) sinh ra \(0\).
  \item
    Không được dùng \(h_i \leftarrow \alpha h_i\)
    (\(h_i \leftrightarrow h_j\) hay \(c_i \leftrightarrow c_j\) thì nhớ
    hệ quả đổi dấu).
  \item
    Chọn cột để biến đổi thành ``\(0 \ldots 1\)'' thì dùng phép biến đổi
    trên hàng và ngược lại.
  \end{itemize}
\end{itemize}

    \subsection{Ứng dụng}\label{ux1ee9ng-dux1ee5ng}

    \subsubsection{Khả nghịch}\label{khux1ea3-nghux1ecbch}

    \begin{itemize}
\tightlist
\item
  \(A\) khả nghịch thì
\end{itemize}

\[
A^{-1} = \frac{1}{\det(A)} (\text{adj} (A))^T = \frac{1}{\det(A)} C^T
\]

    trong đó, \(A\) ma trận vuông cấp \(n\) và

    \[
\text{adj}(A) = C =
\begin{pmatrix}
c_{11} & c_{12} & \ldots & c_{1n} \\
c_{21} & c_{22} & \ldots & c_{2n} \\
\vdots & \vdots & \ddots & \vdots \\
c_{n1} & c_{n2} & \ldots & c_{nn}
\end{pmatrix}
= \text{cof}(A) 
\]

    trong đó, \(c_{ij}\) là phần bù đại số của \(a_{ij}\)

    \[
c_{ij} = (-1)^{i+j} \overline{D_i^j}
\]

    \textbf{Thuật toán xác định ma trận khả nghịch và tìm ma trận nghịch đảo
bằng định thức}

    B1. Tính \(c\) trên hàng (hoặc cột)

B2. Áp dụng công thức tính \(\det(A)\) trên hàng (hoặc cột) bằng công
thức

\[\det(A) = a_{11} c_{11} + \ldots + a_{1n}c_{1n}\]

B3. (Check khả nghịch)

\begin{itemize}
\tightlist
\item
  Nếu \(\det(A) = 0 \to\) không khả nghịch và kết thúc.
\item
  Nếu \(\det(A) \neq 0 \to A\) khả nghịch và chuyển sang bước 4 (nếu cần
  tính ma trận nghịc đảo)
\end{itemize}

B4. Tính hết tất cả \(c\) còn lại \(\to \text{adj} (A) = C\)

B5. Tìm \(A^{-1}\) bằng công thức

\[
A^{-1} = \frac{1}{\det(A)} (\text{adj} (A))^T = \frac{1}{\det(A)} C^T
\]

    \subsubsection{Hạng}\label{hux1ea1ng}

    \textbf{Thuật toán xác định hạng của ma trận bằng định thức}

    B1. Tính định thức cấp \(1,2,\ldots,n\)

B2. (Kết luận hạng)

\begin{itemize}
\item
  Nếu định thức cấp \(i \neq 0 \to\) (hàng (hoặc cột) độc lập tuyến tính
  \(\to\) dạng bậc thang \(100\%\) có \(n\) hàng \(\neq 0\))
  \(\to \text{rank} (A) \geq i\)
\item
  Nếu định thức cấp \(i = 0 \to \text{rank} (A) = i\)
\end{itemize}

    \section{Hệ phương trình tuyến
tính}\label{hux1ec7-phux1b0ux1a1ng-truxecnh-tuyux1ebfn-tuxednh}

    \begin{itemize}
\tightlist
\item
  \(m, n \in \mathbb{N}\), trong đó, \(m\) phương trình, \(n\) ẩn, ta có
  hệ phương trình tuyến tính tổng quát (1)
\end{itemize}

    \[
\begin{cases}
a_{11}x_{1} + a_{12}x_{2} + \cdots + a_{1n}x_{n} = b_{1} \\
a_{21}x_{1} + a_{22}x_{2} + \cdots + a_{2n}x_{n} = b_{2} \\
\quad \vdots \\
a_{m1}x_{1} + a_{m2}x_{2} + \cdots + a_{mn}x_{n} = b_{m}
\end{cases}
\]

    \begin{itemize}
\item
  trong đó

  \begin{itemize}
  \item
    \(a_{ij}\) là hệ số.
  \item
    \(b_i\) là hệ số tự do.
  \item
    \(x_j\) là ẩn của hệ.
  \end{itemize}
\item
  Bộ số \((c_1, c_2, \ldots , c_n)\) là nghiệm của (1) nếu thay vào (1)
  thoả tất cả phương trình.
\item
  Giải hệ (1) \(\to\) tìm tập nghiệm của (1).
\item
  Hệ pt có nghiệm đgl hệ tương thích.
\end{itemize}

    \begin{itemize}
\tightlist
\item
  Ma trận hệ số của (1)
\end{itemize}

\[
A = \begin{pmatrix}
a_{11} & a_{12} & \cdots & a_{1n} \\
a_{21} & a_{22} & \cdots & a_{2n} \\
\vdots & \vdots & \ddots & \vdots \\
a_{m1} & a_{m2} & \cdots & a_{mn}
\end{pmatrix}
\]

    \begin{itemize}
\tightlist
\item
  Cột ẩn
\end{itemize}

\[
X = \begin{pmatrix}
x_{1} \\
x_{2} \\
\vdots \\
x_{n}
\end{pmatrix}
\]

    \begin{itemize}
\tightlist
\item
  Cột hệ số tự do
\end{itemize}

\[
b = \begin{pmatrix}
b_{1} \\
b_{2} \\
\vdots \\
b_{m}
\end{pmatrix}
\]

    \begin{itemize}
\tightlist
\item
  Ta viết gọn
\end{itemize}

\[
AX = b
\]

    \begin{itemize}
\tightlist
\item
  Ma trận đầy đủ (ma trận bổ sung) k/h
\end{itemize}

\[
A^* = (A | b)
\]

    \subsection{Sự tồn tại và tính duy
nhất}\label{sux1ef1-tux1ed3n-tux1ea1i-vuxe0-tuxednh-duy-nhux1ea5t}

    \begin{itemize}
\tightlist
\item
  \(A^* = (A|b) \to (S|c)\), trong đó \(S\) là dạng bậc thang (hoặc dạng
  rút gọn) của ma trận \(A^*\)
\end{itemize}

    \begin{itemize}
\item
  \textbf{Điều kiện nghiệm:}

  \begin{itemize}
  \item
    Vô nghiệm \(\to \exists\) hàng có dạng
    \((0 \ldots 0 | c), \: c_i \neq 0\).
  \item
    Vô số nghiệm \(\to\) Phần tử chính \(<\) Số ẩn.
  \item
    Có nghiệm duy nhất \(\to\) Phần tử chính \(=\) Số ẩn.
  \end{itemize}
\end{itemize}

    \begin{itemize}
\item
  \textbf{Định lý Kronecker - Capelli}

  \begin{itemize}
  \item
    Hệ vô nghiệm \(\Leftrightarrow \text{rank}(A) < \text{rank}(A^*).\)
  \item
    Hệ có nghiệm \(\Leftrightarrow \text{rank}(A) = \text{rank}(A^*)\):

    \begin{itemize}
    \item
      Có nghiệm duy nhất \(\to \text{rank}(A) =\) Số ẩn.
    \item
      Vô số nghiệm \(\to \text{rank}(A) <\) Số ẩn.
    \end{itemize}
  \end{itemize}
\end{itemize}

    \begin{itemize}
\item
  \textbf{Ẩn phụ thuộc} \(\to\) Là ẩn nằm trong cột chứa PTC trong dạng
  bậc thang (rút gọn).
\item
  \textbf{Ẩn tự do (độc lập)} \(\to\) Các ẩn còn lại.
\end{itemize}

    \textbf{Note}

\begin{itemize}
\item
  Ta kết luận nghiệm theo kiểu

  \begin{itemize}
  \item
    \(X = (x_1, x_2 , \ldots , x_n)\).
  \item
    \(X = \begin{pmatrix} x_1 \\ x_2 \\ \vdots \\ x_n \end{pmatrix}\).
  \item
    \(\begin{cases} x_1 = 1 \\ x_2 = 2 \\ \vdots \\ x_n = 3 \end{cases}\)
  \end{itemize}
\item
  Nghiệm riêng \(\to\) Đề cho sẵn ẩn tự do bằng mấy.
\item
  Nghiệm tổng quát \(\to\) Không cho sẵn ẩn tự do bằng mấy.
\end{itemize}

    \subsection{Thuật toán Gauss}\label{thuux1eadt-touxe1n-gauss}

    B1. Đưa về \(A^*\)

B2. Đưa \(A^* \to (S|c)\) (Bậc thang)

B3. (Check tồn tại nghiệm)

\begin{itemize}
\item
  Nếu không có nghiệm \(\to\) Kết luận.
\item
  Nếu có nghiệm \(\to\) Dạng rút gọn \(\to\) KL nghiệm.
\end{itemize}

    \subsection{Quy tắc Cramer}\label{quy-tux1eafc-cramer}

    \begin{itemize}
\item
  Hệ \(AX=b\) đgl \textbf{Hệ Cramer} nếu

  \begin{itemize}
  \item
    \(A\) vuông
  \item
    \(A\) Khả nghịch
  \end{itemize}
\end{itemize}

    Do đó,

    \[
\begin{split}
X &= A^{-1} b \\
\begin{pmatrix} x_1 \\ x_2 \\ \vdots \\ x_n \end{pmatrix} &= \frac{1}{\det(A)} C^T \begin{pmatrix}
b_{1} \\
b_{2} \\
\vdots \\
b_{m}
\end{pmatrix}
\end{split}
\]

    hay \(1 \leq j \leq n\)

    \[
\begin{split}
x_j &= \frac{1}{\det (A)} (b_1 c_{1j} + b_2 c_{2j} + \ldots + b_n c_{nj}) \\
A_j(b) &= \begin{pmatrix}
a_{11} & \cdots & b_1 & a_{1n} \\
a_{21} & \cdots & b_2 & a_{2n} \\
\vdots & \vdots & \ddots & \vdots \\
a_{m1} & \cdots & b_m & a_{mn}
\end{pmatrix}
\end{split}
\]

    Do đó ta có

    \[
D_j = \det (A_j (b)) = b_1 c_{1j} + \ldots + b_m c_{mj}
\]

    nghiệm của hệ là

    \[
x_j = \frac{D_j}{D}, \quad 1 \leq j \leq n
\]

    \begin{itemize}
\item
  \textbf{Điều kiện nghiệm:}

  \begin{itemize}
  \item
    Nghiệm duy nhất \(x_j = \frac{D_j}{D} \to D \neq 0\).
  \item
    Vô nghiệm \(\to D = 0 \land \exists D_j \neq 0\).
  \item
    Không kết luật gì về tương thích hệ
    \(\to D = 0 \land \forall D_j = 0\).
  \end{itemize}
\end{itemize}

    \textbf{Thuật toán tìm nghiệm của hệ bằng quy tắc Cramer}

    B1. Tìm \(D\) và \(D_j, \: 1 \leq j \leq n\)

B2. (Check điều kiện nghiệm)

\begin{itemize}
\item
  Nếu \(D \neq 0 \to\) KL nghiệm duy nhất
  \(x_j = \frac{D_j}{D}, 1 \leq j \leq n \to\) Kết thúc.
\item
  Nếu \(D = 0 \to\) Sang bước tiếp theo
\end{itemize}

B3. (Check trường hợp)

\begin{itemize}
\item
  Nếu \(\exists D_j \neq 0 \to\) KL vô nghiệm \(\to\) Kết thúc.
\item
  Nếu \(\forall D_j = 0 \to\) KL rằng ta không kết luật gì về tương
  thích hệ \(\to\) Kết thúc.
\end{itemize}

    \textbf{Note}

\begin{itemize}
\tightlist
\item
  Nếu đề hỏi tìm tham số sao cho hệ có nghiệm thì ta tìm trường hợp vô
  số và nghiệm duy nhất.
\end{itemize}

    \subsection{Hệ thuần nhất}\label{hux1ec7-thuux1ea7n-nhux1ea5t}

    \begin{itemize}
\tightlist
\item
  Hệ thuần nhất là hệ (2) như sau
\end{itemize}

    \[
AX = 0
\]

    \begin{itemize}
\item
  Nghiệm tầm thường \(\to X = 0 \to \text{rank}(A) = n\).
\item
  Vô số nghiệm phụ thuộc vào
  \(n - \text{rank}(A) \to \text{rank}(A) < n\).
\end{itemize}

    \begin{itemize}
\item
  \textbf{Nhận xét}

  \begin{itemize}
  \item
    Nếu \(x = (x_1, x_2, \ldots , x_n)\) là nghiệm của (2)
    \(\to tx = (tx_1, tx_2, \ldots , t x_n)\) là nghiệm của (2).
  \item
    Nếu \(x = (x_1, x_2, \ldots , x_n)\) và
    \(y = (y_1, y_2, \ldots , y_n)\) là nghiệm của (2)
    \(\to x+ y = (x_1 + y_1, x_2 + y_2, \ldots , x_n + y_n)\) là nghiệm
    của (2).
  \item
    \(A\) có \(m <n \to\) Vô số nghiệm.
  \end{itemize}
\end{itemize}

    \section{Không gian Vector}\label{khuxf4ng-gian-vector}

    \begin{itemize}
\item
  \(V \neq \varnothing\) đgl 1 \textbf{không gian vector}
  \(\Rightarrow\) được trang bị 2 phép toán:

  \begin{itemize}
  \item
    Cộng vector
    \(+\colon V\times V \longrightarrow V, \quad (u,v)\longmapsto u+v.\)
  \item
    Nhân vô hướng
    \(\cdot \colon \mathbb{K} \times V \longrightarrow V,\quad (\alpha, u) \longmapsto \alpha u.\)
  \end{itemize}
\item
  Thoả mãn các tiên đề sau:
\end{itemize}

\[
\begin{split}
&(1) \quad (u + v) + w = u + (v + w), \quad \forall u,v,w \in V \\
&(2) \quad u+v=v+u,\quad \forall u,v \in V \\
&(3) \quad \exists 0 \in V : u+0=u,\quad \forall u \in V \\
&(4) \quad \forall u \in X,\: \exists u' \in V:u+u'=0 \\
&(5) \quad \alpha (u + v) = \alpha u + \alpha v,\quad \forall \alpha \in \mathbb{K}, \forall u, v \in V \\
&(6) \quad (\alpha + \beta)u=\alpha u+\beta u,\quad \forall \alpha ,\beta \in \mathbb{K},\forall u \in V \\
&(7) \quad \alpha (\beta u) = (\alpha \beta)u,\quad \alpha , \beta \in \mathbb{K},\forall u \in V\\
&(8) \quad 1u = u,\quad \forall u \in V
\end{split}
\]

\begin{itemize}
\item
  \textbf{Note}:

  \begin{itemize}
  \item
    \textbf{\(\mathbb{K}\)-Không gian vector} \(\leftrightarrow\) Không
    gian vector trên trường \(\mathbb{K}\).
  \item
    \(V\) là không gian vector trên trường
    \(\mathbb{K} \leftrightarrow v \in V\) là 1 \textbf{vector}.
  \item
    \(\alpha \in \mathbb{K} \to\) đgl 1 \textbf{vô hướng}.
  \item
    \(0\) trong \((3) \to\) đgl \textbf{vector 0}.
  \item
    \(\forall u \in V, u'\) trong \((4) \to\) đgl vector đối của vector
    \(u, \quad u'=-u\).
  \end{itemize}
\end{itemize}

    \subsection{Tính chất}\label{tuxednh-chux1ea5t}

    Cho \(V\) là 1 không gian vector

    \begin{itemize}
\item
  Vector 0 là duy nhất.
\item
  \(\forall u \in V, \: -u\) là duy nhất.

  \begin{itemize}
  \tightlist
  \item
    \(u-v=u+(-v).\)
  \end{itemize}
\item
  Giản ước:

  \begin{itemize}
  \item
    \(u+w = v+w \Rightarrow u=v.\)
  \item
    \(u+v=w \Rightarrow u = w-v.\)
  \end{itemize}
\item
  \(\alpha u = 0 \Rightarrow \alpha = 0 \lor u=0.\)
\item
  \((- \alpha )u=\alpha (-u) = -(\alpha u).\)
\item
  \(\forall u \in V, \forall \alpha \in \mathbb{K}\) (\(0\) bên dưới là
  vector 0)

  \begin{itemize}
  \item
    \(0u=0.\)
  \item
    \(\alpha 0=0.\)
  \end{itemize}
\end{itemize}

    \begin{itemize}
\item
  \textbf{Note}:

  \begin{itemize}
  \tightlist
  \item
    Để c/m \textbf{tính duy nhất} \(\to\) ta c/m cả 2 thoả cùng 1 tính
    chất.
  \end{itemize}
\end{itemize}

    \subsection{Ví dụ minh hoạ}\label{vuxed-dux1ee5-minh-houx1ea1}

    \begin{itemize}
\tightlist
\item
  \textbf{VD1}
\end{itemize}

\(V=\{\) Vector tự do trong mặt phẳng \(\},+,.\) là 1 không gian vector.

\begin{itemize}
\tightlist
\item
  \textbf{VD2}
\end{itemize}

\(\mathbb{R}^2 = \{ (x,y) | x,y \in \mathbb{R} \}\)

\((x,y)+(x'+y')=(x+x',y+y')\)

\(\alpha (x,y) = (\alpha x, \alpha y)\)

là 1 không gian vector.

\begin{itemize}
\tightlist
\item
  \textbf{VD3}
\end{itemize}

\(n \geq 1, \: \mathbb{K}^n =\{ (x_1, \ldots ,x_n) | x_i \in \mathbb{K} \}\)

\((x_1, \ldots ,x_n)+(y_1,\ldots ,y_n)=(x_1+y_1,\ldots ,x_n+y_n)\)

\(\alpha (x_1,\ldots ,x_n) = (\alpha x_1, \ldots ,\alpha x_n)\)

là 1 không gian vector.

\begin{itemize}
\tightlist
\item
  \textbf{VD4}
\end{itemize}

\(M(m \times n, \mathbb{K}),+,.\) là không gian vector.

\begin{itemize}
\tightlist
\item
  \textbf{VD5}
\end{itemize}

\(\mathbb{K}[x]\) là tập các đa thức biến \(x\) hệ số trong
\(\mathbb{K}\). \(\mathbb{K}[x]\) là không gian vector trên
\(\mathbb{K}\) với phép cộng các đa thức và phép nhân 1 số với đa thức.
(\(0\) là đa thức không)

\begin{itemize}
\tightlist
\item
  \textbf{VD6}
\end{itemize}

Tập các hàm thực xác định trên \(\mathbb{R}\) là \(\mathbb{R}\)-Không
gian vector với

\((f+g)(x) = f(x) + g(x)\)

\((\alpha f)(x) = \alpha f(x)\)

\begin{itemize}
\tightlist
\item
  \textbf{VD7}
\end{itemize}

\(\mathbb{K}\) là 1 không gian vector với các phép toán thông thường
trên \(\mathbb{R},\: \mathbb{Q},\: \mathbb{R},\: \mathbb{C}.\)

\begin{itemize}
\tightlist
\item
  \textbf{VD8}
\end{itemize}

\((\mathbb{R} \times \mathbb{R} \to \mathbb{R}, \quad \mathbb{Q} \times \mathbb{R} \to \mathbb{R}) \to \mathbb{R}\)
là \(\mathbb{Q}\)-không gian vector.

    \subsection{\texorpdfstring{Độc lập tuyến tính \(\&\) phụ thuộc tuyến
tính}{Độc lập tuyến tính \textbackslash\& phụ thuộc tuyến tính}}\label{ux111ux1ed9c-lux1eadp-tuyux1ebfn-tuxednh-phux1ee5-thuux1ed9c-tuyux1ebfn-tuxednh}

    Cho \(S \subset V\) với \(V\) là không gian vector.

    \subsubsection{Tổ hợp tuyến
tính}\label{tux1ed5-hux1ee3p-tuyux1ebfn-tuxednh}

    \begin{itemize}
\tightlist
\item
  \textbf{Tổ hợp tuyến tính} của các vector trong \(S\) là tổng hữu hạn
\end{itemize}

\[
a_1u_1+a_2u_2+\ldots +a_nu_n, \quad a_i\in \mathbb{K},u_i\in S.
\]

    \subsubsection{Biểu thị tuyến
tính}\label{biux1ec3u-thux1ecb-tuyux1ebfn-tuxednh}

    \begin{itemize}
\tightlist
\item
  Cho \(v\in V\), \textbf{Biểu thị tuyến tính} là
\end{itemize}

\[
v = \alpha _1u_1 + \alpha _2u_2 +\ldots + \alpha _k u_k,\quad \alpha _k \in \mathbb{K},u_k \in S.
\]

    \begin{itemize}
\tightlist
\item
  \textbf{VDMH}
\end{itemize}

\(\mathbb{R}^3=S=\{ x_1=(1,2,1), v_2=(1,2,3), u_3=(2,4,1) \}, v=(1,2,2), w=(0,0-3)\),
\(v\) và \(w\) biểu thị tuyến tính được qua \(S\) không?

\(v = x_1u_1+x_2u_2+x_3u_3\)

\((1,2,3)=x_1(\)

    \subsubsection{Phụ thuộc tuyến
tính}\label{phux1ee5-thuux1ed9c-tuyux1ebfn-tuxednh}

    \begin{itemize}
\tightlist
\item
  \(S\) là \textbf{phụ thuộc tuyến tính} nếu
\end{itemize}

\[\exists (\alpha _1, \alpha _2, \ldots , \alpha _k) \neq (0,\ldots 0)\: |\: \alpha _1u_1 + \alpha _2u_2 + \ldots + \alpha _ku_k = 0, \quad u_i\in S.\]

    \subsubsection{Độc lập tuyến
tính}\label{ux111ux1ed9c-lux1eadp-tuyux1ebfn-tuxednh}

    \begin{itemize}
\tightlist
\item
  \(S\) là \textbf{độc lập tuyến tính} nếu không phụ thuộc tuyến tính,
  hay
\end{itemize}

\[
(\alpha _1u_1 + \alpha _2u_2 + \ldots + \alpha _ku_k = 0) \Rightarrow (\alpha _1 = \alpha _2 = \ldots = \alpha _k = 0) \to \text{ Nghiệm tầm thường}.
\]

    \subsubsection{VDMH}\label{vdmh}

    \begin{itemize}
\tightlist
\item
  \textbf{VD1}
\end{itemize}

Trong không gian
\(\mathbb{R}^3, S = \{ u_1 = (1,2,-1), u_2=(\sim \sim \sim), u_3 = (\sim \sim \sim)\}\)

Xét \(\alpha _1u_1 + \alpha _2u_2 + \alpha _3u_3 = 0\)

\(\alpha _1 (1,2,1)+ \alpha _2 (\sim \sim \sim) + \alpha _3 (\sim \sim \sim) = 0\)

\(\begin{cases} \alpha _1 + 2 \alpha _2+3\alpha _3 = 0 \\ \sim \sim \sim =0\\ \sim \sim \sim=0\end{cases} \Rightarrow A = \begin{pmatrix} \sim \sim \sim \\ \sim \sim \sim \\ \sim \sim \sim \end{pmatrix} \Rightarrow \text{ Vô số nghiệm} \Rightarrow \text{ Phụ thuộc tuyến tính}.\)

\begin{itemize}
\tightlist
\item
  \textbf{VD2}
\end{itemize}

Trong không gian
\(\mathbb{R}[x], S = \{ P_1 = 1+x+x^3, P_2 = 1+x+x^2, P_3=1+x, P_4=x+x^3 \}\)

Xét \(x_1P_1 +x_2P_2 +x_3P_3 =0\)

\(x_1(1+x+x^3) + x_2(\sim \sim \sim) + x_3(\sim \sim \sim)=0\)

\(\begin{cases} x_1 + x_2 +x_3 = 0 \\ \sim \sim \sim =0\\ \sim \sim \sim=0\\ \sim \sim \sim=0\end{cases} \Rightarrow \text{ Có nghiệm } (0,0,0,0) \Rightarrow \text{ Độc lập tuyến tính}.\)

(Note: HPT ở trên được lập ra dựa trên bậc, tức mỗi cột tương ứng với
bậc)

    \subsubsection{Tính chất}\label{tuxednh-chux1ea5t}

    \begin{itemize}
\item
  Hệ chỉ gồm 1 vector \(\{ u \}\) phụ thuộc tuyến tính
  \(\Leftrightarrow u =0\).
\item
  \(S\) phụ thuộc tuyến tính

  \begin{itemize}
  \tightlist
  \item
    \(\Rightarrow\) Mọi hệ chứa \(S\) phụ thuộc tuyến tính.
  \item
    \(\Rightarrow\) Mọi hệ vector chứa vector \(0\) đều phụ thuộc tuyến
    tính.
  \end{itemize}
\item
  \(S\) phụ thuộc tuyến tính, \(S' \supset S \Rightarrow S'\) phụ thuộc
  tuyến tính.
\item
  Hệ vô hạn vector \(S\) độc lập tuyến tính \(\Leftrightarrow\) Hệ con
  hữu hạn của \(S\) độc lập tuyến tính.
\end{itemize}

    \subsubsection{Định lý}\label{ux111ux1ecbnh-luxfd}

    \begin{itemize}
\tightlist
\item
  Cho \(S = \{ u_1 , u_2 \ldots , u_k \}, \: k \geq 2\), \(S\)
  \textbf{phụ thuộc tuyến tính}
\end{itemize}

\[\Leftrightarrow \exists u_i, 1 \leq i \leq k \: | \: u_i = \alpha _1u_1 + \ldots + \widehat{\alpha _iu_i} + \ldots + \alpha _k u_k.\]

    \begin{itemize}
\item
  \textbf{Cách làm bài tìm tổ hợp tuyến tính}

  \begin{itemize}
  \item
    B1. Xử lý đến khi ra được hệ phương trình tuyến tính
  \item
    B2. Giải ra nghiệm
  \item
    B3. Cho \(1\) ẩn \(\neq 0 \to\) Thu được tổ hợp tuyến tính (VD.
    \(u_1+u_2+u_3 = \sim \to u_1 = \sim\))
  \end{itemize}
\end{itemize}

    \begin{itemize}
\tightlist
\item
  Cho \(S = \{ u_1 , u_2 \ldots , u_k \}, \: k \geq 2, u_i \neq 0\) ,
  \(S\) \textbf{phụ thuộc tuyến tính}
\end{itemize}

\[
\Leftrightarrow \exists u_i, \: 2 \leq i \leq k \: | \: u_i = \alpha _1 u_1+\ldots +\alpha _{i-1}u_{i-1}.
\]

(Hơn nữa, có thể chọn \(i\) sao cho \(\{ u_1 , \ldots , u_{i-1} \}\) độc
lập tuyến tính. (Định lý giúp tìm tập hợp độc lập tuyến tính lớn nhất
trong 1 tập))

    \subsection{Cơ sở và số
chiều}\label{cux1a1-sux1edf-vuxe0-sux1ed1-chiux1ec1u}

    Cho \(V\) là \(\mathbb{K}\)-Không gian vector và \(B \subset V\).

    \subsubsection{Hệ sinh}\label{hux1ec7-sinh}

    \begin{itemize}
\tightlist
\item
  Mọi vector trong \(V\) biểu thị tuyến tính được qua \(B \Rightarrow\)
  Hệ vector \(B\) đgl \textbf{hệ sinh (tập sinh)} của \(V\).
\end{itemize}

    \subsubsection{Cơ sở}\label{cux1a1-sux1edf}

    \begin{itemize}
\item
  \(\begin{cases} B \text{ Hệ sinh của } V \\ B \text{ Độc lập tuyến tính} \end{cases} \Rightarrow\)
  Hệ vector \(B\) là \textbf{Cơ sở} của \(V\).
\item
  Note:

  \begin{itemize}
  \item
    \(\mathbb{K}\)-không gian vector, \(\mathbb{C}\)-không gian vector
    có 1 cơ sở là \(\{ 1 \}\).
  \item
    \(\mathbb{C}\) là \(\mathbb{R}\)-không gian vector.
    \(z=a+bi=a1+bi \Rightarrow \{ 1,i \}\) là 1 hệ sinh.
    \(a+bi=0 \Leftrightarrow a=0=b \Rightarrow \{ 1,i \}\) là độc lập
    tuyến tính.
  \item
    \(\mathbb{R}\) là \(\mathbb{R}\)-không gian vector với cơ sở là
    \(\{ 1 \}\).
  \item
    \(\mathbb{R}\) là \(\mathbb{Q}\)-không gian vector với cở sở là vô
    hạn.
  \end{itemize}
\end{itemize}

    \paragraph{VDMH}\label{vdmh}

\begin{itemize}
\tightlist
\item
  \textbf{VD1}
\end{itemize}

\(\mathbb{K}^n, \varepsilon = \{ e_1, e_2, \ldots , e_k \}\), trong đó
\(e_1=(0,\ldots ,1, \ldots ,0)\)

\(\forall x \in \mathbb{K}^n,x=(x_1,x_2,\ldots,x_n)=x_1e_1 + x_2e_2 + \ldots + x_ne_n\)

\(\Rightarrow \varepsilon\) là 1 hệ sinh của \(\mathbb{K}^n\) (*)

\(x_1e_1 + x_2e_2 + \ldots + x_ne_n = 0 \Leftrightarrow (x_1, \ldots , x_n)=(0,\ldots ,0) \Rightarrow x_1=x_2=\ldots =0\)

\(\Rightarrow \varepsilon\) độc lập tuyến tính (**)

Từ (*)(**) \(\Rightarrow \varepsilon\) là 1 cơ sở của \(\mathbb{K}^n\)
(và là cơ sở chính tắc của \(\mathbb{K}^n\)).

\begin{itemize}
\tightlist
\item
  \textbf{VD2}
\end{itemize}

\(M(m\times n, \mathbb{K}), \varepsilon = \{ E_{ij}\}_{i=1,m}^{j=1,m}\)
trong đó
\(E_{ij}= \begin{pmatrix} & \vdots & \\ \ldots & 1 & \ldots \\ & \vdots & \end{pmatrix}\).
(C/m tương tự) \(\Rightarrow \varepsilon\) là \(1\) cơ sở của
\(M(m\times n, \mathbb{K}).\)

\begin{itemize}
\tightlist
\item
  \textbf{VD3}
\end{itemize}

\(\mathbb{K}[x], \varepsilon = \{1,x,x^2,\ldots ,x^n, \ldots \}.\)

\begin{itemize}
\tightlist
\item
  \textbf{VD4}
\end{itemize}

\(\mathbb{K}_n[x], \varepsilon = \{ 1, x, \ldots , x^n \}.\)

    \subsubsection{Độc lập tuyến tính cực
đại}\label{ux111ux1ed9c-lux1eadp-tuyux1ebfn-tuxednh-cux1ef1c-ux111ux1ea1i}

    Cho \(S = \{ u_1, u_2, \ldots , u_n \} \subset V\)

    \begin{itemize}
\tightlist
\item
  Nếu
  \(\begin{cases} S \text{ Độc lập tuyến tính} \\ S \varsubsetneq S', S' \text{ Phụ thuộc tuyến tính} \end{cases} \Rightarrow S\)
  \textbf{Độc lập tuyến tính cực đại}.
\end{itemize}

    \paragraph{Định lý}\label{ux111ux1ecbnh-luxfd}

    Cho \(S = \{ u_1, u_2, \ldots , u_m \}\)

    \begin{itemize}
\item
  Cho \(S \subset V\), các mệnh đề sau là \(\Leftrightarrow\):

  \begin{itemize}
  \item
    Hệ \(S\) là 1 \textbf{Cơ sở} của \(V\).
  \item
    Hệ \(S\) là Hệ vector \textbf{Độc lập tuyến tính cực đại} của \(V\).
  \item
    \(\forall\) vector của \(V\) đề \textbf{Biểu thị tuyến tính} được 1
    cách \textbf{duy nhất} qua hệ \(S\).
  \end{itemize}
\end{itemize}

    \begin{itemize}
\item
  \(S\) là \textbf{hệ sinh} của \(V\) nếu (1 trong 2 thoả):

  \begin{itemize}
  \item
    \(\exists u_i, 1 \leq i \leq m\) \textbf{Tổ hợp tuyến tính} của các
    vector còn lại trong
    \(S \Rightarrow \bigg(S' = S \setminus  \{ v \}\bigg).\)
  \item
    \(\bigg(V \neq \{0\}\bigg) \Rightarrow V\)
    \(\exists 1 \textbf{ Cơ sở } \subset S.\)
  \end{itemize}
\end{itemize}

    \begin{itemize}
\item
  \textbf{Bổ đề:}

  \begin{itemize}
  \tightlist
  \item
    Không gian vector \(V\) có \(1\) \textbf{cơ sở} gồm \(n\) vector
    \(\Rightarrow \forall\) hệ \textbf{Độc lập tuyến tính} trong \(V\)
    đều chứa không quá \(n\) vector.
  \end{itemize}
\end{itemize}

    \begin{itemize}
\tightlist
\item
  \textbf{Note:}

  \begin{itemize}
  \item
    Cách chứng minh \(A \Rightarrow B\):

    \begin{itemize}
    \item
      B1. Đặt \(A\) là giả thuyết \(\to (1)\)
    \item
      B2. Xác định điều kiện để có được \(B\) (từ tiên đề/định lý/tính
      chất/\(\ldots\)) \(\to (2)\)
    \item
      B3. C/m \((1) \Rightarrow (2) \to\) C/m xong.
    \end{itemize}
  \item
    Cách biểu diễn \textbf{duy nhất}:

    \begin{itemize}
    \item
      B1. Đặt \(\text{Biểu diễn A} = \text{Biểu diễn B}\)
    \item
      B2. C/m
      \(\text{Hệ số của Biểu diễn A} = \text{Hệ số của Biểu diễn B} \to\)
      C/m xong.
    \end{itemize}
  \item
    Cách xây dựng \textbf{cơ sở}:

    \begin{itemize}
    \item
      C1. Từ độc lập tuyến tính \(\to\) Thêm vào \(\to\) Độc lập tuyến
      tính cực đại.
    \item
      C2. Từ hệ sinh \(\to\) Bỏ bớt ra \(\to\) Hệ sinh cực tiểu.
    \end{itemize}
  \end{itemize}
\end{itemize}

    \subsubsection{Số chiều}\label{sux1ed1-chiux1ec1u}

    \[
\dim _{\mathbb{K}} V = n \Leftrightarrow V \text{ có 1 cơ sở gồm } n \text{ vector.}
\]

\begin{itemize}
\item
  \(V \neq \{ 0 \}, V\) không có cơ sở nào gồm hữu hạn vector
  \(\Rightarrow \dim _{\mathbb{K}} V = \infty.\)
\item
  Note: Mọi cơ sở của \(V\) đều có cùng số vector (số chiều là số vector
  trong cơ sở).
\item
  \(\dim \{ 0 \} = 0.\)
\end{itemize}

    \textbf{VD}

\begin{itemize}
\item
  \(\dim _{\mathbb{K}} \mathbb{K}^n = n.\)
\item
  \(\dim _{\mathbb{K}} (M (m \times n, \mathbb{K})) = mn.\)
\item
  \(\dim _{\mathbb{K}} (\mathbb{K} _n [x]) = n + 1.\)
\item
  \(\dim _{\mathbb{K}} (\mathbb{K}[x]) = \infty.\)
\item
  \(\dim _{\mathbb{C}} (\mathbb{C}) = 1,\: \dim _{\mathbb{K}} (\mathbb{C}) = 2.\)
\item
  \(\dim _{\mathbb{K}} (\mathbb{R}) = 1.\)
\end{itemize}

    \paragraph{Định lý}\label{ux111ux1ecbnh-luxfd}

    \begin{itemize}
\tightlist
\item
  \(V \neq \{ 0 \}\) đgl không gian hữu hạn sinh (nghĩa là có 1 tập sinh
  gồm hữu hạn vector) \(\Rightarrow V\) có 1 cơ sở gồm hữu hạn vector.
\end{itemize}

    \paragraph{Tính chất}\label{tuxednh-chux1ea5t}

    \subparagraph{Mệnh đề 1}\label{mux1ec7nh-ux111ux1ec1-1}

    \begin{itemize}
\tightlist
\item
  \(V \neq \{ 0 \}\) (tương tự)
\end{itemize}

    \subparagraph{Mệnh đề 2}\label{mux1ec7nh-ux111ux1ec1-2}

    \(V\) là KG vector \(n\)-chiều,
\(\beta = \{u_1,\ldots , u_n\} \subset V\) (Các mệnh đề sau tương đương)

\begin{itemize}
\item
  \(\beta\) là cở sở của \(V\).
\item
  \(\beta\) là hệ sinh của \(V\).
\item
  \(\beta\) độc lập tuyến tính.
\end{itemize}

    \subsection{Toạ độ}\label{toux1ea1-ux111ux1ed9}

    \begin{itemize}
\item
  Cố định thứ tự của \(B \Rightarrow B = (u_1 , \ldots , u_n )\) là
  \textbf{cơ sở sắp thứ tự}.
\item
  \(B = (u_1 , \ldots , u_n )\) là cơ sở sắp thứ tự của
  \(\mathbb{K}\)-KG vector \(V\)
\end{itemize}

\[\forall v \in V, v = \alpha _1 u_1 + \ldots + \alpha _n u_n.\]

\begin{itemize}
\item
  \((\alpha _1, \ldots , \alpha _n)\) đgl \textbf{toạ độ} của \(v\) đối
  với cơ sở \(B\). K/h:

  \begin{itemize}
  \item
    \((v)_B = (\alpha _1, \ldots , \alpha _n).\)
  \item
    \([v]_B = \begin{pmatrix} \alpha _1 \\ \alpha _2 \\ \vdots \\ \alpha _n \end{pmatrix}\)
  \item
    \(v = \alpha _1 u_1 + \ldots + \alpha _n u_n\).
  \end{itemize}
\end{itemize}

    \textbf{VD: Tìm toạ độ (4.7a)}

\(((2,3,1))_B = (x_1 , x_2, x_3)\)

\((2,3,1) = x_1 (-1,2,4)+x_2 (\sim ) + x_3 (\sim)\)

\(\begin{cases} \sim = 2 \\ \sim  =3 \\ \sim =1 \end{cases} \Rightarrow \begin{cases} x_1 = -\frac{19}{3} \\ x_2 = -11 \\ x_3 = \frac{20}{3} \end{cases} \Rightarrow ((2,3,1))_B = (-\frac{19}{3}, -11, \frac{20}{3}).\)

    \subsubsection{Tính chất}\label{tuxednh-chux1ea5t}

    Cho \(B = (u_1 , \ldots , u_m)\) là cơ sở của
\(V, \: v \in V, \: (v)_B = (\alpha _1, \ldots , \alpha _n) \in \mathbb{K} ^n, \: [0] = 0\)

\begin{itemize}
\item
  \(\forall u, v \in V \quad [u _ v]_B = [u]_B + [v]_B.\)
\item
  \(\forall \alpha \in \mathbb{K} \quad [\alpha u] _B = \alpha [u]_B.\)
\item
  \(\forall \alpha , \beta \in \mathbb{K} \quad [\alpha u + \beta v] _B = \alpha [u]_B + \beta [v]_B.\)
\item
  \([\alpha _1 u_1 + \ldots + \alpha _n u_n]_B = \alpha _1 [u_1]_B + \ldots + \alpha _n [u_n]_B.\)
\end{itemize}

    \subsubsection{Mệnh đề}\label{mux1ec7nh-ux111ux1ec1}

    Cho \(B\) là 1 cơ sở của
\(V, \: S = \{ v_1 , \ldots , v_k \} \subset V.\)

\begin{itemize}
\item
  \(S\) độc lập tuyến tính
  \(\Leftrightarrow \{ [\alpha _1 ]_B,\ldots ,[\alpha _k]_B \}\) độc lập
  tuyến tính.
\item
  \(S\) độc lập tuyến tính
  \(\Leftrightarrow \text{rank} ([v _1 ]_B \ldots [v _k]_B) = k.\)
\end{itemize}

    \subsubsection{Định lý}\label{ux111ux1ecbnh-luxfd}

    Cho \(B\) và \(C\) là 2 cơ sở của \(V\)

\begin{itemize}
\tightlist
\item
  \(\exists\) duy nhất ma trận \(A\) vuông, khả nghịch sao cho
\end{itemize}

\[[v]_B = A [v]_C, \forall v \in V\]

\begin{itemize}
\tightlist
\item
  \(A\) đgl \textbf{ma trận chuyển cơ sở} từ \(B\) sang \(C\),
  \([v]_B = A [v]_C\) là công thức đổi toạ độ
\end{itemize}

\[A = P _{B,C} = ([v_1]_B \ldots [v_n]_B)\]

    \subsection{Không gian con - Hạng của hệ
vector}\label{khuxf4ng-gian-con---hux1ea1ng-cux1ee7a-hux1ec7-vector}

    Cho \(\varnothing \neq E \subset V\)-KG vector

\begin{itemize}
\tightlist
\item
  \(E\) là KG vector cùng với 2 phép toán trên \(V \Rightarrow E\) là
  \textbf{KG con} của \(V\).
\end{itemize}

    \textbf{VD}

\begin{itemize}
\item
  \(V\) là KG vector

  \begin{itemize}
  \item
    \(\{0 \}\) là KG con của \(V\) đgl KG con tầm thường.
  \item
    \(V \subset V.\)
  \end{itemize}
\item
  \(\mathbb{R}^3 \quad P=\{ (x,y,0) \mid x,y \in \mathbb{R} \}\) là 1 KG
  con của \(\mathbb{R}^3\).
\item
  \(\mathbb{K}_n [X] \subset \mathbb{K}[X].\)
\end{itemize}

    \subsubsection{Định lý}\label{ux111ux1ecbnh-luxfd}

    Cho \(\varnothing \neq E \subset V\)-KG vector, \(E\) là KG con của
\(V \Rightarrow\) thoả điều kiện sau:

\begin{itemize}
\item
  \(u+v \in E, \quad \forall u,v \in E.\)
\item
  \(\alpha u \in E, \quad \forall u \in E, \forall \alpha \in \mathbb{K}.\)
\end{itemize}

hoặc

\begin{itemize}
\tightlist
\item
  \((\alpha u + \beta v ) \in E, \quad \forall \alpha, \beta \in \mathbb{K}, \forall u,v \in E.\)
\end{itemize}

    \textbf{VD}

Cho \(\mathbb{Q} = \{ (x,y,z) x - 2y + z = 0 \}\). C/m \(\mathbb{Q}\) là
KG con của \(\mathbb{R}^3.\)

Ta có
\(0-20+0=0 \Rightarrow (0,0,0) \in \mathbb{Q} \Rightarrow \mathbb{Q} \neq \varnothing\)

\(\forall  u= (x,y,z), v=(a,b,c) \Rightarrow x+2y+z=0\) và \(a+2b+c=0\)

Xét \(u+v=(x+a,y+b,z+c)\)

\((x+a)-2(y+b)+(z+c) = (x-2y+z) + (a-2b+c)=0 \Rightarrow u,v \in \mathbb{Q}\)

\(\forall \alpha \in \mathbb{R}, \forall u = (x,y,z) \in \mathbb{Q} \Rightarrow x - 2y + z = 0\)

Xét \(\alpha u = (\alpha x, \alpha y, \alpha z)\)

Vì
\(\alpha x - 2 \alpha y + \alpha z = \alpha (x+2y +z) = 0 \Rightarrow \alpha u \in \mathbb{Q} \Rightarrow \mathbb{Q}\)
KG con của \(\mathbb{R}^3.\)

    \begin{itemize}
\tightlist
\item
  Note: Cho 3 toạ độ nếu đi qua gốc toạ độ \(\Rightarrow\) KG con (hình
  dung hình học).
\end{itemize}

    \subsubsection{Mệnh đề}\label{mux1ec7nh-ux111ux1ec1}

    \begin{itemize}
\item
  Nếu \(E\) là KG con của \(V\)

  \begin{itemize}
  \item
    \(\dim V < \infty \Rightarrow \dim E < \infty \land \dim E \leq \dim V.\)
  \item
    \(\dim E = \dim V \Rightarrow E = V.\)
  \end{itemize}
\item
  \(\{ v_i \} _{i \in I}\) \(v_i\) là KG con của
  \(V,\: \forall i \in I.\)
\item
  \(\bigcup = \bigcap _{i \in I} v_i\) là KG con của \(V\).
\end{itemize}

    \subsubsection{Span}\label{span}

    \begin{itemize}
\item
  Cho \(X\) là tập con của \(\mathbb{K}\)-KG vector \(V\), giao của tất
  cả các KG con của \(V \supset X\) là 1 KG con của \(V \supset X\), đgl
  \textbf{KG con của \(V\) sinh bởi \(X\)}.
\item
  K/h: \(\mathcal{L}(X) = \text{Span}(X) = \text{Sp}(X).\)
\end{itemize}

    \begin{itemize}
\item
  \textbf{Note}

  \begin{itemize}
  \item
    \(\mathcal{L} (\{ 0 \}) = \{ 0 \}.\)
  \item
    \(\mathcal{L} (\varnothing) = \{ 0 \}.\)
  \item
    \(\mathcal{L} (X)\) là KG con nhỏ nhất của \(V \supset X.\)
  \item
    \(\mathcal{L} ( \mathcal{L} (X)) = \mathcal{L}(X).\)
  \item
    Nếu \(E\) là KG con của \(V, \: \mathcal{L}(E) = E.\)
  \end{itemize}
\end{itemize}

    \paragraph{Định lý}\label{ux111ux1ecbnh-luxfd}

    \begin{itemize}
\tightlist
\item
  Cho \(X \subset V\), khi đó
\end{itemize}

\[
\mathcal{L} (X) = \Bigg\{ \sum_{\text{hữu hạn}} \alpha _i u_i \mid \alpha _i \in \mathbb{K}, u_i \in X \Bigg\} = \text{Span}(X).
\]

\begin{itemize}
\tightlist
\item
  Hạng của hệ vector \(X\) là số chiều KG con sinh bởi \(X\)
\end{itemize}

    \subsection{Không gian hàng, cột,
nghiệm}\label{khuxf4ng-gian-huxe0ng-cux1ed9t-nghiux1ec7m}

    Cho \(A = (a_{ij})_{m \times n} , a_{ij} \in \mathbb{K}\)

\begin{itemize}
\tightlist
\item
  \(\text{Row}A = \text{Span} \{ m \text{ hàng của } A \} \subset \mathbb{K}^n.\)
\item
  \(\text{Col}A=\text{Im}A=\text{Span}(A_1, A_2, \ldots ,A_n) \subset \mathbb{K}^m.\)
\item
  \(\text{Nul}A=\text{Ker}A = \{ x \in \mathbb{K}^n \mid Ax=0 \}.\)
\end{itemize}

    \subsubsection{Mệnh đề}\label{mux1ec7nh-ux111ux1ec1}

    Cho \(A\) là ma trận cỡ \(m \times n\)

\[
\dim A (\text{Row}A) = \dim (\text{Col}A) = r(A).
\]

\[
n = \text{rank}A + \dim (\text{Nul}A).
\]

    \begin{itemize}
\tightlist
\item
  Note: Hỏi số chiều \(\to\) thoải mái, hỏi cơ sở \(\to\) tìm hết.
\end{itemize}

    \subsubsection{VDMH}\label{vdmh}

    \textbf{VD1}

Tìm cơ sở (bài 4.23a)

Ta có
\(A = \begin{pmatrix} \sim \sim \sim \\ \sim \sim \sim \\ \sim \sim \sim \end{pmatrix}\)

\(\text{Col}A\) sinh bởi \(A_1, A_2, A_3, A_4\)

Xét \(x_1 A_1 + x_2 A_2 + x_3 A_3 + x_4 A_4 = 0\)

\(\Rightarrow \begin{cases} -2 x_1 + 4x_2 + \sim - 4x_4  = 0 \\ \sim \sim \sim = 0 \\ \sim \sim \sim = 0 \end{cases} \to\)
giải ra dạng bậc thang \(\Rightarrow\) Phụ thuộc tuyến tính.

\(\Rightarrow \{ A_1 , A_2 \}\) là cơ sở của \(\text{Col}A\)

\(\Rightarrow\) row có cơ sở là \(\{ (1,-2,1,2), (0,-2,-5,-3) \}\) (Hàng
chứa PTC trong dạng bậc thang)

\(x \in \text{Nul}A \Leftrightarrow Ax = 0\)

\(A \to\) dạng rút gọn
\(\begin{pmatrix} 1 & 0 & 6 & 5 \\ 0 & 1 & \frac{5}{2} & \frac{3}{2} \\ 0 & 0 & 0 & 0 \end{pmatrix} \Rightarrow\)
Có nghiệm
\(X = (\sim, \sim, \sim, \sim) \forall x_3, x_4 \in \mathbb{R}\)

Đặt
\(x_3 = 1, x_4 = \sim \Rightarrow X = \{ (\sim, \sim, \sim, \sim),(\sim, \sim, \sim, \sim) \}\)
là cơ sở của \(\text{Nul}A\) (Không gian nghiệm).

    \begin{itemize}
\tightlist
\item
  \textbf{Note}: Để kiểm tra không gian con hay không có thể dùng cách
  này trừ dạng đa thức với ma trận.
\end{itemize}

    \textbf{VD2} 4.4.12 (Giáo trình)

\begin{enumerate}
\def\labelenumi{(\roman{enumi})}
\tightlist
\item
  C/m \(H\) là kgian con của \(\mathbb{R}^3\)
\end{enumerate}

\(\forall u \in H, \exists a,b,c \in \mathbb{R}\) sao cho

\(u = \begin{pmatrix} a - 2b + c \\ a+b-c \\ -3b + 2c \end{pmatrix} = \begin{pmatrix} 1 & -2 & 1 \\ \sim & \sim & \sim \\ \sim & \sim & \sim\end{pmatrix} \begin{pmatrix} a \\ b \\ c \end{pmatrix} \in \text{Col}A \Rightarrow H \subset \text{Col}A\)

\(H \subset \text{Col}A, \forall u \in \text{Col}A \to Ax = u\) có
nghiệm

\(\to \exists a, b,c \in \mathbb{R}\) sao cho
\(u=A \begin{pmatrix}a\\b\\c \end{pmatrix} = \begin{pmatrix} \sim \\ \sim \\ \sim \end{pmatrix} \in H \Rightarrow\)
có \(\text{Col}A \subset H \Rightarrow H = \text{Col}A.\)

\begin{enumerate}
\def\labelenumi{(\roman{enumi})}
\setcounter{enumi}{1}
\tightlist
\item
\end{enumerate}

\(\forall u = (x,y,z) \in \mathbb{K}\)

\(\begin{cases} x - 2y + 3z = 0 \\ 2x - y+z=0 \end{cases} \Rightarrow \begin{pmatrix} 1 & -2 & 3 \\ \sim & \sim & \sim \\ \sim & \sim & \sim \end{pmatrix} \begin{pmatrix} x \\ y \\ z \end{pmatrix} = B u^T \Rightarrow u \in \text{Nul}B\)

    \subsection{\texorpdfstring{Tổng \(\&\) Tổng trực
tiếp}{Tổng \textbackslash\& Tổng trực tiếp}}\label{tux1ed5ng-tux1ed5ng-trux1ef1c-tiux1ebfp}

    \begin{itemize}
\item
  \(V_1 + V_2 + \ldots + V_m\) là \textbf{KG tổng} của
  \(V_1, V_2 , \ldots , V_m.\)
\item
  Tổng \(V_1 + \ldots + V_m\) đgl \textbf{tổng trực tiếp} nếu
  \(\forall v \in V_1 + \ldots + V_m\) đều có duy nhất 1 cách phân tích
  \(V = u_1 + u_2 + \ldots + u_{n-1}, \forall u_i \in V_i\), k/h:
  \(V_1 \oplus V_2 \oplus \ldots \oplus V_m.\)
\end{itemize}

    \subsubsection{Mệnh đề}\label{mux1ec7nh-ux111ux1ec1}

    Cho \(V_1, V_2, \ldots , V_m\) là các KG con của \(\mathbb{K}\)-KG
vector \(V\)

\[
V_1 + V_2 + \ldots + V_m = \{ u_1 + u_2 + \ldots + u_m \mid u_i \in V_i \}
\]

là 1 KG con của \(V\).

    \subsubsection{Định lý}\label{ux111ux1ecbnh-luxfd}

    Tổng \(V_1 + \ldots + V_m\) là tổng trực tiếp nếu 1 trong 2 điều kiện
tương đương sau thoả:

\begin{itemize}
\item
  \(V _j \bigcap (\sum_{i \neq j} V_i) = \{ 0 \} \quad 1 \leq j \leq m.\)
\item
  \(V _j \bigcap (\sum_{i > j} V_i) = \{ 0 \} \quad 1 \leq j \leq m-1.\)
\end{itemize}

    \subsubsection{Số chiều không gian
tổng}\label{sux1ed1-chiux1ec1u-khuxf4ng-gian-tux1ed5ng}

    \paragraph{Định lý}\label{ux111ux1ecbnh-luxfd}

    Cho \(V_1, V_2\) là 2 KG con của \(V\)

\[
\dim (V_1 + V_2) = \dim V_1 + \dim V_2 - \dim (V_1 \cap V_2).
\]

    \paragraph{Hệ quả}\label{hux1ec7-quux1ea3}

    \begin{itemize}
\item
  \(V_1 = \text{Sp}(S_1), \: V_2 = \text{Sp}(S_2) \Rightarrow V_1 + V_2 = \text{Sp}(S_1 \cup S_2).\)
\item
  \(V_1 \cap V_2 = \varnothing \Rightarrow \dim (V_1 + V_2 ) = \dim V_1 + \dim V_2.\)
\end{itemize}

    \section{Ánh Xạ Tuyến Tính}\label{uxe1nh-xux1ea1-tuyux1ebfn-tuxednh}

    \begin{itemize}
\item
  Cho \(U\) và \(V\) là các \(\mathbb{K}\)-KGVT, Ánh xạ \(f:U \to V\)
  đgl \textbf{AXTT} nếu:

  \begin{itemize}
  \item
    \(f(u+v)=f(u)+f(v), \forall u,v \in V.\)
  \item
    \(f(\alpha u) = \alpha f(u), \forall \alpha \in \mathbb{K}, \forall u \in U.\)
  \end{itemize}
\end{itemize}

    \begin{itemize}
\tightlist
\item
  \textbf{Định lý:} Cho \(U\) và \(V\) là các \(\mathbb{K}\)-KGVT, Ánh
  xạ \(f:U \to V\) đgl \textbf{AXTT} nếu
\end{itemize}

\[
f(\alpha u + \beta v) = \alpha f(u) + \beta f(v) , \forall \alpha , \beta \in \mathbb{K}, \forall u,v \in U
\]

    \begin{itemize}
\item
  \textbf{Tính chất}

  \begin{itemize}
  \item
    \(f(0_V) = 0_V.\)
  \item
    \(f(-u) = -f(u).\)
  \item
    \(f(\alpha _1 u_1 + \ldots + \alpha _k u_k) = \alpha _1 f(u_1) + \ldots + \alpha _k f(u_k).\)
  \end{itemize}
\end{itemize}

    \begin{itemize}
\tightlist
\item
  \textbf{Mệnh đề:} \(U\), \(V\), \(W\) là \(\mathbb{K}\)-KGTV, ta có:
\end{itemize}

\[
\begin{aligned}
f &: U \longrightarrow V, && u \longmapsto f(u), \\
g &: V \longrightarrow W, && v \longmapsto g(v), \\
g \circ f &: U \longrightarrow W, && u \longmapsto g(f(u)).
\end{aligned}
\]

    \begin{itemize}
\tightlist
\item
  \textbf{Định lý về Sự xác định của AXTT:} Cho \(U\), \(V\) là các
  \(\mathbb{K}\)-KGTV, \((u_1, \ldots , u_n)\) là 1 cơ sở của
  \(U, v_1 , \ldots , v_n \in V\), khi đó \(\exists !\) AXTT
  \(f : U \to V\) thoả
\end{itemize}

\[
f(u_i) = v_i, 1 \leq i \leq n.
\]

    \subsection{Đơn cấu, Toàn cấu, Đẳng
cấu}\label{ux111ux1a1n-cux1ea5u-touxe0n-cux1ea5u-ux111ux1eb3ng-cux1ea5u}

    \begin{itemize}
\item
  \textbf{Định nghĩa:} \(u\), \(v\) là các \(\mathbb{K}\)-KGTV
  \(f:U \to V\), gọi là \textbf{đồng cấu}, ta có, \(f\):

  \begin{itemize}
  \item
    Đơn ánh \(\to\) \textbf{Đơn cấu}.
  \item
    Toàn ánh \(\to\) \textbf{Toàn cấu}.
  \item
    Song ánh \(\to\) \textbf{Đẳng cấu}. \(U\) đẳng cấu với \(V\) k/h:
    \(U \cong V.\)
  \end{itemize}
\end{itemize}

    \begin{itemize}
\item
  \textbf{Nhận xét:}

  \begin{itemize}
  \tightlist
  \item
    \(U \cong U.\)
  \item
    \(U \cong V \to V \cong U.\)
  \item
    \(U \cong V, V \cong W \to U \cong W.\)
  \end{itemize}
\end{itemize}

    \textbf{Tính chất Đồng cấu}

    \begin{itemize}
\item
  \textbf{Định lý:} \(f : U \to V\) là đồng cấu,
  \(s = \{ u_1 , \ldots , u_k \}\) Phụ thuộc tuyến tính trong
  \(U \Rightarrow f(s) = \{ f(u_1), \ldots , f(u_k) \}\) Phụ thuột tuyến
  tính trong \(V.\)

  \begin{itemize}
  \item
    \(f\) đơn cấu \(\Leftrightarrow\) (\(s\) độc lập tuyến tính
    \(\Rightarrow\) f(s) Độc lập tuyến tính, \(\forall s\)).
  \item
    \(f\) toàn cấu \(\Leftrightarrow\) (\(s\) hệ sinh của
    \(U \Rightarrow f(s)\) là hệ sinh của \(V\)).
  \item
    \(f\) đẳng cấu \(\Leftrightarrow\) (\(s\) là cơ sở của
    \(U \Rightarrow f(s)\) là cơ sở của \(V\)).
  \end{itemize}
\end{itemize}

    \subsection{Nhân và Ảnh (Kernel và
Image)}\label{nhuxe2n-vuxe0-ux1ea3nh-kernel-vuxe0-image}

    \begin{itemize}
\item
  \textbf{Định nghĩa:} Cho \(f: U \to V\) là AXTT, E là KG con \(U\), F
  là KG con \(V\). \(f(E) = \{ f(u) \mid u \in E \}\),
  \(f^{-1} (F) = \{ u \in U \mid f(u) \in F \} \subset U\)

  \begin{itemize}
  \item
    \textbf{Ảnh} của \(f\) là \(f(U) = \text{Im}f.\)
  \item
    \textbf{Nhân} của \(f\) là \(f^{-1} (\{ 0_V \}) = \text{Ker}f.\)
  \end{itemize}
\end{itemize}

    \begin{itemize}
\item
  \textbf{Định lý:} Cho \(f: U \to V\) là AXTT, E là KG con \(U\), F là
  KG con \(V\)

  \begin{itemize}
  \item
    \(f(E)\) là KG con \(V.\)
  \item
    \(f^{-1} (F)\) là KG con \(U.\)
  \end{itemize}
\end{itemize}

    \begin{itemize}
\item
  \textbf{Hệ quả:} Cho \(f: U \to V\) là AXTT

  \begin{itemize}
  \item
    \(\text{Im}f\) là 1 KG con của \(V.\)
  \item
    \(\text{Ker}f\) là 1 KG con của \(U.\) (\(f(U)=0\))
  \end{itemize}
\end{itemize}

    \begin{itemize}
\item
  \textbf{Định lý:} Cho \(f: U \to V\) là đẳng cấu

  \begin{itemize}
  \item
    \(f\) là đơn cấu \(\Leftrightarrow \text{Ker}f = \{ 0_U \}.\)
  \item
    \(f\) là toàn cấu \(\Leftrightarrow \text{Im}f = V.\)
  \end{itemize}
\end{itemize}

    \begin{itemize}
\item
  \textbf{Định lý:} Cho \(f: U \to V\) là AXTT

  \begin{itemize}
  \item
    \(f\) đơn cấu \(\Leftrightarrow r(f) = \dim U.\)
  \item
    \(f\) toàn cấu \(\Leftrightarrow r(f) = \dim V.\)
  \item
    \(f\) đẳng cấu \(\Leftrightarrow r(f) = \dim U = \dim V.\)
  \end{itemize}
\end{itemize}

    \begin{itemize}
\tightlist
\item
  \textbf{Định lý về hạng của AXTT:} Cho \(f: U \to V\) là AXTT, khi đó
\end{itemize}

\[
\dim U = \dim (\text{Ker}f ) + \dim (\text{Im}f) = \dim (\text{Ker}f) + \text{rank}(f)
\]

    \subsection{Ma trận của AXTT}\label{ma-trux1eadn-cux1ee7a-axtt}

    \begin{itemize}
\item
  \textbf{Định lý:} Cho \(U\), \(V\) là các KGVT, \(\dim U=n\),
  \(\dim V = m\), \(\mathcal{B}\) là cơ sở của \(U\), \(\mathcal{C}\) là
  cơ sở của \(V\).

  \begin{itemize}
  \item
    \(\forall\) AXTT \(f:U \to V, \exists !\) ma trận \(A_{m \times n}\)
    sao cho
    \([ f(u) ]_{\mathcal{C}} = A [ u ]_{\mathcal{B}}, \forall u \in U.\)
  \item
    \(\forall A _{m \times n}\), \(\exists !\) AXTT \(f : U \to V\) thoả
    ở trên.
  \end{itemize}
\end{itemize}

    \begin{itemize}
\tightlist
\item
  Note: \(f: \mathbb{K}^n \to \mathbb{K}^m, x \mapsto Ax\).
  \(\text{Im}f = \text{Col}A\), \(\text{Ker}f = \text{Nul}A.\)
\end{itemize}

    \begin{itemize}
\tightlist
\item
  \textbf{Định nghĩa:} giáo trình
\end{itemize}

    \begin{itemize}
\tightlist
\item
  \textbf{Mệnh đề:} giáo trình
\end{itemize}

\[
[f]_{\mathcal{B} ',\mathcal{C} '} = Q^{-1} [f]_{\mathcal{B} ,\mathcal{C} } P
\]

    \begin{itemize}
\tightlist
\item
  \textbf{Định lý:} Cho \(A\) ma trận đẳng cấu \(f: U \to V\)
\end{itemize}

\[
\text{Im}f \cong \text{Col}A, \text{Ker}f \cong \text{Nul}A \Leftrightarrow \text{rank}f = \text{rank}A
\]

\begin{itemize}
\tightlist
\item
  \textbf{Hệ quả:} \(f\) Đẳng cấu \(\Leftrightarrow A\) Khả nghịch.
\end{itemize}

    \subsection{Không gian các đồng
cấu}\label{khuxf4ng-gian-cuxe1c-ux111ux1ed3ng-cux1ea5u}

    \begin{itemize}
\item
  \textbf{Định nghĩa:} Cho \(U\), \(V\) là các \(\mathbb{K}\)-KGVT,
  \textbf{Không gian các đồng cấu} k/h:
  \(\text{Hom}(U,V) = \text{Hom}_{\mathbb{K}}(U,V) \neq \varnothing\),
  \(\text{Hom}(U,V)\) là KGVT với 2 phép toán:

  \begin{itemize}
  \item
    Phép cộng: \(\forall f, g \in \text{Hom}(U, V)\)

    \begin{itemize}
    \item
      \(f + g : U \to V, u \mapsto f(u) + g(u)\)
    \item
      \((f + g)(u) = f(u) +g(u).\)
    \end{itemize}
  \item
    Phép nhân:
    \(\forall u \in \mathbb{K}, \forall f \in \text{Hom}(U,V)\)

    \begin{itemize}
    \item
      \(\alpha f : U \to V, u \mapsto \alpha f(u)\)
    \item
      \((\alpha f)(u) = \alpha f(u)\)
    \end{itemize}
  \end{itemize}
\end{itemize}

    \subsubsection{Cấu trúc không gian đồng
cấu}\label{cux1ea5u-truxfac-khuxf4ng-gian-ux111ux1ed3ng-cux1ea5u}

    \begin{itemize}
\item
  \textbf{Định lý:} \(\dim U = n\), \(\dim V =m\)

  \begin{itemize}
  \item
    \(\text{Hom}(U,V) \cong M (m \times n, \mathbb{K}).\)
  \item
    \(\text{End}(V) \cong M (n \times n, \mathbb{K})\).
    \(\text{GL}(U) \cong \text{GL}_n(\mathbb{K}).\) (general linear
    group)
  \end{itemize}
\end{itemize}

    \begin{itemize}
\tightlist
\item
  \textbf{Mệnh đề:} \(\begin{aligned}
  f &: U \longrightarrow V, && u \longmapsto f(u), \\
  g &: V \longrightarrow W, && v \longmapsto g(v), \\
  g \circ f &: U \longrightarrow W, && u \longmapsto g(f(u)),
  \end{aligned}\), Gọi \(\mathcal{B}\) là cơ sở của \(U\),
  \(\mathcal{C}\) là cơ sở của \(V\), \(\mathcal{D}\) là cơ sở của
  \(W\).
\end{itemize}

\[
[g \circ f]_{\mathcal{B},\mathcal{D}} = [g]_{\mathcal{C},\mathcal{D}} [f]_{\mathcal{B} ,\mathcal{C} }
\]

    \section{Chéo hoá ma trận}\label{chuxe9o-houxe1-ma-trux1eadn}

    \subsection{\texorpdfstring{Giá trị riêng \(\&\) Vector
riêng}{Giá trị riêng \textbackslash\& Vector riêng}}\label{giuxe1-trux1ecb-riuxeang-vector-riuxeang}

    \begin{itemize}
\item
  \textbf{Định nghĩa:} Cho \(A\) vuông cấp \(n\), \(\exists\) vector
  \(u \in \mathbb{K}^n \setminus \{ 0 \}\) \[A u = \lambda u\]

  \begin{itemize}
  \item
    Giá trị \(\lambda\) đgl \textbf{giá trị riêng} của \(A\).
  \item
    Vector \(u\) đgl \textbf{vector riêng} của \(A\) ứng với
    \(\lambda\).

    \begin{itemize}
    \item
      \(A (\alpha u) = \alpha (Au) = A (\alpha u), \forall \alpha \in \mathbb{K} \setminus \{ 0 \}\).
    \item
      \(A(u+v) = Au + Av = Au + \lambda v = \lambda (u+v), \forall v \in \mathbb{K} \setminus \{ 0 \}\).
    \item
      \(Au = \lambda u, Au = \mu u \Rightarrow \lambda u = \mu u \Leftrightarrow (\lambda - \mu) u = 0\).
    \end{itemize}
  \end{itemize}
\end{itemize}

    \begin{itemize}
\item
  \textbf{Thuật toán tìm giá trị riêng, vector riêng}

  \begin{itemize}
  \item
    B1.
    \(P_A(x) = \det (A - x I_n) = \begin{vmatrix} \ldots \ldots \ldots \\ \ldots \ldots \ldots \\ \ldots \ldots \ldots \end{vmatrix} = 0\).
  \item
    B2. Giải \(P_A(x) \to\) nghiệm \(x\), với từng nghiệm \(x\) tương
    ứng với từng \(\lambda\)
  \item
    B3. Thay \(\lambda\) vào \((A - \lambda I_n)=0 \to\) giải tìm nghiệm
    của \(X\).
  \item
    B4. Lặp lại đến khi ko còn \(\lambda\) nào \(\to\) kết luận \(\to\)
    với mỗi \(\lambda\) là giá trị riêng, \(X\) là các vector riêng của
    \(A\) ứng với \(\lambda \to\) kết thúc.
  \end{itemize}
\end{itemize}

    \begin{itemize}
\tightlist
\item
  \textbf{Định lý:} \(A\) có \(k\) giá trị riêng khác nhau
  (\(k \leq n\)) \(\Rightarrow k\) vector riêng tương ứng \emph{độc lập
  tuyến tính}.
\end{itemize}

    \begin{itemize}
\item
  \textbf{Định nghĩa:} Cho \(A\) vuông cấp \(n\), \(\exists\) ma trận
  \(P\) vuông cấp \(n\) và ma trận chéo \(D\) cấp \(n\) sao cho
  \[A = PDP^{-1} \Leftrightarrow AP =PD \Leftrightarrow P^{-1} AP = D\]

  \begin{itemize}
  \item
    \(A\) đgl \textbf{ma trận chéo hoá được}.
  \item
    \(P\) đgl \textbf{ma trận chéo hoá} \(A\).
  \item
    \(D\) đgl \textbf{dạng chéo} của \(A\).
  \end{itemize}
\end{itemize}

    \begin{itemize}
\tightlist
\item
  \textbf{Định lý:} \(A_{m \times n}\) chéo hoá được
  \(\Leftrightarrow A\) có \(n\) vector riêng \emph{độc lập tuyến tính}.
\end{itemize}

    \subsection{Ma trận chéo hoá
được}\label{ma-trux1eadn-chuxe9o-houxe1-ux111ux1b0ux1ee3c}

    \begin{itemize}
\tightlist
\item
  \(E_{\lambda}\) đgl \textbf{không gian riêng} của \(A\) ứng với
  \(\lambda\)
\end{itemize}

\[E_{\lambda} = \{ \lambda \in \mathbb{K}^n : (A - \lambda I) X = 0 \} = \text{Nul}(A-\lambda I)\]

    \begin{itemize}
\tightlist
\item
  \textbf{Mệnh đề:} \(A\) có \(k\) giá trị riêng khác nhau
  \(\lambda _1, \lambda _2, \ldots ,\lambda _k \Rightarrow E_{\lambda _1} + E_{\lambda _2} + \ldots + E_{\lambda _k}\)
  là \textbf{tổng trực tiếp}.
\end{itemize}

    \begin{itemize}
\item
  \textbf{Định lý:} \(A\) có \(k\) giá trị riêng khác nhau
  \(\lambda _1, \lambda _2, \ldots ,\lambda _k \Rightarrow A\)
  \textbf{chéo hoá được} \(\Leftrightarrow\)

  \begin{itemize}
  \item
    \(E_{\lambda _1} \bigoplus E_{\lambda _2} \bigoplus \ldots \bigoplus E_{\lambda _k} = \mathbb{K}^n\)
  \item
    \(\sum _{i=1}^k \text{dim}(E_{\lambda _i})=n\)
  \item
    \(\sum _{i=1}^k \text{dim}(\text{Nul}(A - \lambda _i I )) = n\)
  \end{itemize}
\end{itemize}

    \begin{itemize}
\item
  \textbf{Thuật toán xác định chéo hoá ma trận}

  \begin{itemize}
  \item
    B1. (Thuật toán tìm giá trị riêng, vector riêng)
    \(\to \lambda _1, \lambda _2, \ldots ,\lambda _k \to\) vector
    \(u_1, u_2, \ldots u_k\) tương ứng với từng
    \(\lambda _1, \lambda _2, \ldots ,\lambda _k \to\) \textbf{số chiều}
    của cơ sở không gian riêng \(E_i\) (Bắt đầu với \(E_0\)) tương ứng
    với từng vector \(u_1, u_2, \ldots u_k\)
  \item
    B2. (Check chéo hoá)

    \begin{itemize}
    \item
      \(\sum _{i=1}^k \text{dim}(E_{\lambda _i})=n \to A\) Chéo hoá được
      \(\to\) B3
    \item
      \(\sum _{i=1}^k \text{dim}(E_{\lambda _i}) \neq n \to A\) Không
      chéo hoá được \(\to\) Kết thúc
    \end{itemize}
  \item
    B3. (Kết luận ma trận \(P\) và \(D\))

    \begin{itemize}
    \tightlist
    \item
      Từ vector \(u_1, u_2, \ldots u_k \to\) hợp các vector lại thành 1
      ma trận (theo cột) \(\to\) \(P\).
    \item
      Tương ứng với giá trị \(\lambda _i\) của \(u_i\) trong cột trong
      \(P \to D = \begin{pmatrix} \lambda _1 & 0 & 0 \\ 0 & \lambda _2 & 0 \\ 0 & 0 & \ddots \end{pmatrix}\).
    \end{itemize}
  \end{itemize}
\end{itemize}

    \subsection{Tự đồng cấu}\label{tux1ef1-ux111ux1ed3ng-cux1ea5u}

    \subsubsection{Giá trị riêng và vector
riêng}\label{giuxe1-trux1ecb-riuxeang-vuxe0-vector-riuxeang}

    \begin{itemize}
\tightlist
\item
  \textbf{Định nghĩa:} \(f:V \to V\) là tự đồng cấu,
  \(\exists v \in V \setminus \{ 0 \}\) \[f(v) = \lambda v\]

  \begin{itemize}
  \tightlist
  \item
    \(\lambda\) đgl \textbf{giá trị riêng} của \(f\).
  \item
    \(v\) đgl \textbf{vector riêng} của \(f\) ứng với \(\lambda\).
  \end{itemize}
\end{itemize}

    \begin{itemize}
\item
  \textbf{Mệnh đề:} \(\mathcal{B}\) là cơ sở của \(\mathbb{K}\)-KGVT
  \(V\), \(A=[f]_{\mathcal{B}}\)

  \begin{itemize}
  \item
    \(\lambda\) là giá trị riêng của \(f \Leftrightarrow \lambda\) là
    giá trị riêng của \(A\).
  \item
    \(v\) là vector riêng của \(f\) ứng với
    \(\lambda \Leftrightarrow [v]_{\mathcal{B}}\) là vector riêng của
    \(A\) ứng với \(\lambda\).
  \end{itemize}
\end{itemize}

    \begin{itemize}
\tightlist
\item
  \textbf{Note:} Để tìm vector riêng của \(f\) ta tìm
  \(X = (\ldots , \ldots , \ldots)\) như thông thường rồi lấy tổng của
  tích với từng \(v_i\) tương ứng
  (\(v = \ldots v_1 + \ldots v_2 + \ldots v_3 + \ldots\))
\end{itemize}

    \subsubsection{Chéo hoá}\label{chuxe9o-houxe1}

    \begin{itemize}
\tightlist
\item
  \textbf{Định nghĩa:} \(f:V \to V\) là tự đồng cấu, \(\exists\) cơ sở
  \(\mathcal{C}\) của \(V\) sao cho \([f]_{\mathcal{C}}\) là ma trận
  chéo \(\Rightarrow f\) đgl \textbf{chéo hoá được}.
\end{itemize}

    \begin{itemize}
\item
  \textbf{Note:}

  \begin{itemize}
  \tightlist
  \item
    \(\mathcal{B} = (u_1 , u_2, u_3) \to\) Thuật toán tìm giá trị riêng,
    vector riêng
    \(\to X \to p_1 = (.. , .. ,..), p_2 =..., p_3 = ... \to\) Cơ sở
    \(\mathcal{C} = (.. u_1 + .. u_2 + .. u_3, .. u_1 + .. u_2 + .. u_3, \ldots) \to [f]_{\mathcal{C}} =\)
    ma trận giống ma trận \(D\).
  \end{itemize}
\end{itemize}

    \section{Không gian vector Euclide}\label{khuxf4ng-gian-vector-euclide}

    \begin{itemize}
\item
  \(\mathbb{E}\) là \(\mathbb{R}\)-KGVT, tích vô hướng trên
  \(\mathbb{E}\) là 1 ánh xạ
  \(f: \mathbb{E} \times \mathbb{E} \to \mathbb{R}\) thoả:

  \begin{itemize}
  \item
    \textbf{Song tuyến tính}:

    \begin{itemize}
    \item
      \(f(\alpha _1 u_1 + \alpha _2 u_2, v) = \alpha _1 f(u_1, v) + \alpha _2 f (u_2, v)\)
    \item
      \(f(u, \beta_1 v_1 + \beta_2 v_2) = \beta_1 f(u, v_1) + \beta_2 f(u, v_2)\)
    \end{itemize}
  \item
    \textbf{Đối xứng}:

    \begin{itemize}
    \tightlist
    \item
      \(f(u, v) = f(v, u) \quad\forall u,v \in \mathbb{E}\)
    \end{itemize}
  \item
    \textbf{Xác định dương}:

    \begin{itemize}
    \tightlist
    \item
      \(f(u, u) \geq 0\) và
      \(f(u, u) = 0 \Leftrightarrow u = 0 \quad\forall u \in \mathbb{E}\)
    \end{itemize}
  \end{itemize}
\end{itemize}

    \begin{itemize}
\tightlist
\item
  \(f: \mathbb{E} \times \mathbb{E} \to \mathbb{R}\) là 1 tích vô hướng,
  \(\langle u, v \rangle = f(u,v)\) là tích vô hướng của \(u\) và \(v\).
\end{itemize}

    \begin{itemize}
\tightlist
\item
  \textbf{KGVT Euclid} \(\mathbb{E}\) là KGVT thực có tích VH.
\end{itemize}

    \begin{itemize}
\tightlist
\item
  Cho \(\mathbb{E}\) là KGVT Euclid \(\langle, \rangle\),
\end{itemize}

\[
\| u \| = \sqrt{\langle u, u \rangle}, \forall u \in \mathbb{E}
\]

\(\| u \| \geq 0, \forall u \in \mathbb{E}, \| u \| =0 \Leftrightarrow u=0\)

    \begin{itemize}
\tightlist
\item
  \textbf{Mệnh đề}: (Bất đẳng thức Cauchy-Schwarz) trong KGVT Euclid
  \(\mathbb{E}\)
\end{itemize}

\[
| \langle u, v \rangle | \leq \| u \| \| v \| , \forall u,v \in \mathbb{E}
\]

    \begin{itemize}
\tightlist
\item
  \textbf{Định nghĩa}
\end{itemize}

\[
\cos \theta = \dfrac{\langle u, v \rangle}{\|u\|\|v\|}, 0 \leq \theta \leq \pi, -1 \leq \dfrac{\langle u, v \rangle}{\|u\|\|v\|} \leq 1
\]


    % Add a bibliography block to the postdoc
    
    
    
\end{document}
