\documentclass[11pt]{article}

    \usepackage[breakable]{tcolorbox}
    \usepackage{parskip} % Stop auto-indenting (to mimic markdown behaviour)
    

    % Basic figure setup, for now with no caption control since it's done
    % automatically by Pandoc (which extracts ![](path) syntax from Markdown).
    \usepackage{graphicx}
    % Keep aspect ratio if custom image width or height is specified
    \setkeys{Gin}{keepaspectratio}
    % Maintain compatibility with old templates. Remove in nbconvert 6.0
    \let\Oldincludegraphics\includegraphics
    % Ensure that by default, figures have no caption (until we provide a
    % proper Figure object with a Caption API and a way to capture that
    % in the conversion process - todo).
    \usepackage{caption}
    \DeclareCaptionFormat{nocaption}{}
    \captionsetup{format=nocaption,aboveskip=0pt,belowskip=0pt}

    \usepackage{float}
    \floatplacement{figure}{H} % forces figures to be placed at the correct location
    \usepackage{xcolor} % Allow colors to be defined
    \usepackage{enumerate} % Needed for markdown enumerations to work
    \usepackage{geometry} % Used to adjust the document margins
    \usepackage{amsmath} % Equations
    \usepackage{amssymb} % Equations
    \usepackage{textcomp} % defines textquotesingle
    % Hack from http://tex.stackexchange.com/a/47451/13684:
    \AtBeginDocument{%
        \def\PYZsq{\textquotesingle}% Upright quotes in Pygmentized code
    }
    \usepackage{upquote} % Upright quotes for verbatim code
    \usepackage{eurosym} % defines \euro

    \usepackage{iftex}
    \ifPDFTeX
        \usepackage[T1]{fontenc}
        \IfFileExists{alphabeta.sty}{
              \usepackage{alphabeta}
          }{
              \usepackage[mathletters]{ucs}
              \usepackage[utf8x]{inputenc}
          }
    \else
        \usepackage{fontspec}
        \usepackage{unicode-math}
    \fi

    \usepackage{fancyvrb} % verbatim replacement that allows latex
    \usepackage{grffile} % extends the file name processing of package graphics
                         % to support a larger range
    \makeatletter % fix for old versions of grffile with XeLaTeX
    \@ifpackagelater{grffile}{2019/11/01}
    {
      % Do nothing on new versions
    }
    {
      \def\Gread@@xetex#1{%
        \IfFileExists{"\Gin@base".bb}%
        {\Gread@eps{\Gin@base.bb}}%
        {\Gread@@xetex@aux#1}%
      }
    }
    \makeatother
    \usepackage[Export]{adjustbox} % Used to constrain images to a maximum size
    \adjustboxset{max size={0.9\linewidth}{0.9\paperheight}}

    % The hyperref package gives us a pdf with properly built
    % internal navigation ('pdf bookmarks' for the table of contents,
    % internal cross-reference links, web links for URLs, etc.)
    \usepackage{hyperref}
    % The default LaTeX title has an obnoxious amount of whitespace. By default,
    % titling removes some of it. It also provides customization options.
    \usepackage{titling}
    \usepackage{longtable} % longtable support required by pandoc >1.10
    \usepackage{booktabs}  % table support for pandoc > 1.12.2
    \usepackage{array}     % table support for pandoc >= 2.11.3
    \usepackage{calc}      % table minipage width calculation for pandoc >= 2.11.1
    \usepackage[inline]{enumitem} % IRkernel/repr support (it uses the enumerate* environment)
    \usepackage[normalem]{ulem} % ulem is needed to support strikethroughs (\sout)
                                % normalem makes italics be italics, not underlines
    \usepackage{soul}      % strikethrough (\st) support for pandoc >= 3.0.0
    \usepackage{mathrsfs}
    

    
    % Colors for the hyperref package
    \definecolor{urlcolor}{rgb}{0,.145,.698}
    \definecolor{linkcolor}{rgb}{.71,0.21,0.01}
    \definecolor{citecolor}{rgb}{.12,.54,.11}

    % ANSI colors
    \definecolor{ansi-black}{HTML}{3E424D}
    \definecolor{ansi-black-intense}{HTML}{282C36}
    \definecolor{ansi-red}{HTML}{E75C58}
    \definecolor{ansi-red-intense}{HTML}{B22B31}
    \definecolor{ansi-green}{HTML}{00A250}
    \definecolor{ansi-green-intense}{HTML}{007427}
    \definecolor{ansi-yellow}{HTML}{DDB62B}
    \definecolor{ansi-yellow-intense}{HTML}{B27D12}
    \definecolor{ansi-blue}{HTML}{208FFB}
    \definecolor{ansi-blue-intense}{HTML}{0065CA}
    \definecolor{ansi-magenta}{HTML}{D160C4}
    \definecolor{ansi-magenta-intense}{HTML}{A03196}
    \definecolor{ansi-cyan}{HTML}{60C6C8}
    \definecolor{ansi-cyan-intense}{HTML}{258F8F}
    \definecolor{ansi-white}{HTML}{C5C1B4}
    \definecolor{ansi-white-intense}{HTML}{A1A6B2}
    \definecolor{ansi-default-inverse-fg}{HTML}{FFFFFF}
    \definecolor{ansi-default-inverse-bg}{HTML}{000000}

    % common color for the border for error outputs.
    \definecolor{outerrorbackground}{HTML}{FFDFDF}

    % commands and environments needed by pandoc snippets
    % extracted from the output of `pandoc -s`
    \providecommand{\tightlist}{%
      \setlength{\itemsep}{0pt}\setlength{\parskip}{0pt}}
    \DefineVerbatimEnvironment{Highlighting}{Verbatim}{commandchars=\\\{\}}
    % Add ',fontsize=\small' for more characters per line
    \newenvironment{Shaded}{}{}
    \newcommand{\KeywordTok}[1]{\textcolor[rgb]{0.00,0.44,0.13}{\textbf{{#1}}}}
    \newcommand{\DataTypeTok}[1]{\textcolor[rgb]{0.56,0.13,0.00}{{#1}}}
    \newcommand{\DecValTok}[1]{\textcolor[rgb]{0.25,0.63,0.44}{{#1}}}
    \newcommand{\BaseNTok}[1]{\textcolor[rgb]{0.25,0.63,0.44}{{#1}}}
    \newcommand{\FloatTok}[1]{\textcolor[rgb]{0.25,0.63,0.44}{{#1}}}
    \newcommand{\CharTok}[1]{\textcolor[rgb]{0.25,0.44,0.63}{{#1}}}
    \newcommand{\StringTok}[1]{\textcolor[rgb]{0.25,0.44,0.63}{{#1}}}
    \newcommand{\CommentTok}[1]{\textcolor[rgb]{0.38,0.63,0.69}{\textit{{#1}}}}
    \newcommand{\OtherTok}[1]{\textcolor[rgb]{0.00,0.44,0.13}{{#1}}}
    \newcommand{\AlertTok}[1]{\textcolor[rgb]{1.00,0.00,0.00}{\textbf{{#1}}}}
    \newcommand{\FunctionTok}[1]{\textcolor[rgb]{0.02,0.16,0.49}{{#1}}}
    \newcommand{\RegionMarkerTok}[1]{{#1}}
    \newcommand{\ErrorTok}[1]{\textcolor[rgb]{1.00,0.00,0.00}{\textbf{{#1}}}}
    \newcommand{\NormalTok}[1]{{#1}}

    % Additional commands for more recent versions of Pandoc
    \newcommand{\ConstantTok}[1]{\textcolor[rgb]{0.53,0.00,0.00}{{#1}}}
    \newcommand{\SpecialCharTok}[1]{\textcolor[rgb]{0.25,0.44,0.63}{{#1}}}
    \newcommand{\VerbatimStringTok}[1]{\textcolor[rgb]{0.25,0.44,0.63}{{#1}}}
    \newcommand{\SpecialStringTok}[1]{\textcolor[rgb]{0.73,0.40,0.53}{{#1}}}
    \newcommand{\ImportTok}[1]{{#1}}
    \newcommand{\DocumentationTok}[1]{\textcolor[rgb]{0.73,0.13,0.13}{\textit{{#1}}}}
    \newcommand{\AnnotationTok}[1]{\textcolor[rgb]{0.38,0.63,0.69}{\textbf{\textit{{#1}}}}}
    \newcommand{\CommentVarTok}[1]{\textcolor[rgb]{0.38,0.63,0.69}{\textbf{\textit{{#1}}}}}
    \newcommand{\VariableTok}[1]{\textcolor[rgb]{0.10,0.09,0.49}{{#1}}}
    \newcommand{\ControlFlowTok}[1]{\textcolor[rgb]{0.00,0.44,0.13}{\textbf{{#1}}}}
    \newcommand{\OperatorTok}[1]{\textcolor[rgb]{0.40,0.40,0.40}{{#1}}}
    \newcommand{\BuiltInTok}[1]{{#1}}
    \newcommand{\ExtensionTok}[1]{{#1}}
    \newcommand{\PreprocessorTok}[1]{\textcolor[rgb]{0.74,0.48,0.00}{{#1}}}
    \newcommand{\AttributeTok}[1]{\textcolor[rgb]{0.49,0.56,0.16}{{#1}}}
    \newcommand{\InformationTok}[1]{\textcolor[rgb]{0.38,0.63,0.69}{\textbf{\textit{{#1}}}}}
    \newcommand{\WarningTok}[1]{\textcolor[rgb]{0.38,0.63,0.69}{\textbf{\textit{{#1}}}}}


    % Define a nice break command that doesn't care if a line doesn't already
    % exist.
    \def\br{\hspace*{\fill} \\* }
    % Math Jax compatibility definitions
    \def\gt{>}
    \def\lt{<}
    \let\Oldtex\TeX
    \let\Oldlatex\LaTeX
    \renewcommand{\TeX}{\textrm{\Oldtex}}
    \renewcommand{\LaTeX}{\textrm{\Oldlatex}}
    % Document parameters
    % Document title
    \title{Cấu trúc rời rạc}
    
    
    
    
    
    
    
% Pygments definitions
\makeatletter
\def\PY@reset{\let\PY@it=\relax \let\PY@bf=\relax%
    \let\PY@ul=\relax \let\PY@tc=\relax%
    \let\PY@bc=\relax \let\PY@ff=\relax}
\def\PY@tok#1{\csname PY@tok@#1\endcsname}
\def\PY@toks#1+{\ifx\relax#1\empty\else%
    \PY@tok{#1}\expandafter\PY@toks\fi}
\def\PY@do#1{\PY@bc{\PY@tc{\PY@ul{%
    \PY@it{\PY@bf{\PY@ff{#1}}}}}}}
\def\PY#1#2{\PY@reset\PY@toks#1+\relax+\PY@do{#2}}

\@namedef{PY@tok@w}{\def\PY@tc##1{\textcolor[rgb]{0.73,0.73,0.73}{##1}}}
\@namedef{PY@tok@c}{\let\PY@it=\textit\def\PY@tc##1{\textcolor[rgb]{0.24,0.48,0.48}{##1}}}
\@namedef{PY@tok@cp}{\def\PY@tc##1{\textcolor[rgb]{0.61,0.40,0.00}{##1}}}
\@namedef{PY@tok@k}{\let\PY@bf=\textbf\def\PY@tc##1{\textcolor[rgb]{0.00,0.50,0.00}{##1}}}
\@namedef{PY@tok@kp}{\def\PY@tc##1{\textcolor[rgb]{0.00,0.50,0.00}{##1}}}
\@namedef{PY@tok@kt}{\def\PY@tc##1{\textcolor[rgb]{0.69,0.00,0.25}{##1}}}
\@namedef{PY@tok@o}{\def\PY@tc##1{\textcolor[rgb]{0.40,0.40,0.40}{##1}}}
\@namedef{PY@tok@ow}{\let\PY@bf=\textbf\def\PY@tc##1{\textcolor[rgb]{0.67,0.13,1.00}{##1}}}
\@namedef{PY@tok@nb}{\def\PY@tc##1{\textcolor[rgb]{0.00,0.50,0.00}{##1}}}
\@namedef{PY@tok@nf}{\def\PY@tc##1{\textcolor[rgb]{0.00,0.00,1.00}{##1}}}
\@namedef{PY@tok@nc}{\let\PY@bf=\textbf\def\PY@tc##1{\textcolor[rgb]{0.00,0.00,1.00}{##1}}}
\@namedef{PY@tok@nn}{\let\PY@bf=\textbf\def\PY@tc##1{\textcolor[rgb]{0.00,0.00,1.00}{##1}}}
\@namedef{PY@tok@ne}{\let\PY@bf=\textbf\def\PY@tc##1{\textcolor[rgb]{0.80,0.25,0.22}{##1}}}
\@namedef{PY@tok@nv}{\def\PY@tc##1{\textcolor[rgb]{0.10,0.09,0.49}{##1}}}
\@namedef{PY@tok@no}{\def\PY@tc##1{\textcolor[rgb]{0.53,0.00,0.00}{##1}}}
\@namedef{PY@tok@nl}{\def\PY@tc##1{\textcolor[rgb]{0.46,0.46,0.00}{##1}}}
\@namedef{PY@tok@ni}{\let\PY@bf=\textbf\def\PY@tc##1{\textcolor[rgb]{0.44,0.44,0.44}{##1}}}
\@namedef{PY@tok@na}{\def\PY@tc##1{\textcolor[rgb]{0.41,0.47,0.13}{##1}}}
\@namedef{PY@tok@nt}{\let\PY@bf=\textbf\def\PY@tc##1{\textcolor[rgb]{0.00,0.50,0.00}{##1}}}
\@namedef{PY@tok@nd}{\def\PY@tc##1{\textcolor[rgb]{0.67,0.13,1.00}{##1}}}
\@namedef{PY@tok@s}{\def\PY@tc##1{\textcolor[rgb]{0.73,0.13,0.13}{##1}}}
\@namedef{PY@tok@sd}{\let\PY@it=\textit\def\PY@tc##1{\textcolor[rgb]{0.73,0.13,0.13}{##1}}}
\@namedef{PY@tok@si}{\let\PY@bf=\textbf\def\PY@tc##1{\textcolor[rgb]{0.64,0.35,0.47}{##1}}}
\@namedef{PY@tok@se}{\let\PY@bf=\textbf\def\PY@tc##1{\textcolor[rgb]{0.67,0.36,0.12}{##1}}}
\@namedef{PY@tok@sr}{\def\PY@tc##1{\textcolor[rgb]{0.64,0.35,0.47}{##1}}}
\@namedef{PY@tok@ss}{\def\PY@tc##1{\textcolor[rgb]{0.10,0.09,0.49}{##1}}}
\@namedef{PY@tok@sx}{\def\PY@tc##1{\textcolor[rgb]{0.00,0.50,0.00}{##1}}}
\@namedef{PY@tok@m}{\def\PY@tc##1{\textcolor[rgb]{0.40,0.40,0.40}{##1}}}
\@namedef{PY@tok@gh}{\let\PY@bf=\textbf\def\PY@tc##1{\textcolor[rgb]{0.00,0.00,0.50}{##1}}}
\@namedef{PY@tok@gu}{\let\PY@bf=\textbf\def\PY@tc##1{\textcolor[rgb]{0.50,0.00,0.50}{##1}}}
\@namedef{PY@tok@gd}{\def\PY@tc##1{\textcolor[rgb]{0.63,0.00,0.00}{##1}}}
\@namedef{PY@tok@gi}{\def\PY@tc##1{\textcolor[rgb]{0.00,0.52,0.00}{##1}}}
\@namedef{PY@tok@gr}{\def\PY@tc##1{\textcolor[rgb]{0.89,0.00,0.00}{##1}}}
\@namedef{PY@tok@ge}{\let\PY@it=\textit}
\@namedef{PY@tok@gs}{\let\PY@bf=\textbf}
\@namedef{PY@tok@ges}{\let\PY@bf=\textbf\let\PY@it=\textit}
\@namedef{PY@tok@gp}{\let\PY@bf=\textbf\def\PY@tc##1{\textcolor[rgb]{0.00,0.00,0.50}{##1}}}
\@namedef{PY@tok@go}{\def\PY@tc##1{\textcolor[rgb]{0.44,0.44,0.44}{##1}}}
\@namedef{PY@tok@gt}{\def\PY@tc##1{\textcolor[rgb]{0.00,0.27,0.87}{##1}}}
\@namedef{PY@tok@err}{\def\PY@bc##1{{\setlength{\fboxsep}{\string -\fboxrule}\fcolorbox[rgb]{1.00,0.00,0.00}{1,1,1}{\strut ##1}}}}
\@namedef{PY@tok@kc}{\let\PY@bf=\textbf\def\PY@tc##1{\textcolor[rgb]{0.00,0.50,0.00}{##1}}}
\@namedef{PY@tok@kd}{\let\PY@bf=\textbf\def\PY@tc##1{\textcolor[rgb]{0.00,0.50,0.00}{##1}}}
\@namedef{PY@tok@kn}{\let\PY@bf=\textbf\def\PY@tc##1{\textcolor[rgb]{0.00,0.50,0.00}{##1}}}
\@namedef{PY@tok@kr}{\let\PY@bf=\textbf\def\PY@tc##1{\textcolor[rgb]{0.00,0.50,0.00}{##1}}}
\@namedef{PY@tok@bp}{\def\PY@tc##1{\textcolor[rgb]{0.00,0.50,0.00}{##1}}}
\@namedef{PY@tok@fm}{\def\PY@tc##1{\textcolor[rgb]{0.00,0.00,1.00}{##1}}}
\@namedef{PY@tok@vc}{\def\PY@tc##1{\textcolor[rgb]{0.10,0.09,0.49}{##1}}}
\@namedef{PY@tok@vg}{\def\PY@tc##1{\textcolor[rgb]{0.10,0.09,0.49}{##1}}}
\@namedef{PY@tok@vi}{\def\PY@tc##1{\textcolor[rgb]{0.10,0.09,0.49}{##1}}}
\@namedef{PY@tok@vm}{\def\PY@tc##1{\textcolor[rgb]{0.10,0.09,0.49}{##1}}}
\@namedef{PY@tok@sa}{\def\PY@tc##1{\textcolor[rgb]{0.73,0.13,0.13}{##1}}}
\@namedef{PY@tok@sb}{\def\PY@tc##1{\textcolor[rgb]{0.73,0.13,0.13}{##1}}}
\@namedef{PY@tok@sc}{\def\PY@tc##1{\textcolor[rgb]{0.73,0.13,0.13}{##1}}}
\@namedef{PY@tok@dl}{\def\PY@tc##1{\textcolor[rgb]{0.73,0.13,0.13}{##1}}}
\@namedef{PY@tok@s2}{\def\PY@tc##1{\textcolor[rgb]{0.73,0.13,0.13}{##1}}}
\@namedef{PY@tok@sh}{\def\PY@tc##1{\textcolor[rgb]{0.73,0.13,0.13}{##1}}}
\@namedef{PY@tok@s1}{\def\PY@tc##1{\textcolor[rgb]{0.73,0.13,0.13}{##1}}}
\@namedef{PY@tok@mb}{\def\PY@tc##1{\textcolor[rgb]{0.40,0.40,0.40}{##1}}}
\@namedef{PY@tok@mf}{\def\PY@tc##1{\textcolor[rgb]{0.40,0.40,0.40}{##1}}}
\@namedef{PY@tok@mh}{\def\PY@tc##1{\textcolor[rgb]{0.40,0.40,0.40}{##1}}}
\@namedef{PY@tok@mi}{\def\PY@tc##1{\textcolor[rgb]{0.40,0.40,0.40}{##1}}}
\@namedef{PY@tok@il}{\def\PY@tc##1{\textcolor[rgb]{0.40,0.40,0.40}{##1}}}
\@namedef{PY@tok@mo}{\def\PY@tc##1{\textcolor[rgb]{0.40,0.40,0.40}{##1}}}
\@namedef{PY@tok@ch}{\let\PY@it=\textit\def\PY@tc##1{\textcolor[rgb]{0.24,0.48,0.48}{##1}}}
\@namedef{PY@tok@cm}{\let\PY@it=\textit\def\PY@tc##1{\textcolor[rgb]{0.24,0.48,0.48}{##1}}}
\@namedef{PY@tok@cpf}{\let\PY@it=\textit\def\PY@tc##1{\textcolor[rgb]{0.24,0.48,0.48}{##1}}}
\@namedef{PY@tok@c1}{\let\PY@it=\textit\def\PY@tc##1{\textcolor[rgb]{0.24,0.48,0.48}{##1}}}
\@namedef{PY@tok@cs}{\let\PY@it=\textit\def\PY@tc##1{\textcolor[rgb]{0.24,0.48,0.48}{##1}}}

\def\PYZbs{\char`\\}
\def\PYZus{\char`\_}
\def\PYZob{\char`\{}
\def\PYZcb{\char`\}}
\def\PYZca{\char`\^}
\def\PYZam{\char`\&}
\def\PYZlt{\char`\<}
\def\PYZgt{\char`\>}
\def\PYZsh{\char`\#}
\def\PYZpc{\char`\%}
\def\PYZdl{\char`\$}
\def\PYZhy{\char`\-}
\def\PYZsq{\char`\'}
\def\PYZdq{\char`\"}
\def\PYZti{\char`\~}
% for compatibility with earlier versions
\def\PYZat{@}
\def\PYZlb{[}
\def\PYZrb{]}
\makeatother


    % For linebreaks inside Verbatim environment from package fancyvrb.
    \makeatletter
        \newbox\Wrappedcontinuationbox
        \newbox\Wrappedvisiblespacebox
        \newcommand*\Wrappedvisiblespace {\textcolor{red}{\textvisiblespace}}
        \newcommand*\Wrappedcontinuationsymbol {\textcolor{red}{\llap{\tiny$\m@th\hookrightarrow$}}}
        \newcommand*\Wrappedcontinuationindent {3ex }
        \newcommand*\Wrappedafterbreak {\kern\Wrappedcontinuationindent\copy\Wrappedcontinuationbox}
        % Take advantage of the already applied Pygments mark-up to insert
        % potential linebreaks for TeX processing.
        %        {, <, #, %, $, ' and ": go to next line.
        %        _, }, ^, &, >, - and ~: stay at end of broken line.
        % Use of \textquotesingle for straight quote.
        \newcommand*\Wrappedbreaksatspecials {%
            \def\PYGZus{\discretionary{\char`\_}{\Wrappedafterbreak}{\char`\_}}%
            \def\PYGZob{\discretionary{}{\Wrappedafterbreak\char`\{}{\char`\{}}%
            \def\PYGZcb{\discretionary{\char`\}}{\Wrappedafterbreak}{\char`\}}}%
            \def\PYGZca{\discretionary{\char`\^}{\Wrappedafterbreak}{\char`\^}}%
            \def\PYGZam{\discretionary{\char`\&}{\Wrappedafterbreak}{\char`\&}}%
            \def\PYGZlt{\discretionary{}{\Wrappedafterbreak\char`\<}{\char`\<}}%
            \def\PYGZgt{\discretionary{\char`\>}{\Wrappedafterbreak}{\char`\>}}%
            \def\PYGZsh{\discretionary{}{\Wrappedafterbreak\char`\#}{\char`\#}}%
            \def\PYGZpc{\discretionary{}{\Wrappedafterbreak\char`\%}{\char`\%}}%
            \def\PYGZdl{\discretionary{}{\Wrappedafterbreak\char`\$}{\char`\$}}%
            \def\PYGZhy{\discretionary{\char`\-}{\Wrappedafterbreak}{\char`\-}}%
            \def\PYGZsq{\discretionary{}{\Wrappedafterbreak\textquotesingle}{\textquotesingle}}%
            \def\PYGZdq{\discretionary{}{\Wrappedafterbreak\char`\"}{\char`\"}}%
            \def\PYGZti{\discretionary{\char`\~}{\Wrappedafterbreak}{\char`\~}}%
        }
        % Some characters . , ; ? ! / are not pygmentized.
        % This macro makes them "active" and they will insert potential linebreaks
        \newcommand*\Wrappedbreaksatpunct {%
            \lccode`\~`\.\lowercase{\def~}{\discretionary{\hbox{\char`\.}}{\Wrappedafterbreak}{\hbox{\char`\.}}}%
            \lccode`\~`\,\lowercase{\def~}{\discretionary{\hbox{\char`\,}}{\Wrappedafterbreak}{\hbox{\char`\,}}}%
            \lccode`\~`\;\lowercase{\def~}{\discretionary{\hbox{\char`\;}}{\Wrappedafterbreak}{\hbox{\char`\;}}}%
            \lccode`\~`\:\lowercase{\def~}{\discretionary{\hbox{\char`\:}}{\Wrappedafterbreak}{\hbox{\char`\:}}}%
            \lccode`\~`\?\lowercase{\def~}{\discretionary{\hbox{\char`\?}}{\Wrappedafterbreak}{\hbox{\char`\?}}}%
            \lccode`\~`\!\lowercase{\def~}{\discretionary{\hbox{\char`\!}}{\Wrappedafterbreak}{\hbox{\char`\!}}}%
            \lccode`\~`\/\lowercase{\def~}{\discretionary{\hbox{\char`\/}}{\Wrappedafterbreak}{\hbox{\char`\/}}}%
            \catcode`\.\active
            \catcode`\,\active
            \catcode`\;\active
            \catcode`\:\active
            \catcode`\?\active
            \catcode`\!\active
            \catcode`\/\active
            \lccode`\~`\~
        }
    \makeatother

    \let\OriginalVerbatim=\Verbatim
    \makeatletter
    \renewcommand{\Verbatim}[1][1]{%
        %\parskip\z@skip
        \sbox\Wrappedcontinuationbox {\Wrappedcontinuationsymbol}%
        \sbox\Wrappedvisiblespacebox {\FV@SetupFont\Wrappedvisiblespace}%
        \def\FancyVerbFormatLine ##1{\hsize\linewidth
            \vtop{\raggedright\hyphenpenalty\z@\exhyphenpenalty\z@
                \doublehyphendemerits\z@\finalhyphendemerits\z@
                \strut ##1\strut}%
        }%
        % If the linebreak is at a space, the latter will be displayed as visible
        % space at end of first line, and a continuation symbol starts next line.
        % Stretch/shrink are however usually zero for typewriter font.
        \def\FV@Space {%
            \nobreak\hskip\z@ plus\fontdimen3\font minus\fontdimen4\font
            \discretionary{\copy\Wrappedvisiblespacebox}{\Wrappedafterbreak}
            {\kern\fontdimen2\font}%
        }%

        % Allow breaks at special characters using \PYG... macros.
        \Wrappedbreaksatspecials
        % Breaks at punctuation characters . , ; ? ! and / need catcode=\active
        \OriginalVerbatim[#1,codes*=\Wrappedbreaksatpunct]%
    }
    \makeatother

    % Exact colors from NB
    \definecolor{incolor}{HTML}{303F9F}
    \definecolor{outcolor}{HTML}{D84315}
    \definecolor{cellborder}{HTML}{CFCFCF}
    \definecolor{cellbackground}{HTML}{F7F7F7}

    % prompt
    \makeatletter
    \newcommand{\boxspacing}{\kern\kvtcb@left@rule\kern\kvtcb@boxsep}
    \makeatother
    \newcommand{\prompt}[4]{
        {\ttfamily\llap{{\color{#2}[#3]:\hspace{3pt}#4}}\vspace{-\baselineskip}}
    }
    

    
    % Prevent overflowing lines due to hard-to-break entities
    \sloppy
    % Setup hyperref package
    \hypersetup{
      breaklinks=true,  % so long urls are correctly broken across lines
      colorlinks=true,
      urlcolor=urlcolor,
      linkcolor=linkcolor,
      citecolor=citecolor,
      }
    % Slightly bigger margins than the latex defaults
    
    \geometry{verbose,tmargin=1in,bmargin=1in,lmargin=1in,rmargin=1in}
    
    

\begin{document}
    
    \maketitle
    
    

    
    \section{Cơ sở Logic}\label{cux1a1-sux1edf-logic}

    \subsection{Phép toán}\label{phuxe9p-touxe1n}

    \subsubsection{Phủ định}\label{phux1ee7-ux111ux1ecbnh}

{\def\LTcaptype{none} % do not increment counter
\begin{longtable}[]{@{}ll@{}}
\toprule\noalign{}
\(P\) & \(\neg P\) \\
\midrule\noalign{}
\endhead
\bottomrule\noalign{}
\endlastfoot
0 & 1 \\
1 & 0 \\
\end{longtable}
}

\subsubsection{Hội (và)}\label{hux1ed9i-vuxe0}

{\def\LTcaptype{none} % do not increment counter
\begin{longtable}[]{@{}lll@{}}
\toprule\noalign{}
\(P\) & \(Q\) & \(P \land Q\) \\
\midrule\noalign{}
\endhead
\bottomrule\noalign{}
\endlastfoot
0 & 0 & 0 \\
0 & 1 & 0 \\
1 & 0 & 0 \\
\textbf{1} & \textbf{1} & \textbf{1} \\
\end{longtable}
}

\subsubsection{Tuyển (hoặc)}\label{tuyux1ec3n-houx1eb7c}

{\def\LTcaptype{none} % do not increment counter
\begin{longtable}[]{@{}lll@{}}
\toprule\noalign{}
\(P\) & \(Q\) & \(P \lor Q\) \\
\midrule\noalign{}
\endhead
\bottomrule\noalign{}
\endlastfoot
\textbf{0} & \textbf{0} & \textbf{0} \\
0 & 1 & 1 \\
1 & 0 & 1 \\
1 & 1 & 1 \\
\end{longtable}
}

\subsubsection{Kéo theo}\label{kuxe9o-theo}

{\def\LTcaptype{none} % do not increment counter
\begin{longtable}[]{@{}lll@{}}
\toprule\noalign{}
\(P\) & \(Q\) & \(P \to Q\) \\
\midrule\noalign{}
\endhead
\bottomrule\noalign{}
\endlastfoot
0 & 0 & 1 \\
0 & 1 & 1 \\
\textbf{1} & \textbf{0} & \textbf{0} \\
1 & 1 & 1 \\
\end{longtable}
}

\subsubsection{Tương đương (Nếu và chỉ
nếu)}\label{tux1b0ux1a1ng-ux111ux1b0ux1a1ng-nux1ebfu-vuxe0-chux1ec9-nux1ebfu}

{\def\LTcaptype{none} % do not increment counter
\begin{longtable}[]{@{}lll@{}}
\toprule\noalign{}
\(P\) & \(Q\) & \(P \leftrightarrow Q\) \\
\midrule\noalign{}
\endhead
\bottomrule\noalign{}
\endlastfoot
\textbf{0} & \textbf{0} & \textbf{1} \\
0 & 1 & 0 \\
1 & 0 & 0 \\
\textbf{1} & \textbf{1} & \textbf{1} \\
\end{longtable}
}

    \subsection{Dạng mệnh đề}\label{dux1ea1ng-mux1ec7nh-ux111ux1ec1}

    \begin{itemize}
\item
  Sơ cấp
\item
  Hằng đúng
\item
  Hằng sai
\end{itemize}

    \subsection{Tương đương logic \& Hệ quả
logic}\label{tux1b0ux1a1ng-ux111ux1b0ux1a1ng-logic-hux1ec7-quux1ea3-logic}

    \subsubsection{Tương đương
logic}\label{tux1b0ux1a1ng-ux111ux1b0ux1a1ng-logic}

    \begin{itemize}
\item
  \(P, \: Q \to\) công thức \(\ldots\)
\item
  \(P \Leftrightarrow Q (P \equiv Q, P = Q)\) đgl tương đương logic.
\item
  \(P \leftrightarrow Q\) đgl hằng đúng.
\end{itemize}

    \subsubsection{Hệ quả logic}\label{hux1ec7-quux1ea3-logic}

    \begin{itemize}
\tightlist
\item
  \(P \Rightarrow Q\) đgl hệ quả logic.
\item
  \(P \to Q\) đgl hằng đúng.
\end{itemize}

    \textbf{Note}

\begin{itemize}
\tightlist
\item
  C/m \(P \Rightarrow Q\) \(\to\) c/m \(P \to Q\) chân trị \(1\).
\end{itemize}

    \subsection{Quy luật logic}\label{quy-luux1eadt-logic}

    {\def\LTcaptype{none} % do not increment counter
\begin{longtable}[]{@{}
  >{\raggedright\arraybackslash}p{(\linewidth - 2\tabcolsep) * \real{0.5000}}
  >{\raggedright\arraybackslash}p{(\linewidth - 2\tabcolsep) * \real{0.5000}}@{}}
\toprule\noalign{}
\begin{minipage}[b]{\linewidth}\raggedright
\textbf{Luật}
\end{minipage} & \begin{minipage}[b]{\linewidth}\raggedright
\textbf{Công thức}
\end{minipage} \\
\midrule\noalign{}
\endhead
\bottomrule\noalign{}
\endlastfoot
Phủ định của phủ định & \(\neg \neg P \equiv P\) \\
De Morgan & \(\neg (P \land Q) \equiv \neg P \lor \neg Q\)
\(\neg (P \lor Q) \equiv \neg P \land \neg Q\) \\
Giao hoán & \(P \lor Q \equiv Q \lor P\)
\(P \land Q \equiv Q \land P\) \\
Kết hợp & \(P \land (Q \land R) \equiv (P \land Q) \land R\)
\(P \lor (Q \lor R) \equiv (P \lor Q) \lor R\) \\
Phân phối & \(P \land (Q \lor R) \equiv (P \land Q) \lor (P \land R)\)
\(P \lor (Q \land R) \equiv (P \lor Q) \land (P \lor R)\) \\
Luỹ đẳng & \(P \land P \equiv P\) \(P \lor P \equiv P\) \\
Trung hoà & \(P \land 1 \equiv P\) \(P \lor 0 \equiv P\) \\
Phần tử bù & \(P \land \neg P \equiv 0\) \(P \lor \neg P \equiv 1\) \\
Thống trị & \(P \land 0 \equiv 0\) \(P \lor 1 \equiv 1\) \\
Hấp thụ & \(P \land (P \lor Q) \equiv P\)
\(P \lor (P \land Q) \equiv P\) \\
Phản chứng & \(P \to Q \equiv \neg P \lor Q \equiv \neg Q \to \neg P\)
\(\neg (P \to Q) \equiv P \land \neg Q\) \\
\end{longtable}
}

    \subsection{Quy tắc suy diễn}\label{quy-tux1eafc-suy-diux1ec5n}

    \begin{itemize}
\tightlist
\item
  Khẳng định
\end{itemize}

\[
\frac{P \to Q, \: P}{\therefore Q}
\]

\begin{itemize}
\tightlist
\item
  Phủ định
\end{itemize}

\[
\frac{P \to Q, \: \neg Q}{\therefore \neg P}
\]

\begin{itemize}
\tightlist
\item
  Tam đoạn luận
\end{itemize}

\[
\frac{P \to Q, \: Q \to R}{\therefore P \to R}
\]

\begin{itemize}
\tightlist
\item
  Tam đoạn luận rời
\end{itemize}

\[
\frac{P \lor Q, \: \neg Q}{\therefore P}
\]

or

\[
\frac{P \lor Q, \: \neg P}{\therefore Q}
\]

\begin{itemize}
\tightlist
\item
  Mâu thuẫn
\end{itemize}

\[
P \to Q \equiv (P \land \neg Q) \to 0
\]

trong đó \(P = P_1 \land P_2 \land \ldots \land P_N\)

    \subsection{Vị từ - lượng
từ}\label{vux1ecb-tux1eeb---lux1b0ux1ee3ng-tux1eeb}

    \subsubsection{Vị từ}\label{vux1ecb-tux1eeb}

    \(P(x,y,\ldots) \to P(a,b,\ldots)\) có chân trị \(0\) hoặc \(1\).

    \subsubsection{Lượng từ}\label{lux1b0ux1ee3ng-tux1eeb}

    \paragraph{Với mọi}\label{vux1edbi-mux1ecdi}

    \begin{itemize}
\item
  \(\forall\) trong đó \(\land\)

  \begin{itemize}
  \item
    \textbf{Đúng} với tất cả.
  \item
    \textbf{Sai} với một.
  \end{itemize}
\end{itemize}

    \begin{itemize}
\tightlist
\item
  \(\neg (\forall x \in A, \: P(x)) \equiv \exists x \in A, \neg P(x)\)
\end{itemize}

    \paragraph{Tồn tại}\label{tux1ed3n-tux1ea1i}

    \begin{itemize}
\item
  \(\exists\) trong đó \(\lor\)

  \begin{itemize}
  \item
    \textbf{Đúng} với một.
  \item
    \textbf{Sai} với tất cả.
  \end{itemize}
\end{itemize}

    \begin{itemize}
\tightlist
\item
  \(\neg (\exists x \in A, \: P(x)) \equiv \forall x \in A, \: \neg P(x)\)
\end{itemize}

    \paragraph{Đặc biệt hoá phổ
dụng}\label{ux111ux1eb7c-biux1ec7t-houxe1-phux1ed5-dux1ee5ng}

\paragraph{Tổng quát hoá phổ
dụng}\label{tux1ed5ng-quuxe1t-houxe1-phux1ed5-dux1ee5ng}

    \subsubsection{Quy tắc suy diễn}\label{quy-tux1eafc-suy-diux1ec5n}

    \paragraph{C/m Phản chứng}\label{cm-phux1ea3n-chux1ee9ng}

\[
P \Rightarrow Q \equiv (P \land \neg Q) \Rightarrow 0
\]

    \paragraph{C/m trực tiếp}\label{cm-trux1ef1c-tiux1ebfp}

    \paragraph{C/m theo trường hợp (vét
cạn)}\label{cm-theo-trux1b0ux1eddng-hux1ee3p-vuxe9t-cux1ea1n}

    \paragraph{C/m gián tiếp (PC)}\label{cm-giuxe1n-tiux1ebfp-pc}

    \(P \Rightarrow Q\) và \(\neg Q \Rightarrow \neg P\)

\$ \neg Q \Rightarrow \neg P \equiv P \Rightarrow Q \$

    \paragraph{C/m quy nạp}\label{cm-quy-nux1ea1p}

    \[
\frac{P(n_0), \quad \forall n > n_0, \: P(n) \to P(n+1)}{\therefore \forall n \geq n_0, \: P(n)}
\]

    B1. C/m \(P(n_0)\) đúng

B2. G/s \(n \in \mathbb{N}\) và \(n \geq n_0\), \(P(n)\) đúng. C/m
\(P(n+1)\) đúng.

\(\Rightarrow \: P(n)\) đúng \(\forall n \geq n_0\)

    \section{Tập hợp - Ánh xạ}\label{tux1eadp-hux1ee3p---uxe1nh-xux1ea1}

    \subsection{Tập hợp}\label{tux1eadp-hux1ee3p}

    \begin{itemize}
\item
  Cách diễn tả

  \begin{itemize}
  \item
    Bằng lời
  \item
    Liệt kê
  \item
    Tính chất đặc trưng
  \end{itemize}
\end{itemize}

    \begin{itemize}
\tightlist
\item
  Lực lượng k/h \(|A|\)
\end{itemize}

    \begin{itemize}
\item
  Tích Descarte

  \begin{itemize}
  \item
    2 tập hợp
  \item
    Nhiều tập hợp
  \end{itemize}
\end{itemize}

    \begin{itemize}
\item
  Tập con

  \begin{itemize}
  \item
    k/h
    \(B \subset A \Leftrightarrow \{ \forall x | x \in B \Rightarrow x \in A\}\)
  \item
    Tập hợp tập con của \(A\) k/h \(P(A)\)
  \end{itemize}
\end{itemize}

    \subsubsection{Phép toán}\label{phuxe9p-touxe1n}

    \begin{itemize}
\tightlist
\item
  Hợp \(A \cup B = \{ x | x \in A \lor x \in B \}\)
\item
  Giao \(A \cap B = \{ x | x \in A \land x \in B \}\)
\item
  Hiệu \(A \backslash B = \{ x | x \in A \land x \notin B \}\)
\item
  Phần bù \(B \subset A\), k/h \(\overline{B_A}\) or \(\overline{B}\)
\end{itemize}

    \paragraph{Tính chất}\label{tuxednh-chux1ea5t}

    {\def\LTcaptype{none} % do not increment counter
\begin{longtable}[]{@{}
  >{\raggedright\arraybackslash}p{(\linewidth - 2\tabcolsep) * \real{0.5000}}
  >{\raggedright\arraybackslash}p{(\linewidth - 2\tabcolsep) * \real{0.5000}}@{}}
\toprule\noalign{}
\begin{minipage}[b]{\linewidth}\raggedright
\textbf{Luật}
\end{minipage} & \begin{minipage}[b]{\linewidth}\raggedright
\textbf{Công thức}
\end{minipage} \\
\midrule\noalign{}
\endhead
\bottomrule\noalign{}
\endlastfoot
Giao hoán & \(A \cap B = B \cap A\) \(A \cup B = B \cup A\) \\
Kết hợp & \((A \cap B) \cap C = A \cap (B \cap C)\)
\((A \cup B) \cup C = A \cup (B \cup C)\) \\
Phân phối & \(A \cap (B \cup C) = (A \cap B) \cup (A \cap C)\)
\(A \cup (B \cap C) = (A \cup B) \cap (A \cup C)\) \\
De Morgan & \(\overline{A \cap B} = \overline{A} \cup \overline{B}\)
\(\overline{A \cup B} = \overline{A} \cap \overline{B}\) \\
Luỹ đẳng & \(A \cap A = A\) \(A \cup A = A\) \\
\end{longtable}
}

    \subsection{Ánh xạ}\label{uxe1nh-xux1ea1}

    Ánh xạ bằng nhau

Ảnh

Ảnh ngược

Tính chất

    \subsubsection{Loại}\label{loux1ea1i}

    \paragraph{Đơn ánh}\label{ux111ux1a1n-uxe1nh}

\begin{itemize}
\tightlist
\item
  Có tối đa 1 nghiệm
\end{itemize}

\[
\forall x, x' \in X, x \neq x' \Rightarrow f(x) \neq f(x')
\]

hay

\[
\forall x, x' \in X, f(x) \neq f(x') \Rightarrow x \neq x' 
\]

\begin{itemize}
\tightlist
\item
  \textbf{Không} là đơn ánh (có nhiều hơn 1 nghiệm)
\end{itemize}

\[
\exists x, x' \in X, x \neq x' \land f(x) = f(x')
\]

    \paragraph{Toàn ánh}\label{touxe0n-uxe1nh}

\begin{itemize}
\tightlist
\item
  Luôn có ít nhất 1 nghiệm
\end{itemize}

\[
\forall y \in Y, \exists x \in X, y=f(x)
\]

hay

\[
\forall y \in Y, f^{-1}(y) \neq \varnothing
\]

\begin{itemize}
\tightlist
\item
  \textbf{Không} là toàn ánh (Tồn tại vô nghiệm)
\end{itemize}

\[
\exists y \in Y, f^{-1} (y) = \varnothing
\]

hay

\[
\exists y \in Y, \forall x \in X, y \neq f(x)
\]

    \paragraph{Song ánh}\label{song-uxe1nh}

\begin{itemize}
\tightlist
\item
  Vừa đơn ánh vừa toàn ánh.
\end{itemize}

\[
\forall y \in Y, \exists ! x \in X, y = f(x)
\]

hay

\[
\forall y \in Y, f^{-1}(y) \text{ có đúng 1 phần tử}
\]

    \subsubsection{Ngược}\label{ngux1b0ux1ee3c}

    \[
f^{-1} : Y \to X \\
y \mapsto f^{-1} (y) = x, \quad f(x) = y
\]

    \subsubsection{Hợp}\label{hux1ee3p}

    \[
h = g \circ f : X \to Y \to Z \\
x \mapsto f(x) \mapsto g(f(x))
\]

    \begin{itemize}
\tightlist
\item
  Định lý
\end{itemize}

\(f: X \to Y\) song ánh

\[
f \circ f^{-1} = I d_Y \\
f^{-1} \circ f = I d_X
\]

Trong đó

\begin{itemize}
\item
  \(Id_X(x) : X \to X, \quad Id_X(x) = x\)
\item
  \(Id_Y(y) : Y \to Y, \quad Id_Y(y) = y\)
\end{itemize}

    \section{Phương pháp đếm}\label{phux1b0ux1a1ng-phuxe1p-ux111ux1ebfm}

    \subsection{Cơ bản}\label{cux1a1-bux1ea3n}

    \begin{itemize}
\item
  Cộng

  \begin{itemize}
  \tightlist
  \item
    Trường hợp
  \end{itemize}
\item
  Nhân

  \begin{itemize}
  \item
    Bước
  \item
    Thứ tự không quan trọng
  \item
    Bắt buộc
  \end{itemize}
\item
  Nguyên lý chuồng bồ câu

  \begin{itemize}
  \tightlist
  \item
    \(n\) vật, \(k\) hộp \(\to \lceil \frac{n}{k} \rceil\) vật trong ít
    nhất 1 hộp.
  \end{itemize}
\item
  Nguyên lý bù trừ

  \begin{itemize}
  \item
    \(|A \cup B| = |A| + |B| - |A \cap B|\)
  \item
    \(|A \cup B \cup C| = |A| + |B| + |C| - |A \cap B| - |A \cap C| - |B \cap C| + |A \cap B \cap C |\)
  \end{itemize}
\end{itemize}

    \subsection{Hoán vị - Tổ hợp - Chỉnh
hợp}\label{houxe1n-vux1ecb---tux1ed5-hux1ee3p---chux1ec9nh-hux1ee3p}

    \begin{itemize}
\item
  Hoán vị

  \begin{itemize}
  \tightlist
  \item
    \(P_n=n!\)
  \end{itemize}
\item
  Tổ hợp

  \begin{itemize}
  \item
    \(C_n^k = \frac{n!}{(n-k)!k!}\)
  \item
    Không thứ tự, không lặp lại.
  \end{itemize}
\item
  Chỉnh hợp

  \begin{itemize}
  \item
    \(A_n^k = \frac{n!}{(n-k)!}\)
  \item
    Có thứ tự, không lặp lại.
  \end{itemize}
\end{itemize}

    \subsection{Hoán vị - Tổ hợp - Chỉnh hợp
(Lặp)}\label{houxe1n-vux1ecb---tux1ed5-hux1ee3p---chux1ec9nh-hux1ee3p-lux1eb7p}

    \begin{itemize}
\item
  Hoán vị lặp

  \begin{itemize}
  \item
    \(\frac{n!}{n_1!n_2!\ldots n_k!}\)
  \item
    Nhóm \(n_1, n_2, \ldots , n_k\).
  \end{itemize}
\item
  Chỉnh hợp lặp

  \begin{itemize}
  \tightlist
  \item
    \(\overline{A}_n^k = n^k\)
  \end{itemize}
\item
  Tổ hợp lặp

  \begin{itemize}
  \item
    \(\overline{C}_n^k = C_{n+(k-1)}^k\)
  \item
    Bài toán chia \(k\) vật đồng chất vào \(n\) hộp phân biệt

    \begin{itemize}
    \item
      \(x_1+x_2+\ldots + n_n=k\)
    \item
      \(k \to\) không phân biệt.
    \item
      \(n \to\) phân biệt.
    \end{itemize}
  \end{itemize}
\end{itemize}

    \section{\texorpdfstring{Hệ thức truy hồi \(\&\) Hàm
sinh}{Hệ thức truy hồi \textbackslash\& Hàm sinh}}\label{hux1ec7-thux1ee9c-truy-hux1ed3i-huxe0m-sinh}

    \subsection{\texorpdfstring{Tuyến tính thuần nhất bậc
\(k\)}{Tuyến tính thuần nhất bậc k}}\label{tuyux1ebfn-tuxednh-thuux1ea7n-nhux1ea5t-bux1eadc-k}

    (Đề thi sẽ cho (bậc 2, 3 tối đa))

    \[
a_n = c_1 a_{n-1} + c_2 a_{n-2} + \ldots + c_k a_{n-k}
\]

    \begin{itemize}
\item
  \(a_0=C_0, a_1 = C_1 , \ldots , a_{k-1} = C_{k-1}.\)
\item
  Các dạng không phải tuyến tính thuần nhất bậc \(k\):

  \begin{itemize}
  \item
    \(H_n = H_{n-1} \textbf{+ 1}\)
  \item
    \(A_n = \textbf{n} A_{n-1}\)
  \item
    \(B_n = 4B_{n-1} - 3 B_{n-2}^{\textbf{2}}\)
  \end{itemize}
\end{itemize}

    \subsubsection{Phương pháp giải}\label{phux1b0ux1a1ng-phuxe1p-giux1ea3i}

    \paragraph{Phương pháp thế (giải bậc
1)}\label{phux1b0ux1a1ng-phuxe1p-thux1ebf-giux1ea3i-bux1eadc-1}

    B1. Thay \(a_n\) bởi \(a_{n-1}\), \(a_{n-1}\) bởi
\(a_{n-2}, \ldots, a_0\) bởi \(C_0 \to\) Thu được công thức trực tiếp
cho \(a_n\)

B2. Chứng minh tính đúng đắn của công thức \(a_n\)

    \paragraph{Phương pháp phương trình đặc
trưng}\label{phux1b0ux1a1ng-phuxe1p-phux1b0ux1a1ng-truxecnh-ux111ux1eb7c-trux1b0ng}

    \subparagraph{Bậc 2}\label{bux1eadc-2}

\[
a_n = c_1 a_{n-1} + c_2 a_{n-2}
\]

B1. Tính phương trình đặc trưng \(r^2 = c_1 r + c_2 \to\) nghiệm \(r\)

B2. (Check nghiệm)

\begin{itemize}
\item
  2 nghiệm phân biệt \(r_1, r_2 \to a_n = d_1 r_1^n + d_2 r_2^n\)
\item
  Nghiệm kép \(r_1 \to a_n = (d_1 + d_2 n) r_1^n\) và
  \(a_0 = C_0, a_1 = C_1\)
\end{itemize}

B3. Dùng điều kiện đề cho \(a_0, a_1, \ldots \to d_1, d_2\)

B4. Thay \(d_1, d_2\) vào \(a_n\) ở B2 và kết luận.

    \subparagraph{\texorpdfstring{Bậc 3 \((k \geq 3)\) (tỉ lệ thi
thấp)}{Bậc 3 (k \textbackslash geq 3) (tỉ lệ thi thấp)}}\label{bux1eadc-3-k-geq-3-tux1ec9-lux1ec7-thi-thux1ea5p}

    \[
a_n = c_1 a_{n-1} + c_2 a_{n-2} + \ldots + c_k a_{n-k}
\]

    B1. Tính phương trình đặc trưng
\(r^k = c_1 r^{k-1} + c_2 r^{k-2} + \ldots + c_k\)

B2. (Nghiệm tổng quát)

\begin{itemize}
\item
  Nghiệm thực phân biệt
  \(r_1, r_2, r_k \to a_n = d_1r_1^n + d_2 r_2^n + \ldots + d_kr_k^n\)
\item
  \(t\) nghiệm thực phân biệt \(r_1, r_2, \ldots , r_t\) tương ứng với
  bội \(m_1, m_2, \ldots , m_t\)
\end{itemize}

\[\to a_n = (d_{10} + d_{11}n + \ldots + d_{1(m_1 -1)}n^{m_1 =1})r_1^n + \ldots + (d_{t0} + d_{t1}n + \ldots + d_{t(m_t -1)}n^{m_t -1})\]

    \section{Quan hệ}\label{quan-hux1ec7}

    \begin{itemize}
\tightlist
\item
  \((a,b) \in \mathcal{R}\)
\end{itemize}

\[
a\mathcal{R} b
\]

\begin{itemize}
\item
  \(\mathcal{R} = \{ (a,b) \mid a \text{ là ... } b \}\)
\item
  \(\mathcal{R} = \{ (.., ..),(.. , ..), .. \}\)
\end{itemize}

    \subsection{Các loại quan hệ}\label{cuxe1c-loux1ea1i-quan-hux1ec7}

    \subsubsection{Phản xạ}\label{phux1ea3n-xux1ea1}

    \[
(a,a) \in \mathcal{R}, \forall a \in A
\]

\begin{itemize}
\item
  Tính chất:

  \begin{itemize}
  \tightlist
  \item
    \(\leq\) Có phản xạ
  \item
    \(<\) Không phản xạ
  \item
    \(\mid\) (ước) Có phản xạ
  \end{itemize}
\end{itemize}

    \subsubsection{Đối xứng}\label{ux111ux1ed1i-xux1ee9ng}

    \(\forall a, b \in A\)

\[
a \mathcal{R} b \to b \mathcal{R} a
\]

    \subsubsection{Phản xứng}\label{phux1ea3n-xux1ee9ng}

    \(\forall a, b \in A\)

\[
(a \mathcal{R} b) \land (b \mathcal{R} a) \to (a = b)
\]

    \subsubsection{Bắc cầu}\label{bux1eafc-cux1ea7u}

    \(\forall a, b, c \in A\)

\[
(a \mathcal{R} b) \land (b \mathcal{R} c) \to a \mathcal{R} c
\]

    \subsection{Biểu diễn}\label{biux1ec3u-diux1ec5n}

    \(A = \{ 1,2,3,4\}\), \(B = \{ u, v, w\}\),
\(\mathcal{R} = \{ (1,u), (1,v),(2,w),(3,w),(4,u) \}\)

Cách biểu diễn

\[
\begin{array}{c|ccc}
 & u & v & w \\
\hline
1 & 1 & 1 & 0 \\
2 & 0 & 0 & 1 \\
3 & 0 & 0 & 1 \\
4 & 1 & 0 & 0 \\
\end{array}
\]

    \(A=\{ a_1, a_2, \ldots , a_m \}\),
\(B = \{ b_1, b_2 , \ldots , b_n \}\),
\(M_{\mathcal{R}} = [m_{ij}]_{m\times n}\)

\[
m_{ij} = \begin{cases} 0 \quad (a_i,b_j) \notin \mathcal{R} \\ 1 \quad (a_i,b_j) \in \mathcal{R} \end{cases}
\]

    \begin{itemize}
\item
  \(\mathcal{R}\) quan hệ trên \(A\), k/h \(M_{\mathcal{R}}\)
\item
  Các loại quan hệ:

  \begin{itemize}
  \item
    Phản xạ \(\to m_{ii} = 1\) (đường chéo chính)
  \item
    Đối xứng \(\to m_{ij} = m_{ji}, \forall i,j\)
  \item
    Phản xứng \$\to m\_\{ij\} = 0 \lor m\_\{ji\} = 0, \forall i,j, i
    \neq j \$
  \end{itemize}
\end{itemize}

    \subsection{Quan hệ tương
đương}\label{quan-hux1ec7-tux1b0ux1a1ng-ux111ux1b0ux1a1ng}

    \begin{itemize}
\item
  Thoả:

  \begin{itemize}
  \item
    Phản xứng
  \item
    Đối xứng
  \item
    Bắc cầu
  \end{itemize}
\item
  Lớp tương đương, \(\mathcal{R}\) quan hệ tương đương trên
  \(A, a \in A\) k/h:
  \([a]_{\mathcal{R}} = \overline{a} = \{ b \in A \mid b \mathcal{R} A \}\)
\item
  Định lý:

  \begin{itemize}
  \item
    \(a \mathcal{R} b \Rightarrow [a]_{\mathcal{R}} = [b]_{\mathcal{R}}\)
  \item
    \([a]_{\mathcal{R}} \neq [b]_{\mathcal{R}} \Rightarrow [a]_{\mathcal{R}} \bigcap [b]_{\mathcal{R}} = \varnothing\)
  \end{itemize}
\end{itemize}

    \subsection{Quan hệ thứ tự}\label{quan-hux1ec7-thux1ee9-tux1ef1}

    \begin{itemize}
\item
  Thoả:

  \begin{itemize}
  \item
    Phản xạ
  \item
    Đối xứng
  \item
    Bắc cầu
  \end{itemize}
\item
  Tập sắp thứ tự (Poset), k/h: \((A, \prec )\)
\item
  Sắp thứ tự:

  \begin{itemize}
  \item
    Toàn phần
  \item
    Bộ phận
  \end{itemize}
\item
  Trội và Trội trực tiếp
\item
  Tối tiểu và Tối đại
\item
  Phần tử nhỏ nhất và Phần tử lớn nhất
\end{itemize}


    % Add a bibliography block to the postdoc
    
    
    
\end{document}
